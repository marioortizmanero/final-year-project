% vim: spelllang=en

\begin{center}
{\LARGE \bfseries ABSTRACT}

\vspace{2.5cm}
\end{center}

Decision-making in many modern companies like Wayfair is based on the collection
and analysis of data within their systems. To do this efficiently on a massive
scale, it is essential to use high-performance tools such as Tremor, written
with Rust, asynchronous programming, and SIMD.

As Tremor evolves, its compilation times grow and, consequently, its development
experience deteriorates. This can be alleviated by implementing a plugin system
that divides its single binary into smaller, independently-compilable
components.

There are multiple available technologies for its development: interpreted
languages, WebAssembly, eBPF, inter-process communication, or dynamic loading.
However, many of them must be discarded for not meeting Tremor's efficiency
standards. Among the remaining alternatives, dynamic loading is chosen for being
the most usable and popular one.

Dynamic loading is impossible with types and functions declared with pure Rust,
because its Application Binary Interface (ABI) is not stable. It will be
necessary to convert its types to C's ABI, which is stable, and vice-versa. To
facilitate the process, existing libraries and language tools such as procedural
macros can be used.

Since dynamic loading is a very new ecosystem in Rust, it is necessary to
contribute in open source to a great amount of dependencies in order to
implement the functionality necessary for a plugin system.

The complexity of the project increased significantly compared to the original
plan, so even if functional, it does not attain some of the initial objectives,
mainly related to performance. However, it serves as a great base for future
versions of tremor that do include it in production, and it will continue to
evolve with the program.
