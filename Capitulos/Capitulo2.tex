% vim: spelllang=es

\chapter{Contexto}\label{sec:investigation}

La propuesta para el sistema de plugins asumía que se iba a implementar con un
método que cubriré posteriormente, denominado \emph{cargado dinámico}. Esto se
debe a razones de rendimiento, pero el método también incluye otros problemas
importantes, principalmente relacionados con seguridad. Por ello, es una buena
idea considerar las alternativas existentes para el PDK, en caso de que hubiera
alguno con la misma eficiencia pero menos vulnerabilidades.

\section{Requisitos}

La tecnología final escogida y el diseño del sistema de plugins tendrán que
cumplir una serie de requerimientos impuestos por el equipo de Tremor:

\begin{itemize}
    \item Debe ser posible añadir y quitar plugins tanto en el inicio del
        programa como durante su ejecución.

    \item Disponibilidad y madurez de la tecnología en el ecosistema de Rust.
        Debería ser viable en el largo plazo y no quedar obsoleto.

    \item Soporte multi-plataforma. Tremor actualmente da funciona en Windows,
        MacOS y Linux, así que debería estar disponible para al menos todas
        ellos.

    \item No debe tener un impacto excesivo en el rendimiento. El equipo de
        Tremor está dispuesto a sufrir una degradación de rendimiento siempre y
        cuando esta sea menor. La definición de `menor' es flexible;
        aproximadamente un máximo de 10-20\% de reducción en eficiencia para
        poderse incluir en producción. Copiar y serializar eventos, por ejemplo,
        implicaría la rotura de este requerimiento.

\end{itemize}

También existe una lista de tareas opcionales que se pueden tener en cuenta:

\begin{itemize}
    \item Maximizar la seguridad en lo posible, como se especifica en la Sección
        \ref{sec:security}.

    \item Debería ser retro-compatible e implementar algún tipo de versionado
        para evitar roturas inesperadas, como indica la Sección
        \ref{sec:compat}.

    \item Minimizar el esfuerzo necesario de reescribir los conectores para el
        nuevo sistema de plugins.

    \item Gestionar correctamente plugins con identificadores conflictivos.

    \item Desarrollar un sistema que compruebe las invariantes de los plugins y
        proporcione mecanismos de testing para desarrolladores de plugins.

    \item Considerar la generación de documentación para plugins y desarrollar
        herramientas para mejorar la experiencia de uso.

\end{itemize}

\section{Seguridad}\label{sec:security}

\subsection{Código unsafe}

Muchas de las tecnologías que se pueden aplicar para un sistema de plugins usan
código \unsafe. El código \unsafe consiste en una sección que ignora el modelo
de memoria y concurrencia de Rust, permitiendo por ejemplo el uso de punteros
nulos o referencias colgantes. Se suele usar para maximizar el control y
rendimiento del programa.

\begin{minted}{rust}
let p = std::ptr::null::<i32>(); // Puntero nulo a un entero
unsafe {
    // La lectura de `p` es imposible sin un bloque `unsafe`
    println!("Mi número: {}", *p); // Segmentation fault!
}
\end{minted}

Técnicamente, esto no es necesariamente un problema si la implementación está
autocontenida y su corrección es auditada exhaustivamente. Pero al perderse
garantías importantes proporcionadas por Rust, incrementa el coste de
mantenimiento de la librería y existen riesgos de vulnerabilidades.

Asegurarse de que la implementación es segura implica una cantidad
considerablemente mayor de trabajo, aun cuando existen herramientas de
verificación en tiempo de ejecución como \namecite{miri}. Este sería integrado
en Tremor en caso de tenerse que recurrir a \unsafe.

\subsection{Resiliencia a errores}

Rust no protege a sus usuarios de \leaks de memoria. De hecho, es tan sencillo
como llamar a la función \rust{mem::forget}. Si un plugin tuviera un \leak, el
proceso entero también se vería afectado; el rendimiento de Tremor se degradaría
con plugins no desarrollados incorrectamente. Algo similar podría suceder en
caso de que un plugin abortase, lo que terminaría la ejecución del programa por
completo.

Idealmente, Tremor debería poder detectar plugins que no rinden óptimamente y
pararlos antes de que sea demasiado tarde. La runtime debería poder continuar
corriendo aun cuando falle un plugin, posiblemente reiniciándolo y avisando al
usuario.

\subsection{Ejecución de código remota a través de plugins}

Uno de los casos más notorios se dio con Internet Explorer, que usaba COM y
ActiveX, los cuales no disponían de una \sandbox. Dicho mecanismo aisla por
completo parte del programa, de forma que no pueda leer ni escribir a memoria
externa (prohibiendo el acceso a información que no es suya), ni a recursos del
sistema (como ficheros). Por tanto, extensiones maliciosas para el navegador
podían ejecutar código arbitrario en la máquina en la que estuviera
instalado~\cite{iesandbox}. Este problema puede ser menos grave si solo se
instalan extensiones de confianza con firmas digitales, pero sigue siendo un
vector de ataque importante.

Se podría aplicar lo mismo a Tremor. El usuario del producto --- aquellos que
añadan plugins a su configuración --- es un desarrollador, que debería ser más
consciente sobre lo que incluye en sus proyectos. Sin embargo, en la práctica
esto no es cierto.

Podría compararse con cómo funcionan los administradores de paquetes como
\namecite{npm}. Su infraestructura se suele basar por completo en cadenas de
confianza; no hay nadie que te impida crear un paquete malicioso para ejecutar
código remoto o robar credenciales~\cite{npm1}\cite{npm2}. Los plugins son como
dependencias en este caso; tienen acceso completo a la máquina donde se
ejecutan, y por tanto no deberían ser de confianza por defecto.

Una mejora respecto a Node.js y npm sería algo como \namecite{deno}, una runtime
segura por defecto. Esto es posible gracias a \sandboxing y requiere que el
desarrollador configure manualmente, por ejemplo, el acceso al sistema de
ficheros o a la red. No es una solución infalible porque puede que los
desarrolladores acaben activando los permisos que necesitan sin pensarlo, pero
es un mecanismo similar a \unsafe: al menos te hace consciente de que estás en
terreno pantanoso y facilita la búsqueda de vulnerabilidades.

Se podría discutir que, realistamente, el programa va a ejecutarse casi siempre
en una máquina virtual o un contenedor, donde este problema no es tan peligroso.
Pero, ¿debería la seguridad del usuario recaer en el hecho de que el kernel está
aislado? Por no mencionar que un contenedor afecta mucho más al rendimiento que
algunos métodos de \sandboxing. Aunque la máquina por completo estuviera
aislada, seguiría habiendo una posibilidad de \leaks internos: el plugin de
Postgres tiene acceso a todo lo que esté usando el plugin de Apache Kafka, que
posiblemente tenga \emph{logs} sensitivos.

\section{Retro-compatibilidad}\label{sec:compat}

Deberá incluirse algún tipo de gestión de versiones del sistema de plugins. Es
probable que la interfaz de Tremor cambie con frecuencia, lo que romperá plugins
basados en versiones previas. Si un plugin recibe una estructura de la runtime,
pero esta estructura perdiese uno de sus campos en una nueva versión, se estará
invocando comportamiento no definido por intentar acceder a un campo ahora
inexistente.

\subsection{Posibles soluciones}

La idea más sencilla para arreglar problemas con retrocompatibilidad es
serializar y deserializar los datos con un protocolo flexible, en vez de usando
su representación binaria directamente. Si se usara un protocolo como JSON para
comunicarse entre la runtime y los plugins, añadir un campo no rompería nada y
eliminar uno puede ocurrir mediante un proceso de deprecación. Por desgracia,
esto implicaría una degradación en el rendimiento que posiblemente no interese
en la aplicación.

Otras medidas preventivas más elaboradas para representaciones binarias
incluyen~\cite{swiftabi}:

\begin{itemize}
    \item Reservar espacio en la estructura para uso futuro.

    \item Hacer la estructura un tipo opaco, es decir, que sólo se puede acceder
        a sus campos con llamadas a funciones, en lugar de directamente en
        memoria.

    \item Declarar la estructura con un puntero con sus datos de la ``segunda
        versión'' (lo cual sería opaco en la ``primera versión'').

\end{itemize}

\subsection{Control de versiones}

Hay casos donde un error inevitable. Es posible que Tremor quiera reescribir
parte de su interfaz o finalmente eliminar una funcionalidad deprecada sin tener
que preocuparse por romper todos los plugins desarrollados previamente.

Para ello, los plugins deben incluir metadatos sobre las diferentes versiones
del compilador de Rust o de la interfaz para la que fue desarrollado. Antes de
que la runtime los cargue, se podrá comprobar su compatibilidad, en vez de
romperse de formas misteriosas.
