% vim: spelllang=es

\begin{center}
{\LARGE \bfseries RESUMEN}

\vspace{2.5cm}
\end{center}

La toma de decisiones de muchas empresas modernas como Wayfair se basa en la
recolección y análisis de datos de sus sistemas. Para llevarlo a cabo de forma
eficiente en una escala masiva, deben usarse herramientas de alto rendimiento
como Tremor, escrito con Rust, programación asíncrona con hilos y SIMD.

A medida que Tremor evoluciona, sus tiempos de compilación crecen y,
consecuentemente, se deteriora la experiencia de desarrollo. Esto se puede
aliviar implementando un sistema de plugins que divide su único binario en
componentes más pequeños compilables independientemente.

Existen múltiples tecnologías disponibles para su desarrollo: lenguajes
interpretados, WebAssembly, eBPF, comunicación inter-proceso o cargado dinámico.
Sin embargo, muchas de ellas deben descartarse por no cumplir los estándares de
eficiencia de Tremor. Entre las alternativas restantes, se escoge cargado
dinámico por ser la más usable y popular.

El cargado dinámico es imposible con tipos y funciones declarados con Rust puro,
ya que su Interfaz Binaria de Aplicación (ABI) no es estable. Es necesario
convertir los tipos al ABI de C, que sí es estable, y viceversa. Para facilitar
el proceso, se pueden aprovechar librerías existentes y herramientas del
lenguaje como macros procedurales.

Dado que el cargado dinámico en Rust es un ecosistema muy nuevo, se debe
contribuir en código abierto a gran cantidad de dependencias de Tremor para
implementar la funcionalidad necesaria para un sistema de plugins.

La complejidad del proyecto incrementa significativamente respecto al plan
original, por lo que, aunque funcional, la implementación no alcanza alguno de
los objetivos iniciales, principalmente relacionados con el rendimiento. Sin
embargo, sirve como una buena base para futuras versiones de Tremor que sí que
lo incluyan en producción y continuará evolucionando con el programa.
