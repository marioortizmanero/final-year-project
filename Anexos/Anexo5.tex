\chapter{Fases de desarrollo}\label{annex:hours}

Este anexo lista las fases en las que se llevó a cabo el proyecto y las horas
invertidas en cada una de ellas. Se incluye el periodo en el que se realizaron,
comenzando en abril de 2021 con la propuesta a Tremor. Una vez aceptado, el
inicio oficial se dio en agosto. Adicionalmente, junto a sus descripciones se
añaden uno o más enlaces relacionados con la tarea.

\setlist{topsep=2pt,nosep}

\begin{longtable}[H]{| m{9.7cm} | P{1.3cm} | P{3.4cm} |}
\caption{Fases de desarrollo del proyecto}\\

\hline
\textbf{Fase}
    & \textbf{Horas}
    & \textbf{Periodo} \\
\hline
\endhead

Propuesta a Tremor: incluye una introducción sobre quién soy, qué proyecto
quiero hacer y una breve planificación de la metodología a seguir.

\vspace{4mm}
\emph{\url{https://nullderef.com/blog/gsoc-proposal/}}
    & 6
    & Abr.~\textquotesingle21 \\

\hline
Investigación inicial de las tecnologías disponibles para el sistema de
plugins y discusión con el equipo. Esto también formó parte de la propuesta,
aunque de forma no oficial.

\vspace{4mm}
\emph{\url{https://nullderef.com/blog/plugin-tech/}}
    & 36
    & Abr.~\textquotesingle21 -- May.~\textquotesingle21 \\

\hline
Aceptación del proyecto. Introducción a Tremor y a su equipo.

\vspace{4mm}
\emph{\url{https://nullderef.com/blog/plugin-start/}}
    & 8
    & Ago.~\textquotesingle21 \\

\hline
Implementación de prototipos para el sistema de plugins, y medidas de
rendimiento iniciales. También incluye otros experimentos menores para encontrar
el mejor método para tener programación asíncrona o genéricos.

\vspace{4mm}
\emph{\url{https://github.com/marioortizmanero/pdk-experiments}}

\vspace{4mm}
\emph{\url{https://nullderef.com/blog/plugin-start/}}

\vspace{4mm}
\emph{\url{https://nullderef.com/blog/plugin-dynload/}}
    & 51
    & Ago.~\textquotesingle21 -- Feb.~\textquotesingle22 \\

\hline
Investigación del funcionamiento interno de Tremor. Diseño de diagramas de
secuencia.

\vspace{4mm}
\emph{\url{https://nullderef.com/blog/plugin-dynload/}}
    & 13
    & Sep.~\textquotesingle21 \\

\hline
Investigación en detalle de cargado dinámico en Rust.

\vspace{4mm}
\emph{\url{https://nullderef.com/blog/plugin-dynload/}}
    & 22
    & Sep.\textquotesingle21 -- Oct.~\textquotesingle21 \\

\hline
Aprendizaje de la librería \abistable.

\vspace{4mm}
\emph{\url{https://nullderef.com/blog/plugin-abi-stable/}}
    & 17
    & Oct.~\textquotesingle21 -- Nov.~\textquotesingle21 \\

\hline
Soporte de \abistable para programación asíncrona con \code{async_ffi}.

\vspace{4mm}
\emph{\url{https://github.com/oxalica/async-ffi/pull/10}}
    & 15
    & Nov.~\textquotesingle21 -- Ene.~\textquotesingle22 \\

\hline
Primer diseño e implementación del sistema de plugins.

\vspace{4mm}
\emph{\url{https://github.com/tremor-rs/tremor-runtime/pull/1434}}
    & 64
    & Oct.~\textquotesingle21 -- Abr.~\textquotesingle22 \\

\hline
Mediciones de rendimiento.

\vspace{4mm}
\emph{\url{https://github.com/marioortizmanero/nullderef.com/pull/54} (artículo
aún no publicado)}
    & 14
    & Ene.~\textquotesingle22 -- Jun.~\textquotesingle22 \\

\hline
Resolución de los problemas de covarianza y subtipado.

\vspace{4mm}
\emph{\url{https://github.com/rodrimati1992/abi_stable_crates/issues/75}}
    & 26
    & Ene.~\textquotesingle22 -- Mar.~\textquotesingle22\\

\hline
Añadir soporte de la librería \code{halfbrown} para \abistable.

\vspace{4mm}
\emph{\url{https://github.com/rodrimati1992/abi_stable_crates/pull/83}}
    & 19
    & Mar.~\textquotesingle22 -- Jun.~\textquotesingle22 \\

\hline
Segunda versión del sistema de plugins con las dos mejoras anteriores de
rendimiento.

\vspace{4mm}
\emph{\url{https://github.com/tremor-rs/tremor-runtime/pull/1597}}
    & 26
    & May.~\textquotesingle22 -- Jun.~\textquotesingle22 \\

\hline
Documentación final para que Tremor pueda continuar con el desarrollo del
proyecto.

\vspace{4mm}
\emph{\url{https://github.com/tremor-rs/tremor-runtime/pull/1597}}

\vspace{4mm}
\emph{\url{https://github.com/marioortizmanero/tremor-runtime/projects/1}}
    & 6
    & Jun.~\textquotesingle22 \\

\hline
Memoria del Trabajo de Fin de Grado.
    & 41
    & May.~\textquotesingle22 -- Jun.~\textquotesingle22 \\

\hline
\textbf{Total}
    & \textbf{364}
    & \textbf{Abr.~\textquotesingle21 --
Jun.~\textquotesingle22} \\

\hline
\end{longtable}
