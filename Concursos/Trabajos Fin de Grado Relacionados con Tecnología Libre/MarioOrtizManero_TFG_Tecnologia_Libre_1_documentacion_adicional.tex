% vim: spelllang=es

% Página:
% https://osluz.unizar.es/
%
% Bases:
% https://osluz.unizar.es/sites/default/files/BasesConcursoTFG1ed.pdf

\documentclass[a4paper,12pt,twoside,hidelinks,openright]{article}

% \usepackage[T1]{fontenc}
\usepackage[spanish]{babel}
\usepackage[utf8]{inputenc}
\usepackage{hyperref}
\usepackage{enumitem}% http://ctan.org/pkg/enumitem

% Easier to read font
\usepackage{fourier} % For math
\usepackage{fontspec}
\setmainfont{Heuristica} % For the text
\setmonofont{Liberation Mono} % For the code

% Less margin between lists, otherwise after overriding \parskip and etc it's
% too much.

%	CONFIGURACIÓN DE PÁGINA

\setlength{\paperwidth}{21cm}          % Ancho de página
\setlength{\paperheight}{29,7cm}       % Alto de página
\setlength{\textwidth}{15.5cm}         % Ancho de zona con texto
\setlength{\textheight}{24.6cm}        % Ancho de zona con texto
\setlength{\topmargin}{-1.0cm}         % Margen superior
                                      
\setlength{\oddsidemargin}{0.46cm}     % Margen izquierdo 
\setlength{\evensidemargin}{0.46cm}    

% Less margin between lists, otherwise after overriding \parskip and etc it's
% too much.
\usepackage{enumitem}
\setlist{topsep=0pt}

\begin{document}

% PORTADA
\newpage

\title{Documentación adicional para\\I Concurso ``Trabajos Fin de Grado
Relacionados con Tecnología Libre''\\(TFGTL 2023)}
\date{}

\pagenumbering{Roman}

% Párrafos de forma más convencional, me parece más fácil leerlo así.
\begingroup
\setlength{\parskip}{\baselineskip}%
\setlength{\parindent}{0pt}%

\maketitle

\textbf{Acceso libre a todo el código con indicación de repositorio libre, si lo
hay}:\\
La implementación de Tremor y de todas sus herramientas acompañantes se
encuentra en la siguiente organización de GitHub:\\
\url{https://github.com/tremor-rs}.\\ Todos sus repositorios usan la licencia
libre \emph{Apache License 2.0}.\\
La implementación del sistema de plugins como tal se puede encontrar aquí:\\
\url{https://github.com/tremor-rs/tremor-runtime/pull/1597}\\
y una segunda versión en\\
\url{https://github.com/marioortizmanero/tremor-runtime/pull/15}.

También está disponible el código fuente de la memoria y algunos archivos
adicionales en \url{https://github.com/marioortizmanero/final-year-project}.

Notar que una parte muy importante a tener en cuenta del trabajo realizado
consiste en diversas contribuciones a otros proyectos usados por Tremor. Existe
una lista detallada con enlaces en el anexo quinto de la memoria,
``Contribuciones de código abierto'', o en
\url{https://nullderef.com/blog/plugin-end/#os}. Para otros tipos de
contribuciones, como artículos o charlas, se puede consultar la última sección
de este documento.

\textbf{Generalidad de la propuesta, que el software realizado pueda ser
utilizado por un grupo amplio de usuarios}:\\
Tremor, y por tanto, el proyecto que he llevado a cabo, puede ser usado por
cualquier empresa que procese eventos internos o externos. La empresa más grande
que lo utiliza actualmente se trata de Wayfair. En el último año, esta ha
colaborado con más de 11.000 proveedores globales para ofrecer 14 millones de
ítems, alcanzando un procesado diario medio de unos 100TB de datos.  Además, el
número de usuarios provenientes de organizaciones más pequeñas sigue creciendo,
incluyendo empresas como RedPanda o TDEngine, entre otras:

\begin{itemize}
    \item \url{https://www.tremor.rs/blog/2021/11/01/wayfair-case-studies}
    \item \url{https://www.tremor.rs/blog/2022/06/15/tdengine-colaboration}
    \item \url{https://www.tremor.rs/blog/2021/12/13/redpanda}
\end{itemize}

Tremor también dispone de una activa comunidad en Discord con casi 200 usuarios
y contribuidores:

\begin{itemize}
    \item \url{https://discord.com/invite/Wjqu5H9rhQ}
\end{itemize}

\textbf{Quiénes son los potenciales usuarios}:\\
La característica implementada es parte del núcleo del programa, es decir que
será utilizado por cualquier usuario o desarrollador de Tremor.

Por un lado, los contribuidores se beneficiarán por medio de una reducción muy
considerable en los tiempos de compilación, facilitando así el proceso de
desarrollo. Además, esto implica una completa modularización y limpieza del
código.

Por otro lado, el sistema de plugins proporcionará a los usuarios una mayor
flexibilidad, permitiéndoles implementar sus propias funcionalidades sobre
Tremor. Esto incluye, por ejemplo, la capacidad de recibir o enviar eventos de
un sistema interno, cuyo soporte no se podría o debería añadir públicamente en
el código de Tremor.

\textbf{Impacto demostrable hasta la actualidad (número de descargas,
actualizaciones…)}:\\
El prototipo funcional del sistema de plugins desarrollado no ha podido lanzarse
aún en una nueva versión de Tremor, pero sirve de una base muy sólida a partir
de la cual el equipo puede continuar mejorando. Además de las 13 contribuciones
dentro de la organización de Tremor, se crearon 33 pull requests e issues a un
total de 10 dependencias de Tremor, también ayudando al ecosistema general.

Los artículos del blog personal que intentan mejorar la situación del nicho de
sistemas de plugins en Rust han alcanzado 36.565 visitas. Las dos charlas suman
en la actualidad más de 1.000 visualizaciones, sin incluir las del directo.

\textbf{Difusión demostrable del trabajo realizado: artículo, blog, repositorio,
tutorial,...}:\\
Se escribieron 6 artículos en mi blog personal, bajo la licencia libre \emph{GPL
v3}, donde entro en gran detalle sobre las decisiones tomadas y el
funcionamiento interno. Pueden encontrarse en la siguiente página:
\url{https://nullderef.com/series/rust-plugins/}. Cada uno es de unos 30 minutos
de lectura.

Los artículos incluyen un repositorio adicional para explicar con ejemplos
concretos las diferentes alternativas disponibles para desarrollar un sistema de
plugins: \url{https://github.com/marioortizmanero/pdk-experiments}. Actualmente
cuenta con 4 forks y 20 estrellas en GitHub.

Finalmente, se realizaron dos charlas, donde se cubría el trabajo realizado de
forma más visual y amena:

\begin{itemize}
    \item TremorCon 2022 (\emph{a posteriori de entregar el TFG, recomendada
        visualización}):\\
        \url{https://nullderef.com/blog/tremorcon22/}\\
        Tremor dispone de su propia conferencia anual, donde se comparten las
        últimas actualizaciones de contribuidores y se muestran usos reales de
        la herramienta. En esta edición, participé dando una charla de 30
        minutos donde explicaba mi trabajo desde el punto de vista del lenguaje
        de programación Rust, incluyendo algunos detalles técnicos.

    \item LFX Mentorship Showcase 2022:\\
        \url{https://youtu.be/htLCyqY0kt0?t=3166}\\
        La Fundación de Linux (\emph{LFX}) cuenta con un programa de mentorías
        llamado \emph{LFX Mentorship}, a través del cual pude contribuir a
        Tremor. Como parte de este programa, organizan una conferencia para que
        los participantes cuenten su experiencia, especialmente para futuros
        interesados. Opté por participar, dando una charla más general de 15
        minutos.

\end{itemize}


\end{document}
