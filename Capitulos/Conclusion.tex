% vim: spelllang=es

\chapter{Conclusiones y trabajo futuro}

\section{Concusiones}

Como se ha explicado, la complejidad del proyecto resultó ser mucho mayor que lo
esperado por problemas con el ABI. Por tanto, resultó imposible desarrollar en
el tiempo disponible un sistema de plugins tan completo y eficiente como se
especificaba.

La última versión del sistema de plugins es funcional y, mediante contribuciones
de código abierto, ha hecho posible su futura inclusión en una versión de
Tremor. Muchas de las librerías usadas no disponían de la funcionalidad
necesaria para este proyecto, así como \code{async_ffi}, \code{abi_stable},
\code{halfbrown} o \code{simd-json}.

El problema principal tiene que ver con el rendimiento. Dada la naturaleza de
Tremor, es un requerimiento imprescindible para poderlo incluir en una futura
versión. Tras unas mediciones iniciales, se calculó que su inclusión reducía el
rendimiento del programa un 30\%, así que los siguientes pasos consistieron en
reducir dicha cifra.

TODO: incluir todas las figuras de benchmarks con las explicaciones

\section{Futuro}

\begin{itemize}
    \item Optimización

    \item Implementación para otros componentes de Tremor: códecs,
        preprocesadores, postprocesadores, operadores, funciones, extractores,
        etc.

    \item Mejora de la experiencia de usuario: proyectos template, ejemplos,
        documentación exhaustiva, frameworks de testing.

    \item Carga de plugins a petición del usuario

    \item Eliminación de plugins en tiempo de ejecución. Resiliencia a errores

    \item Registro centralizado de plugins (similar a un repositorio maven)

    \item Paquetes de plugins

    \item Gestión de versiones avanzada???

\end{itemize}

\section{Valoración personal}
