% vim: spelllang=es

\chapter{Conclusiones y trabajo futuro}

\section{Concusiones}

La complejidad del proyecto ha resultado ser mucho mayor de lo esperado,
principalmente por el malentendido sobre la estabilidad del ABI de Rust. Por
tanto, ha resultado imposible desarrollar en el tiempo disponible un sistema de
plugins tan completo y eficiente como se especificaba inicialmente.

El problema principal tiene que ver con el rendimiento. Dada la naturaleza de
Tremor, es un requerimiento imprescindible para poderlo incluir en producción.
Tras las pruebas realizadas en el Anexo~\ref{annex:benchmarks}, se ha calculado
que el sistema de plugins reduce el rendimiento un 30\%.

No obstante, la última versión del sistema de plugins es perfectamente funcional
y, mediante su investigación, el diseño de su arquitectura y contribuciones de
código abierto, se ha hecho posible su inclusión en una futura versión de
Tremor.

Parte de esta ralentización en el desarrollo se debe también a Rust. Al ser un
lenguaje tan inmaduro es frecuente encontrar documentación pobre o librerías
incompletas. Muchas de las \crates usadas no disponían inicialmente de la
funcionalidad necesaria para un sistema de plugins, como \code{async_ffi},
\code{abi_stable}, \code{halfbrown} o \code{simd-json}. Se han resuelto
problemas importantes en el entorno, extendiendo el soporte para el ABI de C,
resolviendo tipos con varianzas inflexibles y elaborando conversiones de tipos
no triviales, todo ello manteniendo la máxima seguridad y eficiencia posible.
Con esfuerzos como estos, el desarrollo de proyectos similares en el futuro
resultará mucho más accesible.

\section{Futuro}

Se ha documentado tanto el proceso seguido como lo que queda pendiente, de forma
que el equipo de Tremor pueda continuar trabajando en el sistema de plugins para
su futuro lanzamiento. Sin embargo, aun después de esto el PDK nunca parará de
evolucionar: su uso se extenderá en la base de código y se perfeccionarán otras
características con el tiempo. Algunas ideas son las siguientes:

\begin{itemize}
    \item \textbf{Mejoras de rendimiento}: el enfoque principal para el primer
        lanzamiento del PDK. Esto incluye la realización de \emph{benchmarks}
        más variados y realistas, y el soporte de \abistable en más librerías.

    \item \textbf{Soporte de otros componentes de Tremor}: el PDK únicamente se
        implementa para los conectores, pero también podría funcionar con
        códecs, preprocesadores, postprocesadores, operadores, funciones,
        extractores, etc.

    \item \textbf{Refinamiento de la experiencia de usuario}: creación de
        proyectos modelo como base para plugins nuevos, ejemplos de uso, macros,
        documentación exhaustiva y de más alto nivel, frameworks de testing,
        etc.

    \item \textbf{Carga de plugins a petición del usuario}: además de poder
        cargar los plugins al inicio del programa, sería especialmente útil
        solicitar su carga durante la ejecución. Se podría elaborar un nuevo
        método de configuración iterativo, en el que se cargan y configuran los
        plugins uno a uno, y finalmente se exporta la composición final.

    \item \textbf{Paquetes de plugins}: en ciertos casos, sería más conveniente
        exportar un plugin que implemente más de un componente. Por ejemplo,
        podrían juntarse los conectores de TCP y UDP en único plugin, dado que
        probablemente compartan partes de su código y dependencias.

    \item \textbf{Gestión de versiones alternativas}: la implementación de
        \abistable para comprobar las versiones es rudimentaria y poco
        eficiente; se limita a comprobarlos todos recursivamente. Otra opción
        más simple sería únicamente comprobar una cadena con el versionado
        global para la interfaz, por ejemplo.

    \item \textbf{Registro centralizado de plugins}: en el futuro a largo plazo,
        se podría desarrollar una funcionalidad similar a los repositorios de
        Maven o Cargo. Allí se podrían guardar todos los plugins de la comunidad
        para gestionarlos automáticamente.

    \item \textbf{Eliminación de plugins en tiempo de ejecución}: es
        especialmente complejo de implementar, dado que \abistable
        explícitamente no lo soporta. Sin embargo, esto mejoraría
        considerablemente la resiliencia a errores, siendo posible reiniciar
        plugins completamente.

\end{itemize}

\section{Valoración personal}

Pese a las situaciones de frustración frente a todos los errores y bloqueos que
he encontrado en el camino, ha sido una experiencia extraordinaria. Matthias
bromeó una vez con que ``El infierno de debugging es importante para el
desarrollo de personaje'', y creo que tiene toda la razón. Enfrentarme a errores
que no sabía ni cómo abordar me ha enseñado mucho sobre Rust, y lo que es más
importante, sobre desarrollo de software en general.

Estoy muy satisfecho con haber conseguido lo que he conseguido, y aún más por
haberlo poder hecho junto al increíble equipo que es el de Tremor. Trabajar con
ellos me ha ayudado a descubrir qué quiero hacer tras la graduación, y con qué
tipo de empresa y personas quiero trabajar. Me mantendré en contacto con ellos
para seguir el progreso del sistema de plugins.
