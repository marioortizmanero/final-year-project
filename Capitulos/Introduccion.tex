% vim: spelllang=es

\chapter{Introducción}
\pagenumbering{arabic}

\section{Contexto}

% TODO: quizá algunos footnotes podrían cambiarse por referencias aquí
Este proyecto se ha realizado en colaboración con
\emph{Tremor}\footnote{\url{https://tremor.rs}}, un sistema de procesado de
eventos de alto rendimiento, escrito en el lenguaje de programación
\emph{Rust}\footnote{\url{https://www.rust-lang.org}}. Tremor es un programa de
código abierto bajo la fundación \emph{Cloud Native Computing
Foundation~(CNCF)}\footnote{\url{https://www.cncf.io/}}, que es también parte de
la organización \emph{Linux
Foundation~(LFX)}\footnote{\url{https://www.linuxfoundation.org/}}.

Formalmente, el trabajo se ha llevado a cabo gracias a la iniciativa \emph{LFX
Mentorship}, con el título ``CNCF -- Tremor: Add plugin support for tremor
(PDK)''\footnote{Página oficial de la iniciativa:
\url{https://mentorship.lfx.linuxfoundation.org/project/b90f7174-fc53-40bc-b9e2-9905f88c38ff}}\footnote{\emph{Tracking
issue} en GitHub:
\url{https://github.com/tremor-rs/tremor-runtime/issues/791}}\footnote{RFC en la
documentación de Tremor:
\url{https://www.tremor.rs/rfc/accepted/plugin-development-kit/}}. Esta
iniciativa promueve el aprendizaje de desarrolladores de código abierto,
proporcionando una plataforma transparente y facilitando un sistema de pagos.

Finalmente, \emph{Wayfair} es una empresa estadounidense de comercio digital de
muebles y artículos del hogar\footnote{\url{https://www.wayfair.com/}}.
Actualmente, ofrece 14 millones de ítems de más de 11.000 proveedores
globales~\cite{wayfairItems} y es el principal financiador tanto de Tremor como
de este proyecto.

\section{Objetivo}

La tarea a llevar a cabo es la implementación de un sistema de plugins,
denominado \emph{Plugin Development Kit~(PDK)}, para la base de código ya
existente en Tremor.

Esto es una tarea no-trivial, dado que Rust no tiene un \emph{Application Binary
Interface~(ABI)} estable. Es decir, que si se compila la \emph{runtime} (el
binario principal encargado de cargar funcionalidad externa) y los
\emph{plugins} (los binarios individuales con la funcionalidad) de forma
separada, no hay garantía de que la representación binaria de los datos o la
convención de llamada a funciones --- entre otros --- sea la misma.

Esto implica que \emph{dynamic loading} es imposible de forma segura puramente
con Rust, debiéndose recurrir a otro ABI que sí sea estable, como el del
lenguaje de programación C. Por tanto, se deben escribir \emph{bindings} (la
definición de la interfaz compartida entre runtime y plugins) completas en C y
transformar tipos de Rust a C y viceversa cuando se interactúe con plugins.

\section{Motivación}

\subsection{Tiempos de compilación}

% TODO: incluir modo release?
Actualmente, el problema más importante en Tremor es sus tiempos de compilación.
En un ordenador de gama media de \~{}600 € como el Dell Vostro 5481, compilar el
binario \code{tremor} desde cero requiere de más de 7 minutos en modo debug.
Incluso en el caso de cambios incrementales (una vez las dependencias ya han
sido compiladas), hay que esperar unos 10 segundos. Esto no es una buena
experiencia de desarrollo e impide que nuevos programadores se unan a la
comunidad de Tremor.

Debido a la naturaleza del programa, este problema solo empeorará con el tiempo.
Tremor debe tener soporte para un gran número de protocolos (e.g., TCP o UDP),
software (e.g., Kafka o PostgreSQL) y codecs (e.g., JSON o YAML). El número de
dependencias continuará incrementando hasta que imposibilite la creación de
nuevas prestaciones en Tremor.

Los problemas relacionados con tiempos de compilación excesivamente largos no se
limitan a Tremor. Es uno de las mayores críticas que recibe Rust y un 61\% de
sus usuarios declaran que aún se necesita trabajo para mejorar la
situación~\cite{rustsurvey}.

\subsection{Modularidad}

Otra ventaja que provee un sistema de plugins es modularidad; ser capaz de
tratar la runtime y los plugins de forma separada suele resultar en una
arquitectura más limpia~\cite{baldwin2000design}. También hace posible el
desacoplamiento del ejecutable y sus componentes; algunas dependencias tienen un
ciclo de versionado más rápido que otras y generalmente es más conveniente
actualizar únicamente un plugin, en lugar del programa por completo.

\subsection{Aprender de otros}

Otros proyectos maduros con características similares a las de Tremor, como
\textcite{nginx} o \textcite{apachehttpserver}, llevan beneficiándose de un
sistema de plugins desde hace mucho. Informan mejorías en flexibilidad,
extensibilidad y facilidad de
desarrollo~\cite{nginxPluginsAdvantages}\cite{apachePluginsAdvantages}. Aunque
las desventajas también mencionen un pequeño impacto en el rendimiento y la
posibilidad de caer en un \emph{dependency hell}, sigue siendo una buena idea al
menos considerarlo para Tremor.

\section{Metodología}

\subsection{Organización}

El proyecto ha tenido una duración de unos 10 meses, comenzando en agosto de
2021 y terminando en mayo/junio de 2022. Su realización ha sido completamente
remota y con horarios muy flexibles. Se usó el servidor de Discord de
Tremor\footnote{\url{https://discord.com/invite/Wjqu5H9rhQ}} como plataforma
principal para comunicarse, tanto por texto como por videollamada. Se programó
una llamada por semana, en la que explicaba mi progreso y recibía ayuda de mis
mentores en caso de que me hubiera quedado atascado en algún momento.

% TODO: update Matthias' title
Disponía de tres mentores, que me guiaban en el proceso de desarrollo: Darach
Ennis (\emph{Principal Engineer and Director of Tremor Project}), Matthias Wahl
(\emph{Staff Engineer}) y Heinz N. Gies (\emph{Senior Staff Engineer}), todos
empleados por Wayfair.

La organización de forma más estructurada para las tareas que tenía pendientes,
en las que estaba trabajando en ese momento, y las que ya había realizado, se
basó principalmente en un Kanban en
GitHub\footnote{\url{https://github.com/marioortizmanero/tremor-runtime/projects/1}}.

\subsection{Desarrollo}

Para reducir el coste de desarrollo y asegurarse de que el proceso sea
completamente seguro (en memoria y concurrencia), el sistema de plugins
aprovecha librerías existentes en Rust y herramientas como macros procedurales.
El sistema de compilación usado es solución oficial de Rust: Cargo, que también
incluye un \emph{formatter}, \emph{linter}, y extensiones instalables creadas
por la comunidad. Adicionalmente, existe una gran cantidad de tests y
\emph{benchmarks} que se han de tener en cuenta para mantener el \emph{Code
Coverage} (la cantidad de código cubierta por los tests) y el rendimiento.

\subsection{Recursos públicos}

% TODO: actualizar '5' cuando suba el nuevo
% TODO: actualizar minutos de lectura
% TODO: buscar otra métrica que no sean horas de lectura, no creo que sea buena
% manera de venderlo
Este trabajo está disponible públicamente al completo. Además, a medida que he
investigado e implementado el sistema de plugins, he ido escribiendo todo en mi
blog personal, \emph{NullDeref}. Dispone de una serie con un total de 6
artículos, cada uno de unos 30 minutos de lectura. Gran parte del contenido de
este documento se obtiene de ahí, omitiendo en lo posible detalles específicos
al código, dado que aquí no se asume familiaridad con Rust. La organización
también difiere considerablemente, puesto que los artículos se escribieron a
medida que se realizaba el proyecto, resultando en una estructura menos
estricta. Esto toma un formato de tesis, mientras que el blog cuenta la historia
cronológicamente y sirve mejor como un tutorial para alguien que quiera
implementar un sistema de plugins propio.

\begin{itemize}
    \item El repositorio de GitHub para el binario de Tremor:\\
        \url{https://github.com/tremor-rs/tremor-runtime}
    \item Mi \emph{fork}, con ramas adicionales usadas durante el desarrollo:\\
        \url{https://github.com/marioortizmanero/tremor-runtime}
    \item Mi repositorio con experimentos antes de implementar la versión
        definitiva:\\
        \url{https://github.com/marioortizmanero/pdk-experiments}
    \item La serie de artículos en mi blog personal:\\
        \url{https://nullderef.com/series/rust-plugins/}.
\end{itemize}

TODO: debería ser más específico sobre las horas? Mucha cuenta no he llevado
pero vamos sí que estoy segurísimo de que han sido 300 horas y mucho más
también.

TODO: quizá hacer una tabla que relacione cada sección del índice con los
capítulos relacionados de nullderef? Pero eso ya lo pondría en un anexo.
