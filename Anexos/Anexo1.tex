\chapter{Guía de Rust}\label{annex:rust}

\section{Primeros pasos}

Comenzando por el clásico ``Hola Mundo'', se incluyen algunos ejemplos de cómo
es la sintaxis de Rust más básica. Los binarios o librerías en Rust reciben el
nombre de \crate. Nuestra \crate se podría ejecutar fácilmente con \emph{Cargo},
el administrador de dependencias oficial, o específicamente con el comando
\code{cargo run}.

\begin{minted}{rust}
fn main() {
    println!("Hello World!");
}
\end{minted}

\code{main} es nuestra función principal, que invoca al macro \code{println!}
para escribir por pantalla. Notar que la invocación de macros, a diferencia de
funciones, requiere un \rust{!} al final.

\section{Conceptos principales}

Los bloques básicos (\rust{if}, \rust{else}, \rust{while}, \rust{for}) son muy
similares a en otros lenguajes. También existe \rust{match}, que permite extraer
patrones de variables.

\begin{minted}{rust}
fn factorial(i: u64) -> u64 {
    match i {
        // Primer caso: i = 0
        0 => 1,
        // El resto de casos, asignado a una variable `n`
        n => n * factorial(n-1)
    }
}
\end{minted}

Uso de variables y métodos:

\begin{minted}{rust}
fn main() {
    // Declaración de una variable, cuyo tipo se infiere
    // automáticamente.
    let my_number = 1234;
    // Declaración de una variable con un tipo especificado
    // manualmente. Notar que se puede usar el mismo nombre, y la
    // variable anterior será destruida.
    let my_number: i32 = 4321;
    // Invocación de la función estática (constructor) `new` dentro
    // del tipo `String`. El uso de `mut` indica que la instancia del
    // tipo se puede modificar. Funciona de forma inversa a C++, que
    // por defecto es mutable y `const` indica que *no* se puede
    // modificar.
    let mut my_str = String::new();
    // Invocación del método `push` de `my_str`, que añade un
    // carácter al final de la cadena.
    my_str.push('a');
}
\end{minted}

Otros componentes principales de Rust son:

\begin{itemize}
    \item Estructuras de datos:

\begin{minted}{rust}
struct MessageA {
    // Campo público con una cadena de caracteres
    pub text: String,
    // Campo privado con un entero
    user_id: i32,
}
\end{minted}

\begin{minted}{rust}
// Sin nombres de campos; se pueden acceder con `data.0`
// y `data.1`, respectivamente.
struct MessageB(pub String, i32);
\end{minted}

    \item Enumeraciones, que también permiten contener datos:

\begin{minted}{rust}
enum MessageC {
    Join,
    Text(String, i32),
    Leave(i32),
}
\end{minted}

    \item \emph{Traits}, similares a las interfaces de Java en el sentido de que
        son una serie de requerimientos y que un tipo puede implementar
        múltiples \traits, pero también permiten implementaciones por defecto:

\begin{minted}{rust}
trait Sender {
    // Los métodos requieren especificar `self` explícitamente,
    // que es lo mismo que `this` en Java o C++. En este caso,
    // `&send` tomará una referencia al tipo que implemente
    // `Sender`. También podría ser una referencia mutable con
    // `&mut self`, o el mismo tipo con `self`.
    fn send(&self, msg: String);

    // Implementación por defecto.
    fn send_twice(&self, msg: String) {
        self.send(msg);
        self.send(msg.clone());
    }
}
\end{minted}

        Y para implementar un trait para un tipo:

\begin{minted}{rust}
impl Sender for MessageC {
    fn send(&self, msg: String) {
        match self {
            Join => println!("Joined"),
            Text(txt, id) => println!("{} sent: {}", id, txt),
            // Las variables `_` son ignoradas
            Leave(_) => println!("Left"),
        }
    }

    // `send_twice` se implementará automáticamente.
}
\end{minted}

    Notar que, sin embargo, Rust no es un lenguaje orientado a objetos. Un
    \trait puede heredar de otro \trait, pero un \struct no puede heredar de
    otro \struct.

\end{itemize}

\section{Genéricos y librería estándar}

De forma similar a C++, Rust posee tipos genéricos. Esto permite la
implementación de una librería estándar flexible, con varias estructuras de
datos importantes a conocer:

% NOTE: esto creo que lo puedo evitar si no incluyo cómo se usa async_ffi, que
% me parece demasiado complejo para el documento.
% \begin{minted}{rust}
% // Función genérica, donde `ToString` es un trait
% fn print<T: ToString>(t: T) {}

% // Otra manera de especificar genéricos con diferencias
% // menores que no se explicarán en esta introducción.
% fn print(t: impl ToString) {}
% \end{minted}

\begin{itemize}
    \item Tipos primitivos:
        \begin{itemize}
            \item Carácteres con \rust{char}.

            \item Punto flotante con \rust{f32} y \rust{f64}.

            \item Booleanos con \rust{bool}.

            \item Enteros: \rust{u8}, \rust{i8}, \rust{u16}, \rust{i16},
                \rust{u32}, \rust{i32}, \rust{u64}, \rust{i64}, e incluso
                \rust{i128} y \rust{u128} en las arquitecturas que lo soportan.

            \item Vectores de tamaño fijo: por ejemplo \rust{[1, 2, 3, 4, 5]}.

            \item N-tuplas como \rust{(1, true, 9.2)}.

            \item El tipo ``unidad'', \rust{()}, equivalente a \code{void} en C
                o C++.

            \item Punteros básicos con \rust{*const T} o \rust{*mut T}.

        \end{itemize}

    \item \rust{Vec<T>} representa un vector contiguo y redimensionable.

    \item \rust{HashMap<K, V>} es una tabla hash, genérica respecto a su clave
        \rust{K} y su valor \rust{V}. No se encuentra en el preludio, por lo que
        requeriría la siguiente declaración, similar a un \code{import} de Java:

\begin{minted}{rust}
use std::collections::HashMap;
\end{minted}

    \item \rust{Box<T>}, usado para localizar un tipo \rust{T} no nulo en
        memoria. Además de un \rust{*const T}, incluye el tamaño que ocupa
        \rust{T} y tiene una interfaz limitada para que su uso sea siempre
        seguro.

    \item \rust{str} es una cadena UTF-8 de solo lectura, típicamente usada con
        una referencia \rust{&str}. Va acompañada por su longitud, por lo que no
        hace falta terminarla con \code{\0}, a diferencia de C. \rust{String} es
        su versión modificable asignada en memoria.

\end{itemize}

\section{Gestión de errores}

En Rust, los errores se indican con el tipo \rust{Result<T, E>}. Este se trata
de una enumeración cuyo valor puede ser \rust{Ok(T)}, con el resultado obtenido
satisfactoriamente, o \rust{Err(E)}, con el tipo de error que ha sucedido. Dado
que el resultado está contenido dentro suyo, es imposible olvidar comprobar si
se ha producido algún error. Se puede usar \rust{match} para comprobar el
resultado, o una serie de funciones disponibles para hacer el proceso más
ergonómico:

\begin{minted}{rust}
match load_file(input) {
    Ok(data) => /* ... */,
    Err(e) => eprintln!("Error: {e}"),
}
\end{minted}

En caso de que se produjera un error del que el programa no se puediera
recuperar, como quedarse sin memoria o un fallo inesperado en la implementación,
se usa la funcionalidad de \emph{pánicos}. Un pánico se propaga de forma similar
a una excepción de C++ o Java, y terminará la ejecución por completo. Se puede
invocar con el macro \rust{panic!} o utilidades similares.

\section{Macros}

Rust cuenta con dos tipos de macros: \emph{declarativos} y \emph{procedurales}.
Ambos permiten generar código a tiempo de compilación, pero se diferencian
principalmente en la flexibilidad que ofrecen, a coste de un coste de desarrollo
menor o mayor, respectivamente.

Los macros declarativos se crean con una sintaxis especializada, similar a un
\rust{match} con patrones de tokens (identificadores, tipos, etc) como entrada,
y los tokens nuevos como salida. Son similares a los macros de C o C++, pero más
potentes e higiénicos (i.e., su expansión no captura identificadores
accidentalmente).

Los macros procedurales se describen como extensiones del lenguaje.
Esencialmente, ejecutan código en la compilación que consume y produce sintaxis
de Rust; consisten en directamente transformar el Árbol de Sintaxis Abstracta
(AST)~\cite[Procedural Macros]{rustref}. Consecuentemente, su complejidad es
mucho mayor, pero expanden las posibilidades de los macros enormemente.

\begin{minted}{rust}
some_macro!(1, 2, 3); // Puede ser tanto declarativo como procedural
\end{minted}

\begin{minted}{rust}
// Sintaxis típica de invocación de un macro
some_macro! {
    fn some_function() { /* ... */ }
}

// También permitido en el caso de los procedurales
#[some_macro]
fn some_function() { /* ... */ }
\end{minted}

Finalmente, los macros procedurales se pueden declarar de forma que
\emph{deriven} (implementen automáticamente) un \trait. Esto evita escribir
código repetitivo de forma muy sencilla:

\begin{minted}{rust}
// Con un macro `derive` para el trait `Debug`, que sirve para
// mostrar variables por pantalla.
#[derive(Debug)]
struct X(i32);

// Sin ellos sería lo siguiente. Como es trivial se puede
// simplificar en un macro procedural de tipo `derive`.
impl fmt::Debug for X {
    fn fmt(&self, f: &mut fmt::Formatter) -> fmt::Result {
        write!(f, "{:?}", self.0)
    }
}
\end{minted}

\section{Lifetimes}

La seguridad que provee Rust en memoria se basa en un modelo a tiempo de
compilación con \lifetimes. Una \lifetime es 

TODO: esto depende de cómo se acaba incluyendo la sección de ``problemas con
varianza y subtipado''.

\section{Unsafe}

Para poder mantener control completo a bajo nivel, es posible ignorar sus
garantías de seguridad con el sub-lenguaje llamado \emph{unsafe Rust}. El
análisis estático de Rust es conservativo; en algunas ocasiones es posible que
rechace algunos programas correctos. El desarrollador puede indicar que es
consciente de la situación y puede apagar este análisis para corregirlo por sí
mismo, arriesgándose a cometer un error en su código.

Se puede acceder a \emph{unsafe Rust} conteniendo el código dentro de un bloque
\rust{unsafe { /* ... */ }} o una función \rust{unsafe fn name() { /* ... */ }}.
Funciona igual que rust, pero incluye varias nuevas habilidades, entre otras:

\begin{itemize}
    \item Leer un puntero bruto en memoria

    \item Acceder o modificar una variable estática mutable

    \item Llamar a una función \unsafe

\end{itemize}

\section{Programación asíncrona}

Como muchos lenguajes modernos, Rust da soporte a la programación asíncrona, un
modelo de programación concurrente. Sin entrar en excesivo detalle, esta permite
tener una gran cantidad de \emph{tareas} concurrentes ejecutándose sobre unos
pocos hilos del Sistema Operativo. Su caso de uso principal es programas cuyo
rendimiento está limitado por operaciones de entrada y salida, como servidores o
bases de datos~\cite{rustasyncbook}.

\begin{minted}{rust}
// Con `async` se indica que la función es asíncrona.
async fn get_two_sites_async() {
    // Creación de dos "futuros" que, al completarse, descargarán
    // asíncronamente las páginas web. Similar a la creación de
    // un nuevo hilo.
    let future_one = download_async("https://www.foo.com");
    let future_two = download_async("https://www.bar.com");

    // Ejecutar las dos tareas. Similar a esperar la terminación de
    // los hilos.
    join!(future_one, future_two);

    // Con `.await` se puede esperar a la terminación de un futuro
    // individual.
    let future_three = download_async("https://www.bar.com").await;
}
\end{minted}
