% vim: spelllang=es

\chapter{Introducción}
\pagenumbering{arabic}

\section{Contexto}

Este proyecto ha sido organizado y financiado por \emph{Wayfair}, una empresa
estadounidense de comercio digital de muebles y artículos del
hogar~\cite{wayfair}. Ingresó 13.700 millones de dólares en
2021~\cite{wayfairRevenue} y actualmente ofrece 14 millones de ítems de más de
11.000 proveedores globales~\cite{wayfairItems}.

Se trata de una entidad de gran tamaño, lo que se refleja en su cantidad
disponible de datos o \emph{logs} para describir el comportamiento de sus
servicios. Estos son de vital importancia, dado que permiten encontrar errores
en sus sistemas y ayudan a medir el rendimiento del negocio.

Los logs no suelen seguir un formato único ni necesariamente estructurado (es
decir, texto plano en vez de JSON o XML). Por tanto, para poderlos usar es
necesario algún tipo de procesado, que también puede incluir ignorar información
redundante o transformar ciertos valores. Esto se llevaba a cabo con la
arquitectura ELK, que incluye el sistema de procesado de datos
\namecite{logstash}, entre otros.

A medida que crecía la empresa, estas herramientas de observabilidad dejaban de
ser apropiadas. Cada día, Wayfair procesaba 100 \emph{terabytes} de logs. Para
visualizar mejor esta cantidad, si se escribieran en papel, la pila resultante
sería de unos 5.400 kilómetros, o en otras palabras, 5 idas y vueltas a la
Estación Espacial Internacional. El coste mensual de esta infraestructura
alcanzaba unos 40,000 dólares\footnote{Las cantidades mencionadas en esta
sección son aproximadas para dar una representación del tamaño de datos
procesados y su coste sin comprometer la privacidad de Wayfair.}.

El problema principal residía en Logstash, escrito en Java y con un propósito de
uso general en vez de especializado en rendimiento y escalabilidad. La solución
de Wayfair fue crear la alternativa \namecite{tremor}. Esta nueva herramienta
está escrita con el lenguaje de programación Rust, que ofrece un rendimiento
similar a C o C++ con una mayor seguridad. Adicionalmente, implementa técnicas
para maximizar la eficiencia, como SIMD, programación asíncrona con hilos o, en
un futuro cercano, \emph{clustering}. Una vez la primera versión de Tremor fue
publicada, el coste de la infraestructura se redujo a 780 dólares mensuales, 50
veces menos que el valor original~\cite{tremorcon_lll}.

Tremor ha evolucionado desde entonces y ha expandido sus posibilidades más allá
de logs a algo más abstracto, los \emph{eventos}. Ahora soporta una mayor
cantidad de protocolos, formatos y software de donde recibir y enviar eventos, e
incluso implementa su propio lenguaje de configuración y procesado. El proyecto
se lanzó como código abierto y de forma independiente a Wayfair, y aunque el
equipo de desarrollo sigue siendo principalmente suyo, cualquiera puede
contribuir.

Posteriormente, Tremor se unió a la \emph{Cloud Native Computing Foundation
(CNCF)}~\cite{cncf}, principalmente conocida por mantener y gestionar
\namecite{k8s}. Asimismo, CNCF es parte de la organización \emph{Linux
Foundation~(LFX)}~\cite{lfx}, que además del famoso kernel también ayuda a todo
tipo de proyectos de código abierto, como Node.js o Let's
Encrypt~\cite{lfx_projects}.

Formalmente, el trabajo se ha llevado a cabo gracias a la iniciativa \emph{LFX
Mentorship}, que promueve el aprendizaje de desarrolladores de código abierto,
proporcionando una plataforma transparente y facilitando un sistema de
pagos~\cite{lfx_mentorship}. En ella, proyectos de código abierto especifican
una tarea concreta a realizar, proporcionando a cambio un mentor que le guíe
durante el proceso y una ayuda monetaria.

El título de este proyecto en la plataforma de LFX es ``CNCF -- Tremor: Add
plugin support for tremor (PDK)'', es decir, desarrollar un sistema de plugins
para Tremor. El plazo original era de tres meses comenzando en agosto, pero se
acabó alargando a unos once meses. Disponía de tres mentores --- los
desarrolladores principales de Tremor ---, que me ayudaron a entender el
funcionamiento interno del programa y dieron consejo cuando era necesario.

\section{Objetivo}

La tarea a llevar a cabo es la implementación de un sistema de plugins, también
denominado \emph{Plugin Development Kit~(PDK)}, para la base de código ya
existente de Tremor. El programa se dividirá en dos partes independientes: la
\emph{runtime} y los \emph{plugins}. Los plugins son componentes que implementan
cualquier tipo de funcionalidad, y la runtime es capaz de cargarlos y usarlos
dinámicamente en el programa. Una comparación de más alto nivel se podría dar en
un móvil: el hecho de que el sistema operativo (runtime) pueda instalar
cualquier aplicación (plugin) lo hace mucho más flexible que un dispositivo que
únicamente tuviera una serie de aplicaciones predefinidas.

En el caso de Tremor, un plugin podría dar soporte para recibir eventos de
Apache Kafka o para enviarlos a Postgres, por ejemplo. La runtime debería ser
capaz de cargar y usar esa funcionalidad mientras se está ejecutando el
programa, en vez de a tiempo de compilación. Dividir el binario de Tremor en
varios componentes más pequeños compilables independientemente implica una
reducción en los tiempos de compilación de cada uno de ellos, que era el
objetivo principal del proyecto.

Existen varias tecnologías disponibles para este sistema de plugins, como
\emph{WebAssembly}, \emph{eBPF} o comunicación inter-proceso. Se investigarán
las más importantes antes de implementarlo, pero el equipo de Tremor pensaba
usar \emph{cargado dinámico} desde el principio. El concepto es simple: tanto la
runtime como los plugins son binarios nativos (es decir, código máquina). Para
inicializar un plugin, la runtime lo cargará en su propia memoria RAM reservada.
Posteriormente, mediante una interfaz establecida, denominada \emph{Interfaz
Binaria de Aplicación~(ABI)}, es posible acceder a funciones y recursos en los
plugins.

El \emph{cargado dinámico} es un método especialmente popular en el lenguaje de
programación C, dado que su interfaz de comunicación es relativamente sencilla y
está muy bien definida en su estándar. Asimismo, se trata de una solución
especialmente eficiente, ya que la comunicación es binaria y directa. Las
características de Tremor implican que el método a usar debería tener un alto
rendimiento, así que el cargado dinámico es una de las mejores opciones
disponibles.

Sin embargo, lenguajes modernos como C++ o Rust implementan características más
complejas como excepciones o tipos genéricos y por tanto no especifican
estrictamente un ABI. En el caso de Rust, si se compila la runtime y los plugins
separadamente, no existe ninguna garantía de que la representación binaria de
los datos o de que la convención de llamada a funciones --- entre otros --- sea
la misma.

Por tanto, el cargado dinámico es imposible puramente con Rust. Se debe recurrir
a otro ABI que sí sea estable, como el de C. Todas las funciones y tipos
involucrados en la comunicación entre runtime y plugins deberán declararse
siguiendo el estándar de C. Los tipos seguirán unas reglas que definen cómo se
representan en memoria y las funciones tendrán que seguir una convención
específica.

El problema principal con Tremor es que, al estar escrito puramente con Rust,
todas sus funciones y tipos internos también se declaran con este lenguaje. Este
proyecto tendrá que desarrollar un método para transformar tipos complejos de
Rust a C y viceversa para poder interactuar con los plugins.

\section{Motivación}

\subsection{Tiempos de compilación}

Un problema de creciente importancia en Tremor es su tiempo de compilación. En
mi portátil Dell Vostro 5481 con un Intel i5-8265U, 16GB de RAM a 2667 MHz, un
SSD de 256GB y Arch Linux con el kernel 5.18 de 64 bits, compilar el binario
\code{tremor} desde cero requiere más de 7 minutos en modo de depuración, y
sobre 13 en modo de producción (con optimizaciones del compilador). Incluso en
el caso de cambios incrementales (una vez las dependencias ya han sido
compiladas), hay que esperar unos 10 segundos.

Puede que estas cifras no sean tan preocupantes en comparación con software de
un tamaño mucho mayor. Pero debido a la naturaleza del programa, el problema
solo empeorará con el tiempo. Tremor debe tener soporte para un gran número de
protocolos (como TCP o WebSockets), software (como PostgreSQL o ElasticSearch) y
códecs (como JSON o YAML). El número de dependencias continuará incrementando
hasta resultar en una pésima experiencia de desarrollo e imposibilitar la
creación de nuevas prestaciones.

Los problemas relacionados con tiempos de compilación excesivamente largos no se
limitan a Tremor. Es uno de las mayores críticas que recibe Rust: un 61\% de sus
usuarios declaran que aún se necesita trabajo para mejorar la
situación~\cite{rustsurvey}.

\subsection{Modularidad y flexibilidad}

Otra ventaja que provee un sistema de plugins es modularidad. Ser capaz de
tratar la runtime y los plugins de forma separada suele resultar en una
arquitectura más limpia~\cite{baldwin2000design}. Hace posible el
desacoplamiento completo del ejecutable y sus componentes. Algunas dependencias
tienen un ciclo de versionado más rápido que otras y, generalmente, es más
conveniente desarrollar, realizar tests o actualizar únicamente un plugin, en
lugar del programa por completo.

\subsection{Aprender de otros}

Existen proyectos maduros con características similares a las de Tremor, como
\namecite{nginx} o \namecite{apachehttpserver}, que llevan beneficiándose de un
sistema de plugins desde hace mucho. Informan mejorías en flexibilidad,
extensibilidad y facilidad de
desarrollo~\cite{nginxPluginsAdvantages}\cite{apachePluginsAdvantages}. Aunque
las desventajas también mencionen un pequeño impacto en el rendimiento y la
posibilidad de caer en un ``infierno de dependencias'' (relacionado con plugins
que dependen de otros), sigue siendo una buena idea al menos considerarlo para
Tremor.

\section{Metodología}

\subsection{Organización}

El proyecto ha tenido una duración de unos once meses. La propuesta a Tremor se
realizó de abril a mayo de 2021, pero no se comenzó el desarrollo hasta
aceptarse en agosto del mismo año, y se terminó en junio de 2022.

Su realización ha sido completamente remota y con horarios muy flexibles. Se usó
el servidor de Discord de Tremor~\cite{tremor_discord} como plataforma principal
de comunicación, tanto por texto como por videollamada. Se programó una llamada
por semana, en la que explicaba mi progreso y recibía ayuda en caso de que me
hubiera quedado atascado o necesitara alguna opinión adicional.

Disponía de tres mentores, que me guiaban en el proceso de desarrollo: Darach
Ennis (\emph{Principal Engineer and Director of Tremor Project}), Matthias Wahl
(\emph{Staff Engineer}) y Heinz N. Gies (\emph{Senior Staff Engineer}), todos
empleados por Wayfair. En el plano académico, Francisco Javier Fabra Caro
(\emph{Doctor Ingeniero en Informática}) de la Universidad de Zaragoza fue el
director del proyecto.

La organización de forma más estructurada para las tareas que tenía pendientes,
aquellas en las que estaba trabajando en ese momento, y las que ya había
realizado, se basó principalmente en un Kanban en GitHub.

\subsection{Desarrollo}

Para reducir el coste de desarrollo y asegurarse de que el proceso sea
completamente seguro tanto en memoria como en concurrencia, el sistema de
plugins aprovecha librerías existentes en Rust y herramientas como \emph{macros
procedurales}. Los macros procedurales son mucho más potentes que los
convencionales (\emph{declarativos}), puesto que directamente consumen y generan
sintaxis de Rust a tiempo de compilación para implementar cualquier tipo de
código trivial o repetitivo.

El sistema de compilación usado es la solución oficial de Rust, Cargo, que
también incluye un \emph{formatter}, un \emph{linter} y extensiones instalables
creadas por la comunidad. Adicionalmente, existe una gran cantidad de tests y
\emph{benchmarks} que se han de tener en cuenta para mantener el \emph{code
coverage} (la cantidad de código cubierta por los tests) y el rendimiento.

\subsection{Recursos públicos}

Este trabajo está disponible públicamente al completo. Además, a medida que he
investigado y desarrollado el sistema de plugins, he ido escribiendo todo en mi
blog personal, \namecite{nullderef}. Dispone de una serie con un total de
seis\footnote{El último artículo será publicado poco después de la entrega de
esta memoria.} artículos, con más detalles sobre la implementación final. La
organización difiere considerablemente, puesto que los artículos se escribieron
a medida que se realizaba el proyecto, resultando en una estructura menos
rigurosa. Esto toma un formato de tesis, mientras que el blog cuenta la historia
cronológicamente y sirve mejor como un tutorial para alguien que quiera
implementar su propio sistema de plugins.

\begin{itemize}
    \item La serie de artículos en mi blog personal:\\
        \url{https://nullderef.com/series/rust-plugins/}.

    \item El repositorio para el binario de Tremor:\\
        \url{https://github.com/tremor-rs/tremor-runtime}

    \item Mi \emph{fork}, con ramas adicionales usadas durante el desarrollo:\\
        \url{https://github.com/marioortizmanero/tremor-runtime}

    \item Mi repositorio con experimentos iniciales:\\
        \url{https://github.com/marioortizmanero/pdk-experiments}

    \item Kanban para organizar las ideas y tareas:\\
        \url{https://github.com/marioortizmanero/tremor-runtime/projects/1}

    \item Página del proyecto en la iniciativa LFX Mentorship:\\
        \url{https://mentorship.lfx.linuxfoundation.org/project/b90f7174-fc53-40bc-b9e2-9905f88c38ff}

    \item \emph{Tracking issue} del proyecto en GitHub:\\
        \url{https://github.com/tremor-rs/tremor-runtime/issues/791}

    \item \emph{Request For Comments} (RFC) del proyecto:\\
        \url{https://www.tremor.rs/rfc/accepted/plugin-development-kit/}

\end{itemize}

Se recomienda también consultar el Anexo~\ref{annex:contributions}, que lista
todas las contribuciones de código abierto realizadas durante este proyecto. El
Anexo~\ref{annex:hours} especifica cada fase de desarrollo junto a sus horas
dedicadas y enlaces relacionados.
