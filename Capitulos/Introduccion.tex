% vim: spelllang=es

\chapter{Introducción}
\pagenumbering{arabic}

\section{Contexto}

Este proyecto se ha realizado en colaboración con
\emph{Tremor}\footnote{\url{https://tremor.rs}}, un sistema de procesado de
eventos de alto rendimiento, escrito en el lenguaje de programación
\emph{Rust}\footnote{\url{https://www.rust-lang.org}}. Tremor es un programa de
código abierto bajo la fundación \emph{Cloud Native Computing
Foundation~(CNCF)}\footnote{\url{https://www.cncf.io/}}, que es también parte de
la organización \emph{Linux
Foundation~(LFX)}\footnote{\url{https://www.linuxfoundation.org/}}.

Formalmente, el trabajo se ha llevado a cabo gracias a la iniciativa \emph{LFX
Mentorship}, con el título ``CNCF - Tremor: Add plugin support for tremor
(PDK)''\footnote{\url{https://mentorship.lfx.linuxfoundation.org/project/b90f7174-fc53-40bc-b9e2-9905f88c38ff}}.
Esta iniciativa promueve el aprendizaje de desarrolladores de código abierto,
proporcionando una plataforma transparente, y facilitando un sistema de pagos.

Finalmente, \emph{Wayfair} es una empresa estadounidense de comercio digital de
muebles y artículos del hogar\footnote{\url{https://www.wayfair.com/}}.
Actualmente, ofrece 14 millones de ítems de más de 11.000 proveedores
globales~\cite{wayfairItems}, y es el principal financiador tanto de Tremor como
de este proyecto.

\section{Objetivo}

La tarea a llevar a cabo es la implementación de un sistema de plugins para la
base de código ya existente, lo cual es una tarea no-trivial, dado que Rust no
tiene un \emph{Application Binary Interface~(ABI)} estable.

% TODO: debería mencionar que explicaré luego qué es el ABI?

\section{Motivación}

\subsection{Tiempos de compilación}

% TODO: incluir modo release?
Actualmente, el problema más importante en Tremor es sus tiempos de compilación.
En un ordenador de gama media de \~{}600 € como el Dell Vostro 5481, compilar el
binario \code{tremor} desde cero requiere de más de 7 minutos en modo debug.
Incluso en el caso de cambios incrementales (una vez las dependencias ya han
sido compiladas), hay que esperar unos 10 segundos. Esto no es una buena
experiencia de desarrollo y previene que nuevos ingenieros se unan a la
comunidad de Tremor.

Debido a la naturaleza del programa, este problema solo podrá empeorar con el
tiempo. Tremor debe tener soporte para un gran número de protocolos (e.g., TCP
o UDP), software (e.g., Kafka o PostgreSQL), y codecs (e.g., JSON o YAML). El
número de dependencias continuará incrementando hasta que impida la creación de
nuevas prestaciones en Tremor.

Los problemas relacionados con tiempos de compilación excesivamente largos no se
limitan a Tremor. Es uno de las mayores críticas que recibe Rust, y un 61\% de
sus usuarios declaran que aún se necesita trabajo para mejorar la situación
\cite{rustsurvey}.

\subsection{Modularidad}

Otra ventaja que provee un sistema de plugins es modularidad; ser capaz de
tratar la \emph{runtime} y los \emph{plugins} de forma separada suele resultar
en una arquitectura más limpia \cite{baldwin2000design}.

También hace posible el desacoplamiento del ejecutable y sus componentes.
Algunas dependencias tienen un ciclo de versiones más rápido que otras, y
generalmente es más conveniente actualizar únicamente un plugin, en lugar de el
programa por completo.

\subsection{Aprender de otros}

Otros proyectos maduros con características similares a las de Tremor, como
\textcite{nginx} o \textcite{apachehttpserver}, han estado aprovechando las
ventajas de un sistema de plugins desde hace mucho. Informan mejorías en
flexibilidad, extensibilidad, y facilidad de
desarrollo~\cite{nginxPluginsAdvantages}\cite{apachePluginsAdvantages}. Aunque
las desventajas también mencionen un pequeño impacto en el rendimiento, y la
posibilidad de caer en un \emph{dependency hell}, sigue siendo una buena idea al
menos considerarlo para Tremor.

\section{Metodología}

El proyecto ha tenido una duración de unos 10 meses, comenzando en agosto de
2021, y terminando en mayo/junio de 2022. Su realización ha sido completamente
remota y con horarios muy flexibles.

% TODO: debería ser más específico sobre las horas? Mucha cuenta no he llevado
% pero vamos sí que estoy segurísimo de que han sido 300 horas, y mucho más
% también.
