% vim: spelllang=es

\chapter{¿Qué es Rust?}\label{ch:rust}

Dado que Rust es un lenguaje de programación que tan solo anunció su primera
versión en 2015, aún no es conocido por muchos desarrolladores. Esta memoria no
requiere conocimientos previos sobre Rust. Sin embargo, la implementación en sí
y otras partes más avanzadas, como el Anexo~\ref{annex:covariance}, asumen una
familiaridad con el lenguaje más extensiva. Para esos casos, se recomienda
consultar el Anexo~\ref{annex:rust}, que entra en mayor detalle sobre el
lenguaje.

Rust es un lenguaje de programación de sistemas compilado y de propósito
general. Su objetivo es maximizar rendimiento y seguridad, tanto en memoria como
en concurrencia y sin necesidad de un recolector de basura. Las garantías de
seguridad son comprobadas en tiempo de compilación gracias a un modelo estricto
de programación y a un sistema fuertemente y estáticamente tipado. No permite
punteros nulos, referencias colgantes, ni condiciones de carrera --- aunque sí
fugas de memoria.

Proporciona control a bajo nivel, manteniendo una productividad cercana a
lenguajes de alto nivel. Dispone de programación funcional, genéricos,
inferencia de tipos y macros, entre otros. No obstante, es posible ignorar
explícitamente el modelo de memoria y concurrencia para casos avanzados en los
que se necesite control completo.

Los programas en Rust se diseñan basándose en la composición, en lugar del
polimorfismo convencional de Python o C++. Se puede conseguir la misma
flexibilidad mediante tipos estructurados y un mecanismo llamado \emph{traits}
--- similar a las interfaces de Java, pero más potentes.

El ecosistema básico de Rust es altamente cohesivo: incluye el sistema de
compilado y administrador de paquetes \emph{Cargo}, el \emph{formatter} de
código \emph{Rustfmt} y el \emph{linter} \emph{Clippy}. La instalación de estas
herramientas se suele gestionar con el programa \emph{rustup}.
