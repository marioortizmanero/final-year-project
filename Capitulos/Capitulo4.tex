% vim: spelllang=es

\chapter{Implementación}

\section{Metodología}

Antes de nada, es importante aprender un poco sobre cómo realizar cambios en el
código de Tremor eficientemente. Este proyecto modificará gran cantidad de
líneas y cuanto más rápido sea el desarrollo, menos problemas habrán. Esto se
puede cubrir de forma específica al lenguaje Rust, con trucos o consejos que
puedan facilitar el desarrollo, o de forma más general, con la estrategia de
trabajo a seguir. En esta sección se cubrirá lo último, dado que es menos un
detalle de implementación.

La metodología fue inspirada por mis mentores, que lo denominaron el ``Just make
it work'' o \work. Se basa en que, inicialmente, con lo que más problemas tenía
era el perderme en los detalles. Pero ciertamente, primero de todo lo importante
es que ``funcione''. Siempre y cuando el sistema de plugins se pueda compilar y
ejecutar, lo siguiente es aceptable:

\begin{itemize}
    \item Código ``feo'' (no idiomático, repetitivo o desordenado).

    \item Código de bajo rendimiento.

    \item Documentación pobre.

    \item No tener tests tempranos.

    \item No aplicar sugerencias recomendadas por \emph{linters} (en el caso de
        Rust, \emph{Clippy}).

\end{itemize}

El propio desarrollo invitaba a aprovechar las ventajas de un lenguaje
fuertemente tipado como Rust, evitando realizar testing posiblemente prematuro.
Esto consistía en realizar cambios y posteriormente trabajar en que los aceptara
el compilador, repetidamente. Únicamente se procedió al testing exhaustivo una
vez la interfaz final del sistema de plugins compilaba, pasadas las fases
tempranas o incluso medias.

Adicionalmente, las optimizaciones prematuras son la fuente de todos los
problemas. No es algo que sea importante aún. Solo una vez terminada la primera
iteración se puede dedicar más tiempo a medir el rendimiento para saber cuáles
optimizaciones merecen la pena. Notar que sí es importante escoger un método
general que sea apropiado en términos de rendimiento; fue por ello por lo que se
descartó WebAssembly o IPC en el capítulo anterior. Pero definitivamente el
rendirme en, por ejemplo, evitar una conversión entre dos tipos innecesaria y
que posiblemente no afecte al rendimiento al fin y al cabo.

Lo que quería dejar claro el equipo de Tremor es que todos los tests, limpiezas
u optimizaciones que intentes realizar en este momento acabará muy probablemente
siendo en vano. Se llegará a un punto en el que no se pueda continuar y que
requiera repensar y reescribir la mayoría del trabajo. Cuando todo compile y
aparentemente funcione correctamente, se puede dedicar esfuerzo a trabajar en
estos temas secundarios. Si algo no importante está llevando demasiado tiempo,
se debería marcar como TODO o FIXME y dejarlo para otro momento.

Notar que no hay problema con ``gastar'' el tiempo con métodos que acaban siendo
incorrectos, porque realmente no se está ``gastando'' nada; son un paso
necesario para llegar a la solución final. Pero es doloroso tener que eliminar
código al que le has dedicado tiempo, así que al menos debería intentarse
minimizar el impacto que esto tenga.

\section{Versionado en \abistable}

Dado que \abistable va a ser la librería principal en la que se basará el
sistema de plugins, debe entenderse su funcionamiento al completo. Además de
conocer cómo se usa, es importante saber cómo soluciona los problemas más
comunes en sistemas de plugins.

La librería especifica lo siguiente respecto a su sistema de compatibilidad de
tipos~\cite{abistable_safety}:

\begin{itemize}
    \item ``El ABI de \abistable se comprueba siempre. Cada versión \code{0.y.0}
        y \code{x.0.0} de \abistable define su propio ABI, que es incompatible
        con versiones anteriores.''

    \item ``Los tipos se comprueban recursivamente cuando se carga una librería
        dinámica, antes de llamar ninguna función.''

\end{itemize}

Todo esto se basa en el \trait (equivalente en este caso a una clase abstracta
de Java) \rust{StableAbi}, indicador de que un tipo es seguro para cargado
dinámico. Contiene información como su disposición en memoria y puede ser
implementado automáticamente con un macro procedural. Antes de usar un plugin
con \abistable, la librería comprobará que sus tipos sean compatibles con
aquellos de la runtime.

Sin este mecanismo, sería posible cargar un plugin con una interfaz diferente a
la runtime, resultando en violaciones de acceso a memoria. Pongamos un caso en
el que una estructura en la interfaz de la runtime tuviese un campo adicional.
El plugin exportará esa estructura sin el campo nuevo, puesto que usa una
interfaz anticuada. Cuando la runtime intente acceder a la estructura del
plugin, se leerá un campo que no existe y que por tanto es parte de memoria no
definida.

\section{Conversión de la interfaz a C}

Es importante mantener la interfaz de plugins lo más simple posible. Los
detalles de comunicación deberían dejarse a la runtime, de forma que los plugins
se limiten a exportar una lista de funciones síncronas. De esta forma, se podrá
evitar en lo posible pasar tipos complejos (programación asíncrona, canales de
comunicación, etc.) entre la runtime y los plugins, que implicaría una carga de
trabajo mayor.

Una vez esta interfaz básica se defina, se puede crear un \emph{wrapper} de más
alto nivel en la runtime que se encargue de la comunicación y de mejorar su
usabilidad dentro de Tremor. Esto mismo lo hacen otras librerías como
\namecite{rdkafka}, que implementa una capa de abstracción asíncrona sobre su
interfaz de C en \namecite{rdkafka-sys}.

El primer paso consiste en declarar la interfaz del PDK de forma que use el ABI
de C, en vez del de Rust. Esto se puede hacer con el atributo \rust{#[repr(C)]}
(en lugar del \rust{#[repr(Rust)]} implícito). Para usar \abistable también
tendrá que implementarse el \trait \rust{StableAbi}.

La dificultad reside en que todos los tipos dentro de la interfaz \emph{también}
tendrán que haber sido recursivamente declarados con tanto \rust{#[repr(C)]}
como \rust{StableAbi}. Esto puede convertirse en un problema si alguno de ellos
es parte de una dependencia y no se tiene acceso directo. Se tendrá que crear un
\emph{wrapper} que envuelva el tipo de forma opaca (escondiendo su estructura en
memoria) o abrir un pull request para añadir soporte. Se incluye la explicación
completa de este proceso y su implementación en el Anexo~\ref{annex:abi}.

\section{Cargado de plugins}

Afortunadamente, \abistable ya se encarga de la mayoría del trabajo en este
aspecto. Lo único necesario es definir el cómo se encuentran los plugins en el
sistema. Para ello, se introduce una nueva variable de entorno
\code{TREMOR_PLUGIN_PATH} que liste todos los directorios que pueden contener
plugins, separados con dos puntos como \code{PATH}.

Una vez se tiene la lista de directorios, se comprobarán también sus
subdirectorios, recursivamente. Es importante añadir un límite de profundidad
aquí para evitar que el programa se quede atascado, por ejemplo, si el usuario
incluyese el directorio raíz (\code{/}) accidentalmente. Tampoco deberían
seguirse enlaces simbólicos por simplicidad.

Todos los archivos encontrados cuya extensión sea la de una librería dinámica
tendrán que añadirse a la lista de posibles plugins. Esta extensión varía según
el sistema operativo: en Linux es \code{.so}, en Windows es \code{.dll} y en
MacOS es \code{.dylib}.

Finalmente, si al intentar cargar uno de los plugins encontrados la operación
falla, se mostrará una advertencia y se continuará probando los demás. Un método
más robusto para evitar cargar ficheros que no sean plugins de Tremor sería usar
una extensión personalizada o algún tipo de convención para el nombre. Sin
embargo, al personalizarse manualmente los directorios, se asume que la gran
mayoría de las librerías dinámicas encontradas serán plugins.

\section{Gestión de pánicos}\label{sec:panics}

Los pánicos en Rust se usan para expresar errores de los que un programa no se
puede recuperar. Por ejemplo, acceder a un índice inexistente en una lista, o
quedarse sin memoria. Como las excepciones de C++ o Java, los pánicos se
propagarán hasta llegar a la función principal y terminar la ejecución del
programa entero.

Actualmente, lanzar pánicos a través de la interfaz C es comportamiento no
definido~\cite[FFI and Panics]{nomicon}. Aunque el programa aborte en la mayoría
de los casos, no existe ninguna garantía de que vaya a suceder así; podría
continuar en un estado inválido, con cualquier tipo de consecuencia.

La solución más directa es usar la función \rust{std::panic::catch_unwind}, que
para casos excepcionales como este, puede capturar un pánico propagándose al
llamarla. Se podría usar en todas las funciones exportadas por el plugin
internamente, y en caso de producirse un pánico se terminaría el programa
manualmente, en lugar de dejar que se propague desde el plugin hasta la runtime,
que sería indefinido.

También es posible configurar el programa al compilarlo para que aborte cuando
se produzca un pánico, en vez de propagarlo. De esta forma, no se llegaría a
invocar comportamiento no definido y se mantendría un rendimiento máximo ---
capturar pánicos tiene un coste. Sin embargo, implica varias desventajas
importantes: al abortar, se perderá la información de depuración que dan los
pánicos, no se limpiará el estado del programa, y actualmente es imposible saber
si el plugin ha configurado los pánicos para que aborten~\cite{pluggablepanic}.

Esto es algo que \abistable ha tenido en cuenta desde el principio. La librería
detecta cuándo se va a propagar un pánico a la runtime y en ese caso aborta el
programa por completo. Esta parte se realiza transparentemente; no hace falta
que el desarrollador se preocupe en ningún momento por ello.

Esta solución no es perfecta, ya que tiene un pequeño coste de rendimiento e
imposibilita el recuperarse de errores en los plugins. En una futura versión de
Tremor, podría ser posible reiniciar plugins en caso de que dejen de funcionar,
mejorando la resiliencia a fallos.

El equipo de Rust conoce esta limitación y está trabajando en mejorar la
situación. En una futura versión, planea definir cuándo se puede propagar un
pánico de forma más precisa~\cite{cunwind}.

\section{Programación asíncrona}

El objetivo inicial era simplificar la interfaz lo suficiente como para que no
sea necesario tratar aspectos como programación asíncrona en el PDK. Esto acabó
siendo inevitable, dado que la asincronía es uno de los pilares de Tremor.

Para poder usar programación asíncrona con el ABI de C se puede recurrir a la
librería \namecite{async_ffi}, cuyo único problema era que no funcionaba con
\abistable en concreto. No se podía usar en la interfaz del PDK porque sus tipos
no implementaban el \trait \rust{StableAbi}, así que abrí un pull request para
darle soporte yo mismo~\cite{asyncfficontrib}. El uso de \code{async_ffi}
resulta en código más verboso, pero esto se podría mejorar en el futuro con
macros procedurales~\cite{asyncffimacro}.

\section{Seguridad en hilos}

\abistable utiliza la librería \namecite{libloading} internamente, cuya gestión
de errores no es segura en hilos en algunas plataformas, como \code{dlerror} en
FreeBSD~\cite{hardplugins3}\cite{thsafe_openbsd}. Sí que lo es en
Linux~\cite{thsafe_linux}, macOS~\cite{thsafe_macos} y Windows~\cite{thsafe_ms},
así que en el caso de Tremor no es un problema.

Será importante tenerlo en cuenta en el futuro; al añadir soporte para un nuevo
sistema operativo habrá que asegurarse de que su gestión de errores sea segura
en hilos. En caso contrario, deberá actualizarse \code{libloading} para
sincronizar el acceso con un mútex interno, como lo hace la librería
\namecite{dlopen}~\cite{thsafe_dlopen}. Notar que \code{dlopen} es también
mejorable, dado que usa el mútex \emph{siempre}, incluso en sistemas operativos
donde la gestión de errores sí es segura en hilos~\cite{thsafe_dlopen_issue}.

\section{Complejidad de las conversiones}\label{abiperf}

Un punto vital a tener en cuenta es el coste de realizar conversiones entre
tipos de la librería estándar y tipos de \abistable. Esto se dará en numerosas
ocasiones, dado que usar \abistable cuando no es necesario es subóptimo para
tanto el rendimiento como para la usabilidad. Y si, por ejemplo, convertir un
vector \rust{Vec<T>} a un \rust{RVec<T>} tuviese complejidad $O(n)$,
probablemente \abistable tendría que ser descartado como la solución escogida.

Afortunadamente, tras analizar la implementación de tipos como \rust{RVec<T>},
\rust{RSlice<T>}, \rust{RStr} o \rust{RString}, estas conversiones únicamente
consisten en transferir un puntero con los datos, sin necesidad de copiar nada.
Es decir, las conversiones que realizaremos serán $O(1)$.

\section{Problemas con varianza y subtipado}

Una complicación a la que se dedicó una cantidad considerable de tiempo tiene
que ver con el concepto de \emph{varianza} y \emph{subtipado}. En resumen, si un
tipo es \emph{covariante}, el modelo de memoria de Rust será mucho más flexible
al usarlo. Todo este mecanismo es implícito, es decir, en ningún momento el
desarrollador especifica manualmente la varianza del tipo; el compilador de Rust
lo infiere automáticamente.

Un tipo puede dejar de ser covariante si incumple una serie de reglas
preestablecidas. En ese caso, la flexibilidad adicional se pierde y se producen
errores casi imposibles de entender si uno no es familiar con el concepto de
varianza. Los tipos definidos en \abistable, a diferencia de aquellos en la
librería estándar, no eran covariantes, lo cual fue especialmente complicado de
descubrir y requirió reescribir algunas partes de la librería.

Afortunadamente, una futura versión de Rust hará la depuración de estos errores
mucho más intuitivo~\cite{smarterchecker}. Es un tema que requiere conocimientos
más avanzados de Rust, por lo que se incluye en el Anexo~\ref{annex:covariance}
al completo para más información.

\section{Optimizaciones}

Una vez implementada la primera versión del sistema de plugins, se realizan
optimizaciones iterativamente hasta alcanzar un rendimiento lo suficientemente
bueno. Antes de aplicarlas, sin embargo, es importante realizar mediciones para
asegurarse de que la mejora ofrecida es apreciable. Para ello, debe elaborarse
un entorno de \emph{benchmarking} riguroso y pruebas que tengan en cuenta casos
de uso variados y realistas. Los experimentos iniciales indican una degradación
del 35\% del rendimiento, por lo que tendrán que realizarse mejoras hasta llegar
al rango de 10-20\% propuesto por Tremor. Este paso se describe en detalle en el
Anexo~\ref{annex:benchmarks}.

Las primeras optimizaciones incluyen una limpieza general del código, la
restauración de optimizaciones que se eliminaron temporalmente (mencionado en el
Anexo~\ref{annex:abi}) o la simplificación de la interfaz del PDK. Sin embargo,
no son suficiente para el objetivo marcado, únicamente alcanzándose un 30\% de
degradación.
