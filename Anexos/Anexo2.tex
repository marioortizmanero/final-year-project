% vim: spelllang=es

\chapter{Funcionamiento interno de Tremor}\label{annex:tremor}

\section{Arquitectura}

Antes de comenzar a modificar el código existente en Tremor, era importante
conocer cómo funciona para evitar perder el tiempo. Tremor se basa en el modelo
actor. Citando Wikipedia:

% ORIGINAL:
% ``[The actor model treats the] actor as the universal primitive of concurrent
% computation. In response to a message it receives, an actor can: make local
% decisions, create more actors, send more messages, and determine how to
% respond to the next message received. Actors may modify their own private
% state, but can only affect each other indirectly through messaging (removing
% the need for lock-based synchronization).''

``[El modelo actor trata al] actor como el componente universal de computación
concurrente. En respuesta a un mensaje que recibe, un actor puede: tomar
decisiones locales, crear más actores, enviar más mensajes y determinar cómo
responder al siguiente mensaje recibido. Los actores pueden modificar su propio
estado privado, pero solo pueden afectarse entre sí indirectamente a través de
mensajería (eliminando la necesidad de sincronización con
\emph{locks}).''~\cite{wikiactor}

No usa un lenguaje (como \namecite{erlang}) o framework (como
\namecite{bastion}, quizá en el futuro) que siga estrictamente este modelo, pero
re-implementa los mismos patrones de forma manual. Tremor se basa en
\emph{programación asíncrona}, es decir, que en vez de hilos trabaja con
\emph{tareas}, un concepto de nivel más alto y especializado para operaciones de
entrada/salida. De la documentación de \namecite{async_std}, la runtime
asíncrona que usa Tremor:

% ORIGINAL:
% ``An executing asynchronous Rust program consists of a collection of native OS
% threads, on top of which multiple stackless coroutines are multiplexed. We
% refer to these as “tasks”. Tasks can be named, and provide some built-in
% support for synchronization.''

``La ejecución de un programa asíncrono en Rust consiste en una recopilación de
hilos nativos del Sistema Operativo, sobre los cuales múltiples corutinas no
apilables (\emph{stackless}) son multiplexadas. Nos referimos a ellas como
``tareas''. Las tareas pueden tener nombre e incluir soporte para
sincronización.''~\cite{asyncstd_task}

Podría resumirse su arquitectura con la frase: ``Tremor se basa en actores
corriendo en tareas diferentes, que se comunican asíncronamente con canales''.

\section{Detalles de implementación}

A nivel de implementación, los conectores se definen con el \trait
\rust{Connector}, incluido en la figura \ref{fig:tremor_connector_trait}.
Esencialmente, los plugins de tipo conector exportarán públicamente esta
interfaz en su binario y la runtime deberá ser capaz de cargarlo dinámicamente.
Inicialmente, todos los conectores disponibles se listaban y cargaban de forma
estática al inicio del programa.

El actor principal se llama \rust{World}. Contiene el estado del programa, como
los artefactos disponibles (\emph{repositorios}) y los que se están ejecutando
(\emph{registros}) y se usa para inicializar y controlar el programa.

Los \emph{managers} o \emph{gestores} son simplemente actores en el sistema que
envuelven una funcionalidad. Ayudan a desacoplar la comunicación de los detalles
de implementación. De esta forma, se puede eliminar código repetitivo en la
inicialización, como la creación de canales de comunicación o el lanzamiento del
componente en una tarea nueva. Generalmente, hay un gestor por cada tipo de
artefacto para facilitar su creación y también uno por cada instancia que se
esté ejecutando para controlar su comunicación.

Notar que la inicialización de los conectores ocurre en dos pasos. Primero se
\emph{registran}, es decir, se indica su disponibilidad para cargarlo
añadiéndolo al repositorio. Posteriormente, no se comenzará a ejecutar hasta
conectarse con otro artefacto con \rust{launch_binding}, lo cual lo movería del
repositorio al registro, junto al resto de artefactos ejecutándose.

\begin{figure}[h]
    \centering
    \begin{minted}{rust}
pub trait Connector {
    /// Crea la parte "source" del conector, si es aplicable.
    async fn create_source(
        &mut self,
        _source_context: SourceContext,
        _builder: source::SourceManagerBuilder,
    ) -> Result<Option<source::SourceAddr>> {
        Ok(None)
    }

    /// Crea la parte "sink" del conector, si es aplicable.
    async fn create_sink(
        &mut self,
        _sink_context: SinkContext,
        _builder: sink::SinkManagerBuilder,
    ) -> Result<Option<sink::SinkAddr>> {
        Ok(None)
    }

    /// Intenta conectarse con el mundo exterior. Por ejemplo, inicia la
    /// conexión con una base de datos.
    async fn connect(
        &mut self,
        _c: &ConnectorContext,
        _attempt: &Attempt
    ) -> Result<bool> {
        Ok(true)
    }

    /// Llamado una vez cuando el conector inicia.
    async fn on_start(&mut self, _c: &ConnectorContext) -> Result<()> {
        Ok(())
    }
    /// Llamado cuando el conector pausa.
    async fn on_pause(&mut self, _c: &ConnectorContext) -> Result<()> {
        Ok(())
    }
    /// Llamado cuando el conector continúa.
    async fn on_resume(&mut self, _c: &ConnectorContext) -> Result<()> {
        Ok(())
    }
    /// Llamado ante un evento de "drain", que se asegura de que no
    /// lleguen más eventos a este conector.
    async fn on_drain(&mut self, _c: &ConnectorContext) -> Result<()> {
        Ok(())
    }
    /// Llamado cuando el conector termina.
    async fn on_stop(&mut self, _c: &ConnectorContext) -> Result<()> {
        Ok(())
    }
}
    \end{minted}
    \caption{Simplificación del \trait \rust{Connector}}%
    \label{fig:tremor_connector_trait}
\end{figure}

\subsection{Registro}

La Figura~\ref{fig:tremor_registering} detalla todos los pasos seguidos en el
código. Primero se inicializan los gestores y a continuación se registran los
artefactos. Esta parte se realizaba de forma estática con
\rust{register_builtin_types}, pero después de implementar el PDK, debería
ocurrir dinámicamente. Tremor buscaría automáticamente plugins en sus
directorios configurados e intentaría registrar todos los que encuentre. En una
futura versión, el usuario podría solicitar manualmente el cargado de un plugin
nuevo mientras se está ejecutando Tremor.

\begin{figure}
    \centering
    \includegraphics[width=\textwidth]{./Imagenes/registering.pdf}
    \caption{Registro de un conector en el programa}%
    \label{fig:tremor_registering}
\end{figure}

\subsection{Inicialización}

Ya que es un proceso en múltiples pasos (en la implementación es más complicado
que registro y creación), la primera parte provee las herramientas para
inicializar el conector (un \builder). Cuando el conector necesite comenzar a
ejecutarse porque se haya añadido a una \pipeline, el \builder ayuda a construir
y configurarlo de forma genérica. Finalmente, se añade a una tarea asíncrona
nueva para que se pueda comunicar con otras partes de Tremor. El gestor
\rust{connectors::Manager} contiene todos los conectores ejecutándose en Tremor,
como se muestra en la Figura~\ref{fig:tremor_initializing}.

\begin{figure}
    \centering
    \includegraphics[width=\textwidth]{./Imagenes/initializing.pdf}
    \caption{Inicialización de un conector en el programa}%
    \label{fig:tremor_initializing}
\end{figure}

La Figura~\ref{fig:tremor_pipeline} muestra un ejemplo de una \pipeline,
definida con Troy, su propio lenguaje inspirado en SQL.

\begin{figure}
    \centering
    \begin{minted}[escapeinside=||]{mysql}
|\textcolor{blue}{define pipeline}| main
# El puerto `exit` no existe por defecto, así que tenemos que
# sobreescribir la selección de puertos incorporada.
into |out, exit|
|\textcolor{blue}{pipeline}|
  # Uso del módulo `std::string`.
  use std::|string|;
  use lib::scripts;

  # Creación de nuestro script.
  create |\textcolor{blue}{script}| punctuate from scripts::punctuate;

  # Filtrado de cualquier evento que sea un `"exit"` y enviarlo al
  # puerto de salida.
  select {"graceful": false} from |in| where event == "exit" into |exit|;

  # Conexión de nuestro texto convertido a mayúsculas al script.
  select |string|::capitalize(event) from |in| where event != "exit"
    into punctuate;
  # Conexión de nuestro script a la salida.
  select event from punctuate into |out|;
end;
    \end{minted}
    \caption{Ejemplo de una \pipeline definida para Tremor.}%
    \label{fig:tremor_pipeline}
\end{figure}

\subsection{Configuración}

Una vez haya un conector corriendo, la Figura~\ref{fig:tremor_setting_up}
visualiza cómo se divide en una parte \sink y otra \source. Estas son
opcionales, pero no exclusivas; se puede tener una de las dos o ambas. De forma
similar, un \builder se usa para inicializar las partes y a continuación lanza
una nueva tarea asíncrona para ellos.

\begin{figure}
    \centering
    \includegraphics[width=\textwidth]{./Imagenes/setting-up.pdf}
    \caption{Configuración de un conector en el programa}%
    \label{fig:tremor_setting_up}
\end{figure}

También se crea un gestor por cada instancia de \sink o \source, que se
encargará de la comunicación con otros actores. De esta forma, sus interfaces
pueden mantenerse lo más simples posibles. Esos gestores recibirán peticiones de
conexión de la \pipeline y posteriormente leerán o enviarán eventos en ella.

La diferencia principal entre \sources y \sinks a nivel de implementación es que
este último también puede responder a mensajes usando la misma conexión. Esto es
útil para notificar que el paquete ha llegado (\rust{Ack}) o que algo ha fallado
(\rust{Fail} para un evento específico, \rust{CircuitBreaker} para dejar de
recibir datos por completo).

Los códecs y preprocesadores se involucran aquí para tanto los \sources como los
\sinks. En la parte de \source, los datos son transformados a través de una
serie de preprocesadores y posteriormente se aplica un códec. Para los \sinks,
se sigue el proceso inverso: los datos se codifican primero a bytes con el códec
y posteriormente una serie de postprocesadores se aplican sobre los datos
binarios.

\subsection{Notas adicionales}

Algunos conectores se basan en \emph{flujos}. Son equivalentes a los flujos de
TCP, que agrupan paquetes para evitar entremezclarlos. Su inicio y fin es
marcado mediante mensajes asíncronos. Algunos preprocesadores puedan querer
guardar datos internos, así que el gestor se guarda el estado del flujo en un
campo llamado \rust{states}. Si un conector no necesita flujos, como
\rust{metronome} (que únicamente envía eventos periódicamente), puede
especificar su identificador de flujo como \rust{DEFAULT_STREAM_ID} siempre.

Tras terminar la interfaz de los conectores para el sistema de plugins, las
primeras implementaciones a desarrollar deberían ser:

\begin{itemize}
    \item \emph{Blackhole}, usado para analizar el rendimiento. Realiza
        mediciones de tiempos de final a final para cada evento pasando por la
        \pipeline, y al final guarda un histograma HDR (\emph{High Dynamic
        Range}).

    \item \emph{Blaster}, usado para repetir una serie de eventos de un archivo,
        que es especialmente útil para pruebas de rendimiento.

\end{itemize}

Ambos son relativamente simples y serán de gran ayuda para entender la
inevitable degradación de eficiencia causada por el sistema de plugins. De todos
modos, el equipo de Tremor insistía que lo más importante primero es que
funcione, y después me podría preocupar sobre eficiencia.
