% vim: spelllang=es

% Instrucciones:
% https://eina.unizar.es/sites/eina.unizar.es/files/archivos/secretaria/20210706_instrucciones_tfg_tfm.pdf
%
% TFGs previos:
% https://zaguan.unizar.es/search?ln=en&cc=trabajos-fin-grado&sc=1&p=inform%C3%A1tica&f=Departamento&action_search=Search
%
% Blog:
% https://nullderef.com/series/rust-plugins/
%
% NOTES:
% * Count words with `:TexWordCount`
% * Svg to pdf: `inkscape -D image.svg  -o image.pdf --export-latex`
%
% TODO:
% * Buscar typos (grep):
%   - Comas antes de conjunciones (y, e, o, u)
%   - 'Artifacto'
%   - Eliminar e.g. y i.e.
%   - Eliminar misusos de `así como`
%
% * Debería tener palabras anglosajonas con `\emph` siempre?
%
% * No sé cómo traducir onramp, offramp, sink y source de forma precisa.
%
% * Debería traducir runtime por ejecutor?
%
% * Debería incluir las figuras siempre en su sitio, donde las pone LaTeX por
% defecto, o al final de la sección?
%
% * "Figura" va en mayúsculas?

\documentclass[a4paper,12pt,twoside,hidelinks,openright]{book}

% \usepackage[T1]{fontenc}
\usepackage[spanish]{babel}
\usepackage[utf8]{inputenc}

\usepackage[usenames, dvipsnames]{color}
\usepackage{graphicx}
\usepackage{multirow}
\usepackage{graphics}
\usepackage{appendix}
\usepackage[nottoc,numbib]{tocbibind}
\usepackage{xspace}

\usepackage{tikz}
\usetikzlibrary{shapes,arrows}
\usepackage{subcaption}
\usepackage{amsmath,amsthm,amssymb,mathrsfs} 
\usepackage[final]{pdfpages}
\usepackage[]{placeins,flafter}
\usepackage[none]{hyphenat} \sloppy
\usepackage{xcolor}
\usepackage{adjustbox}
\usepackage{hyperref}

% Better table formatting
\usepackage{longtable} % Multi-page tables
\renewcommand*{\arraystretch}{1.5}
\usepackage{array} % Better alignment
\newcolumntype{P}[1]{>{\centering\arraybackslash}p{#1}}
\newcolumntype{M}[1]{>{\arraybackslash}m{#1}}

% Bibtex for the bibliography
\usepackage[
    backend=bibtex,
    block=space,
    language=spanish,
    sorting=none % Sort by appearance in the document
]{biblatex}
\bibliography{Bibliografia_TFG}

% Easier to read font
\usepackage{fourier} % For math
\usepackage{fontspec}
\setmainfont{Heuristica} % For the text
\setmonofont{Liberation Mono} % For the code

% Less margin between lists, otherwise after overriding \parskip and etc it's
% too much.
% TODO: decidir si hacer párrafos estilo TeX o convencionales.
\usepackage{enumitem}
\setlist{topsep=0pt}

% Code syntax highlighting. The cache is hardcoded with an absolute path for the
% Tectonic engine until this is fixed:
% https://github.com/tectonic-typesetting/tectonic/issues/835
\usepackage[
    cachedir=/home/mario/Downloads/minted/
]{minted}
\usepackage{xcolor}
\usemintedstyle{friendly}
\definecolor{bg}{HTML}{f8f8f8}
\setminted{
    frame=lines,
    framesep=2mm,
    bgcolor=bg,
    fontsize=\footnotesize,
    linenos
}

% Remove red boxes around illegal characters in minted, taken from:
% https://tex.stackexchange.com/a/343506
%
% This way, `mysql` can be used to highlight tremorscript as a quick hack.
\makeatletter
\AtBeginEnvironment{minted}{\dontdofcolorbox}
\def\dontdofcolorbox{\renewcommand\fcolorbox[4][]{##4}}
\makeatother

% Command for inline code
\newmintinline[code]{text}{fontsize=\normalsize}
\newmintinline[rust]{rust}{fontsize=\normalsize}
\newcommand{\image}[1] {
  \begin{figure}[H]
    \centering
    \includegraphics[width=\textwidth]{#1}
  \end{figure}
}
% Shortcuts
\newcommand{\cpp}{C\texttt{++}}
\newcommand{\cratelink}[1] {%
    \code{#1}\footnote{\url{https://crates.io/crates/#1}}\xspace%
}
% Gracias unizar por no evolucionar y solo permitir memorias en castellano...
\newcommand{\unsafe}{\code{unsafe}\xspace}
\newcommand{\crate}{\emph{crate}\xspace}
\newcommand{\crates}{\emph{crates}\xspace}
\newcommand{\onramp}{\emph{onramp}\xspace}
\newcommand{\onramps}{\emph{onramps}\xspace}
\newcommand{\offramp}{\emph{offramp}\xspace}
\newcommand{\offramps}{\emph{offramps}\xspace}
% TODO: sink/source sí que se podrían traducir por salida/entrada
\newcommand{\sink}{\emph{sink}\xspace}
\newcommand{\sinks}{\emph{sinks}\xspace}
\newcommand{\source}{\emph{source}\xspace}
\newcommand{\sources}{\emph{sources}\xspace}
\newcommand{\pipeline}{\emph{pipeline}\xspace}
\newcommand{\pipelines}{\emph{pipelines}\xspace}
\newcommand{\builder}{\emph{builder}\xspace}
\newcommand{\builders}{\emph{builders}\xspace}
\newcommand{\lifetime}{\emph{lifetime}\xspace}
\newcommand{\lifetimes}{\emph{lifetimes}\xspace}
\newcommand{\websockets}{\emph{websockets}\xspace}
\newcommand{\trait}{\emph{trait}\xspace}
\newcommand{\traits}{\emph{traits}\xspace}
\newcommand{\struct}{\emph{struct}\xspace}
\newcommand{\structs}{\emph{structs}\xspace}
\newcommand{\leak}{\emph{leak}\xspace}
\newcommand{\leaks}{\emph{leaks}\xspace}
\newcommand{\scripting}{\emph{scripting}\xspace}
\newcommand{\sandbox}{\emph{sandbox}\xspace}
\newcommand{\sandboxing}{\emph{sandboxing}\xspace}
\newcommand{\socket}{\emph{socket}\xspace}
\newcommand{\sockets}{\emph{sockets}\xspace}
\newcommand{\stdin}{\emph{stdin}\xspace}
\newcommand{\stdout}{\emph{stdout}\xspace}
\newcommand{\stderr}{\emph{stderr}\xspace}
\newcommand{\abistable}{\code{abi_stable}\xspace}
\newcommand{\csharp}{C\#\xspace}
% TODO: lo podría traducir a 'tuberías' pero suena terrible
\newcommand{\pipe}{\emph{pipe}\xspace}
\newcommand{\pipes}{\emph{pipes}\xspace}

%	CONFIGURACIÓN DE PÁGINA

\setlength{\paperwidth}{21cm}          % Ancho de página
\setlength{\paperheight}{29,7cm}       % Alto de página
\setlength{\textwidth}{15.5cm}         % Ancho de zona con texto
\setlength{\textheight}{24.6cm}        % Ancho de zona con texto
\setlength{\topmargin}{-1.0cm}         % Margen superior
                                      
\setlength{\oddsidemargin}{0.46cm}     % Margen izquierdo 
\setlength{\evensidemargin}{0.46cm}    

\usepackage{makeidx}
\makeindex
\index{key}
\newcommand{\myref}[1]{\color{red}\bf(\ref{#1})}


\begin{document}

% PORTADA
\begin{titlepage}

\definecolor{unizarblue}{RGB}{0, 86, 153}

\vspace*{-4mm}
\begin{figure}[!h]
  \centering
	\includegraphics[width=69.62mm]{Imagenes/UnizarLogo}
\end{figure}

\vspace*{17mm}

\fontsize{28pt}{28pt}\selectfont
\begin{center}
\setlength{\fboxsep}{3.4mm}
\adjustbox{minipage=14.4cm,cfbox=unizarblue,center}{\begin{center} Trabajo Fin de Grado \end{center}}
\end{center}

\vspace*{18.7mm}

\fontsize{22pt}{22pt}\selectfont
\begin{center}
\textsc{Cargado dinámico de plugins en Rust\\
en ausencia de estabilidad en la\\
Interfaz Binaria de Aplicación}
\end{center}

\vspace*{0.75cm} 

\fontsize{22pt}{22pt}\selectfont
\begin{center}
\textsc{Dynamic loading of plugins in Rust \\
in the absence of a stable \\
Application Binary Interface}
\end{center}

\vspace*{2cm} 
\baselineskip 36pt
\begin{center}

\fontsize{14pt}{14pt}\selectfont
\center{\rm Autor:}
\vspace*{0mm} 
\fontsize{18pt}{18pt}\selectfont
\center{\textsc{Mario Ortiz Manero}}

\baselineskip 36pt

\fontsize{14pt}{14pt}\selectfont
\center{\rm  Director:}
\vspace*{0mm}
\fontsize{15pt}{15pt}\selectfont
\center{\textsc{Javier Fabra Caro}}

\end{center}

\setcounter{footnote}{1}

\vspace*{2cm}
\fontsize{14pt}{14pt}\selectfont
\begin{center}
Grado en Ingeniería Informática\\ \medskip
Departamento de Ingeniería e Ingeniería de Sistemas\\
Escuela de Ingeniería y Arquitectura\\ \bigskip
Junio 2022\\
\end{center}


\renewcommand{\thefootnote}{\arabic{footnote}}
\pagenumbering{gobble}
\end{titlepage}
\newpage


\title{Dynamic loading of plugins in Rust in the absence of a stable Application Binary Interface}
\author{Mario Ortiz Manero}

\pagebreak
\cleardoublepage%
\baselineskip 19pt

\renewcommand{\labelitemi}{$-$}
\renewcommand{\tablename}{Tabla}

\renewcommand{\appendixname}{Anexos}
\renewcommand{\appendixtocname}{Anexos}
\renewcommand{\appendixpagename}{Anexos}


\pagenumbering{Roman}

% Párrafos de forma más convencional, me parece más fácil leerlo así.
\begingroup
\setlength{\parskip}{\baselineskip}%
\setlength{\parindent}{0pt}%

\newpage
\cleardoublepage%
% vim: spelllang=es

\begin{center}
{\LARGE \bfseries AGRADECIMIENTOS}
\vspace{2.5cm}
\end{center}

Un primer gracias a toda mi familia por apoyarme en cualquier momento que lo
necesitara. En especial, a mi padre y a mi madre por aguantarme siempre y por
dejarse la piel para que pueda estudiar.

A mis amigos que me han acompañado toda la vida con tan buenos momentos, y a los
de la universidad, que fueron la verdadera motivación a ir a clase y seguir día
a día. Las horas incontables juntos en la biblioteca, en el bar de la EINA o
incluso en los montes más recónditos de Juslibol a las dos de la madrugada, han
hecho que estos años merezcan la pena.

También a mis profesores por orientarme, particularmente a Javier Fabra por
ayudarme con mi Erasmus, este documento y todas sus complicaciones. Mis mentores
de Tremor tomaron un papel fundamental, no solo dándome consejo para el
proyecto, sino también para mi carrera profesional y mi vida.

Finalmente, agradecer a todas las organizaciones que han hecho esto posible. A
la Fundación de Linux por ofrecer los medios. A Wayfair por apostar sus fondos
en código abierto y talento joven. Y a la comunidad de código abierto, que me ha
motivado a programar desde el principio y me ha guiado hasta donde estoy hoy.


\newpage
\cleardoublepage%
% vim: spelllang=es

\begin{center}
{\LARGE \bfseries RESUMEN}

\vspace{2.5cm}
\end{center}

TODO, en castellano


\newpage
\cleardoublepage%
% vim: spelllang=en

\begin{center}
{\LARGE \bfseries ABSTRACT}

\vspace{2.5cm}
\end{center}

TODO: en inglés cuando esté corregido el castellano

\endgroup

\newpage
\cleardoublepage%
\renewcommand{\contentsname}{Índice}
\tableofcontents

\newpage
\renewcommand\listfigurename{Lista de Figuras}
\listoffigures

% TODO: quitar si no he acabado usando ninguna
\newpage
\renewcommand\listtablename{Lista de Tablas}
\listoftables

% Capitulos

% Vuelta a configurar los párrafos para que no afecte al índice
\begingroup
\setlength{\parskip}{\baselineskip}%
\setlength{\parindent}{0pt}%

% vim: spelllang=es

\chapter{Introducción}
\pagenumbering{arabic}

\section{Contexto}

Este proyecto se ha realizado en colaboración con
\emph{Tremor}\footnote{\url{https://tremor.rs}}, un sistema de procesado de
eventos de alto rendimiento, escrito en el lenguaje de programación
\emph{Rust}\footnote{\url{https://www.rust-lang.org}}. Tremor es un programa de
código abierto bajo la fundación \emph{Cloud Native Computing
Foundation~(CNCF)}\footnote{\url{https://www.cncf.io/}}, que es también parte de
la organización \emph{Linux
Foundation~(LFX)}\footnote{\url{https://www.linuxfoundation.org/}}.

Formalmente, el trabajo se ha llevado a cabo gracias a la iniciativa \emph{LFX
Mentorship}, con el título ``CNCF - Tremor: Add plugin support for tremor
(PDK)''\footnote{\url{https://mentorship.lfx.linuxfoundation.org/project/b90f7174-fc53-40bc-b9e2-9905f88c38ff}}.
Esta iniciativa promueve el aprendizaje de desarrolladores de código abierto,
proporcionando una plataforma transparente, y facilitando un sistema de pagos.

Finalmente, \emph{Wayfair} es una empresa estadounidense de comercio digital de
muebles y artículos del hogar\footnote{\url{https://www.wayfair.com/}}.
Actualmente, ofrece 14 millones de ítems de más de 11.000 proveedores
globales~\cite{wayfairItems}, y es el principal financiador tanto de Tremor como
de este proyecto.

\section{Objetivo}

La tarea a llevar a cabo es la implementación de un sistema de plugins para la
base de código ya existente, lo cual es una tarea no-trivial, dado que Rust no
tiene un \emph{Application Binary Interface~(ABI)} estable.

% TODO: debería mencionar que explicaré luego qué es el ABI?

\section{Motivación}

\subsection{Tiempos de compilación}

% TODO: incluir modo release?
Actualmente, el problema más importante en Tremor es sus tiempos de compilación.
En un ordenador de gama media de \~{}600 € como el Dell Vostro 5481, compilar el
binario \code{tremor} desde cero requiere de más de 7 minutos en modo debug.
Incluso en el caso de cambios incrementales (una vez las dependencias ya han
sido compiladas), hay que esperar unos 10 segundos. Esto no es una buena
experiencia de desarrollo y previene que nuevos ingenieros se unan a la
comunidad de Tremor.

Debido a la naturaleza del programa, este problema solo podrá empeorar con el
tiempo. Tremor debe tener soporte para un gran número de protocolos (e.g., TCP
o UDP), software (e.g., Kafka o PostgreSQL), y codecs (e.g., JSON o YAML). El
número de dependencias continuará incrementando hasta que impida la creación de
nuevas prestaciones en Tremor.

Los problemas relacionados con tiempos de compilación excesivamente largos no se
limitan a Tremor. Es uno de las mayores críticas que recibe Rust, y un 61\% de
sus usuarios declaran que aún se necesita trabajo para mejorar la situación
\cite{rustsurvey}.

\subsection{Modularidad}

Otra ventaja que provee un sistema de plugins es modularidad; ser capaz de
tratar la \emph{runtime} y los \emph{plugins} de forma separada suele resultar
en una arquitectura más limpia \cite{baldwin2000design}.

También hace posible el desacoplamiento del ejecutable y sus componentes.
Algunas dependencias tienen un ciclo de versiones más rápido que otras, y
generalmente es más conveniente actualizar únicamente un plugin, en lugar de el
programa por completo.

\subsection{Aprender de otros}

Otros proyectos maduros con características similares a las de Tremor, como
\textcite{nginx} o \textcite{apachehttpserver}, han estado aprovechando las
ventajas de un sistema de plugins desde hace mucho. Informan mejorías en
flexibilidad, extensibilidad, y facilidad de
desarrollo~\cite{nginxPluginsAdvantages}\cite{apachePluginsAdvantages}. Aunque
las desventajas también mencionen un pequeño impacto en el rendimiento, y la
posibilidad de caer en un \emph{dependency hell}, sigue siendo una buena idea al
menos considerarlo para Tremor.

\section{Metodología}

El proyecto ha tenido una duración de unos 10 meses, comenzando en agosto de
2021, y terminando en mayo/junio de 2022. Su realización ha sido completamente
remota y con horarios muy flexibles.

% TODO: debería ser más específico sobre las horas? Mucha cuenta no he llevado
% pero vamos sí que estoy segurísimo de que han sido 300 horas, y mucho más
% también.

% vim: spelllang=es

\chapter{Breve introducción a Rust}\label{ch:rust}

Dado que Rust es un lenguaje de programación que tan solo anunció su primera
versión en 2015, aún no es conocido por muchos desarrolladores. Este proyecto
requiere ser familiar con cómo funciona, por lo que en este capítulo se
introducirán los conceptos más básicos necesarios. Sí que se asume conocimiento
de lenguajes de propósito general, como C, C++, Python o Java.

Sin embargo, es posible que se omitan algunos conceptos o que algunas
explicaciones no sean completamente precisas por razones de simplicidad.
\textcite{rustfullbook} es el libro oficial para aprender Rust por completo,
pero es una lectura larga y posiblemente demasiado exhaustiva. Para mayor
brevedad, se recomienda leer \textcite{rustprofessionals},
\textcite{rustgentleintro} o \textcite{rust30min}.

La comunidad dispone de otros libros que explican aspectos más avanzados del
lenguaje en específico, como \unsafe o la programación asíncrona. En esos casos,
se recomienda leer \textcite{rustnomicon} y \textcite{rustasyncbook},
respectivamente.

TODO: revisar traducciones del libro o similares para asegurarse de que la
terminología es la misma.

\section{¿Qué es Rust?}

Rust es un lenguaje de programación de sistemas compilado y de propósito
general. Su objetivo es maximizar rendimiento y usabilidad, esto último
basándose en seguridad integrada en el lenguaje, en vez de en el \

\section{Primeros pasos}

Comenzando por el clásico ``Hola Mundo'', se incluyen algunos ejemplos de cómo
es la sintaxis de Rust más básica. El programa se podría ejecutar fácilmente con
\emph{Cargo}, el administrador de dependencias oficial, o específicamente con el
comando \code{cargo run}.

\begin{minted}{rust}
fn main() {
    println!("Hello World!");
}
\end{minted}

\code{main} es nuestra función principal, que invoca al macro \emph{println}
para escribir por pantalla. Esto se sabe porque, a diferencia de una llamada a
función, la invocación termina con una exclamación (\code{!}).

Los bloques básicos (\code{if}, \code{else}, \code{while}, \code{for}) son los
mismos que en otros lenguajes, con la introducción de \code{match}, que permite
extraer patrones.

\begin{minted}{rust}
fn factorial(i: u64) -> u64 {
    match i {
        0 => 1,
        n => n * factorial(n-1)
    }
}
\end{minted}

\section{Tipos de datos básicos}

* Struct
* Enum
* Trait

\section{Librería estándar}

De forma similar a C++, Rust posee tipos genéricos. Esto permite la
implementación de una librería estándar flexible, con varias estructuras de
datos importantes a conocer:

\begin{itemize}
    \item Tipos primitivos:
        \begin{itemize}
            \item Carácteres con \code{char}.

            \item Punto flotante con \code{f32} y \code{f64}.

            \item Booleanos con \code{bool}.

            \item Enteros: \code{u8}, \code{i8}, \code{u16}, \code{i16},
                \code{u32}, \code{i32}, \code{u64}, \code{i64}, e incluso
                \code{i128} y \code{u128} en las arquitecturas que lo soportan.

            \item Vectores de tamaño fijo: por ejemplo \code{[1, 2, 3, 4, 5]}.

            \item N-tuplas como \code{(1, true, 9.2)}.

            \item El tipo ``unidad'', \code{()}, equivalente a \code{void} en C
                o C++.

            \item Punteros básicos con \code{*const T} o \code{*mut T}.

        \end{itemize}

    \item \code{Vec<T>} representa un vector contiguo y redimensionable.

    \item \code{HashMap<K, V>} es una tabla hash, genérica respecto a su clave
        \code{K} y su valor \code{V}. No se encuentra en el preludio, por lo que
        requeriría la siguiente declaración, similar a un \code{import} de Java:

\begin{minted}{rust}
use std::collections::HashMap;
\end{minted}

    \item \code{Box<T>}, usado para localizar un tipo \code{T} en memoria.
        Incluye el tamaño que ocupa \code{T}.

    \item \code{str} es una cadena UTF-8 de solo lectura, típicamente usada con
        una referencia \code{&str}. Va acompañada por su longitud, por lo que no
        hace falta terminarla con \code{\0}, a diferencia de C. \code{String} es
        su versión modificable asignada en memoria.

\end{itemize}

\section{Gestión de errores}

En Rust, los errores se indican con el tipo \code{Result<T, E>}. Este se trata
de una enumeración cuyo valor puede ser \code{Ok(T)}, con el resultado obtenido
satisfactoriamente, o \code{Err(E)}, con el tipo de error que ha sucedido. Dado
que es un tipo nuevo, si el programador se olvidase de comprobar errores, el
programa no compilaría. Se puede usar \code{match} para comprobar el resultado,
o una serie de funciones disponibles para hacer el proceso más ergonómico:

\begin{minted}{rust}
match load_file(input) {
    Ok(data) => /* ... */,
    Err(e) => eprintln!("Error: {e}"),
}
\end{minted}

En caso de que se produjera un error del que el programa no se puediera
recuperar, como quedarse sin memoria o un fallo inesperado en la implementación,
se usa la funcionalidad de \emph{pánicos}. Un pánico se propaga de forma similar
a una excepción de C++ o Java, y terminará la ejecución por completo. Esto se
puede invocar con el macro \code{panic!} o utilidades similares.

\section{Macros}

Rust da soporte para dos tipos de macros: \emph{declarativos} y
\emph{procedurales}. Ambos permiten generar código a tiempo de compilación, pero
se diferencian principalmente en la flexibilidad que ofrecen, a coste de un
coste de desarrollo mayor o menor.

Los macros declarativos se crean con una sintaxis especializada, similar a un
\code{match} con patrones de tokens como entrada, y los tokens modificados como
salida. Son similares a los macros de C o C++, pero más potentes e higiénicos
(i.e., su expansión no captura identificadores accidentalmente).

Los macros procedurales se describen como extensiones del lenguaje.
Esencialmente, ejecutan código en la compilación que consume y produce sintaxis
de Rust. Consisten en directamente transformar el Árbol de Sintaxis Abstracta
(AST)~\cite{procmacrosref}. Consecuentemente, su complejidad es mucho mayor,
pero expanden las posibilidades de los macros enormemente.

\begin{minted}{rust}
some_macro!(1, 2, 3); // Puede ser tanto declarativo como procedural
\end{minted}

\begin{minted}{rust}
// Sintaxis típica de invocación de un macro
some_macro! {
    fn some_function() { /* ... */ }
}

// También permitido en el caso de los procedurales
#[some_macro]
fn some_function() { /* ... */ }
\end{minted}

Finalmente, los macros procedurales se pueden declarar de forma que
\emph{deriven} (implementen automáticamente) un \trait. Esto evita escribir
código repetitivo de forma muy sencilla:

\begin{minted}{rust}
// Con un macro `derive` para el trait `Debug`, que sirve para
// mostrar variables por pantalla.
#[derive(Debug)]
struct X(i32);

// Sin ellos sería lo siguiente. Como es trivial se puede
// simplificar en un macro procedural de tipo `derive`.
impl fmt::Debug for X {
    fn fmt(&self, f: &mut fmt::Formatter) -> fmt::Result {
        write!(f, "{:?}", self.0)
    }
}
\end{minted}

\section{Programación asíncrona}

Como muchos lenguajes modernos, Rust da soporte a la programación asíncrona, un
modelo de programación concurrente. Sin entrar en excesivo detalle, esta permite
tener una gran cantidad de \emph{tareas} concurrentes ejecutándose sobre unos
pocos hilos del Sistema Operativo. Su caso de uso principal es programas cuyo
rendimiento está limitado por operaciones de entrada y salida, como servidores o
bases de datos~\cite{rustasyncbook}.

\begin{minted}{rust}
// Con `async` se indica que la función es asíncrona.
async fn get_two_sites_async() {
    // Creación de dos "futuros" que, al completarse, descargarán
    // asíncronamente las páginas web. Similar a la creación de
    // un nuevo hilo.
    let future_one = download_async("https://www.foo.com");
    let future_two = download_async("https://www.bar.com");

    // Ejecutar dos tareas. similar a esperar la terminación de
    // los hilos.
    join!(future_one, future_two);

    // Con `.await` se puede esperar a la terminación de un futuro
    // individual.
    let future_three = download_async("https://www.bar.com").await;
}
\end{minted}

% vim: spelllang=es

\chapter{Tecnologías para implementar un sistema de plugins}


% vim: spelllang=es

\chapter{Implementación}


\begin{figure}
    \centering
    \includegraphics[width=\textwidth]{./Imagenes/separation-temporary.pdf}
    \caption{Ejemplo de uso de Tremor}%
    \label{fig:example_tremor}
\end{figure}

\begin{figure}
    \centering
    \includegraphics[width=\textwidth]{./Imagenes/separation.pdf}
    \caption{Ejemplo de uso de Tremor}%
    \label{fig:example_tremor}
\end{figure}

\begin{figure}
    \centering
    \includegraphics[width=\textwidth]{./Imagenes/simplify.pdf}
    \caption{Ejemplo de uso de Tremor}%
    \label{fig:example_tremor}
\end{figure}

% vim: spelllang=es

\chapter{Implementación}

\section{Prototipado eficiente}

\newcommand{\work}{``Primero que Funcione''\xspace}

Antes de nada, es importante aprender un poco sobre cómo realizar cambios en el
código de Tremor eficientemente. Este proyecto modificará gran cantidad de
líneas y cuanto más rápido sea el desarrollo, menos problemas habrán. Esto se
puede cubrir de forma específica al lenguaje Rust, con trucos o consejos que
puedan facilitar el desarrollo, o de forma más general, con la estrategia de
trabajo a seguir. En esta sección se cubrirá lo último, dado que es menos un
detalle de implementación.

TODO: podría mencionar trucos relacionados con Rust en detalle (desactivar
algunos warnings, o quitar statements \code{use}), pero no creo que sea tan
importante en este caso y el documento ya es bastante largo.

La metodología fue insipirada por mis mentores, que lo denominaron el ``Just
Make it Work'', o \work. Con lo que más problemas tenía era el perderme en los
detalles. Pero ciertamente, primero de todo lo importante es que funcione.
Siempre y cuando el sistema de plugins se pueda compilar y ejecutar, lo
siguiente es secundario:

TODO: alguna traducción de "just make it work" un poco más natural? "Solo que
funcione"? "Primero que funcione?

\begin{itemize}
    \item Código ``feo'' (no idiomático, repetitivo o desordenado).

    \item Código de bajo rendimiento.

    \item Documentación pobre.

    \item No tener tests.

    \item Sin aplicar sugerencias aplicar por \emph{linters} (en el caso de
        Rust, \emph{Clippy}).

\end{itemize}

TODO: el siguiente párrafo puede no gustarle a alguno de ingeniería del software
porque rompe todas las metodologías de desarrollo que tienen, pero fue como
sinceramente ocurrió.

El no trabajar con tests es discutible, dado que depende de si el programador
prefiere seguir un desarrollo basado en tests. Sin embargo, personalmente no
sentí la necesidad de escribir ningún test en mi caso; gracias al sistema
fuertemente tipado de Rust, fue principalmente un desarrollo basado en el
compilador. Mi progreso se basaba en realizar algunos cambios y posteriormente
intentar que los aceptara el compilador, repetidamente. Únicamente avancé a la
parte de tests cuando todo parecía funcionar manualmente y estaba lo
suficientemente satisfecho con el resultado.

Adicionalmente, las optimizaciones prematuras son la fuente de todos los
problemas. No es algo que sea importante aún. Una vez terminada la primera
iteración, se puede dedicar más tiempo a medir el rendimiento para saber cuáles
optimizaciones merecen la pena. Notar que sí que sí que debería preocuparse en
escoger un método o tecnología que sea apropiado en términos de rendimiento; por
ello se descartó WebAssembly o IPC en el paso anterior. Pero definitivamente el
desarrollador debería rendirse en, por ejemplo, evitar una conversión de tipos
posiblemente innecesaria que posiblemente no afecte al rendimiento al fin y al
cabo.

Lo que quería dejar claro el equipo de Tremor es que todos los tests, limpiezas
u optimizaciones que intentes realizar en este momento acabará muy probablemente
siendo en vano. Se llegará a un punto en el que no se pueda continuar y que
requiera repensar y reescribir gran parte del trabajo. Cuando todo compile y
aparentemente funcione correctamente, se puede dedicar esfuerzo a trabajar en
estos temas secundarios. Si algo no importante está llevando demasiado tiempo,
se debería marcar como TODO o FIXME y dejarlo para otro momento.

Notar que no hay problema con ``gastar'' el tiempo con métodos que acaban siendo
incorrectos, porque realmente no se está ``gastando'' nada; son un paso
necesario para llegar a la solución final. Pero es doloroso tener que eliminar
código al que le has dedicado tiempo, así que al menos debería intentarse
minimizar las veces que esto ocurra.

\section{\code{abi_stable}}

Dado que \code{abi_stable} va a ser la librería principal en la que se basará el
sistema de plugins, es importante entender cómo funciona al completo. Además de
conocer los detalles de implementación, es importante conocer cómo \abistable
soluciona los problemas a tener en cuenta para implementar un sistema de
plugins:

\subsection{Versionado}

\subsection{Cargado de plugins}

\subsection{Exportando un plugin}

\subsection{Gestión de pánicos}

\subsection{Programación asíncrona}

* \cratelink{async_ffi}

\subsection{Seguridad en hilos}

\subsection{Rendimiento}

\section{Conversión al ABI de C}

El primer paso en el proceso es declarar toda la interfaz del PDK de forma que
use el ABI de C, en vez del de Rust. Esto se puede hacer con el atributo
\code{#[repr(C)]} (en lugar del \code{#[repr(Rust)]} implícito), pero el
problema reside en que todos los tipos dentro suyo \emph{también} tendrán que
haber sido declarados con dicho atributo:

\subsection{Problemas con tipos externos}

Por ilustrarlo mejor, la estructura que más problemas dio al respecto fue
\code{Value}, usado para representar datos pseudo-JSON y definido a continuación
(aproximadamente):

\begin{minted}{rust}
pub enum Value {
    /// Valores estáticos (enteros, booleanos, etc)
    Static(StaticNode),
    /// Tipo para cadenas de caracteres
    String(String),
    /// Tipo para listas
    Array(Vec<Value>),
    /// Tipo para objetos (mapas clave-valor)
    Object(Box<HashMap<String, Value>>),
    /// Tipo para datos binarios
    Bytes(Vec<u8>),
}
\end{minted}

Para poder usar \code{Value} en la interfaz del sistema de plugins, se puede
convertir a:

\begin{minted}{rust}
#[repr(C)] // La representación en memoria de Value seguirá el ABI de C
#[derive(StableAbi)] // Solo necesario cuando se usa abi_stable
pub enum Value {
    Static(StaticNode),
    /// Ahora usa `RString`, la altenativa a `String` de abi_stable
    String(RString),
    /// De forma similar, usa `RVec` en vez de `Vec`
    Array(RVec<Value>),
    /// Cambio de `Box`, `HashMap` y `String` por sus alternativas
    Object(RBox<RHashMap<RString, Value>>),
    /// Otro cambio de `Vec`
    Bytes(RVec<u8>),
}
\end{minted}

El primer problema surge en la variante \code{Static}: \code{StaticNode} es un
tipo externo con \code{#[repr(Rust)]}. Se declara en el \crate
\cratelink{value_trait}, que lo declara tal que:

\begin{minted}{rust}
pub enum StaticNode {
    I64(i64),
    U64(u64),
    F64(f64),
    Bool(bool),
    Null,
}
\end{minted}

Esto se podría arreglar siguiendo el mismo procedimiento recursivamente hasta
que todo sea \code{#[repr(C)]}. Pero como se trata de una librería externa,
tendrá que abrirse un nuevo pull request y esperar que al autor le parezcan bien
los cambios~\cite{openstaticnode}:

\begin{minted}{rust}
#[cfg_attr(feature = "abi_stable", repr(C))]
#[cfg_attr(feature = "abi_stable", derive(abi_stable::StableAbi))]
pub enum StaticNode {
    I64(i64),
    U64(u64),
    F64(f64),
    Bool(bool),
    Null,
}
\end{minted}

El atributo \code{cfg_attr} se asegura de que la estructura es \code{#[repr(C)]}
únicamente cuando opcionalmente se configure a tiempo de compilación como
necesario. De esta forma, el resto de usuarios podrán seguir aprovechándose de
las ventajas de rendimiento que puede ofrecer \code{#[repr(Rust)]}.

\subsection{}

Por desgracia, este cambio no termina ahí; cambiar las variantes de \code{Value}
implica que el código que lo usaba se romperá de numerosas formas:

\begin{minted}{rust}
// No funcionará porque Value::Array contiene un RVec ahora
let value = Value::Array(Vec::new());
\end{minted}

\subsection{Cuando no es tan fácil como añadir \code{#[repr(Rust)]}}

\subsection{Problemas con varianza y subtipado}

Otro problema muy importante 

\section{Separación de runtime e interfaz}

\section{Despliegue en producción}

\section{Lecciones aprendidas}

* Quizá incluir consejos de los mentores?

\begin{figure}
    \centering
    \includegraphics[width=6cm]{./Imagenes/separation-temporary.pdf}
    \caption{Ejemplo de uso de Tremor}%
    \label{fig:separation_temporary}
\end{figure}

\begin{figure}
    \centering
    \includegraphics[width=7cm]{./Imagenes/separation.pdf}
    \caption{Ejemplo de uso de Tremor}%
    \label{fig:separation}
\end{figure}

\begin{figure}
    \centering
    \includegraphics[width=\textwidth]{./Imagenes/simplify.pdf}
    \caption{Ejemplo de uso de Tremor}%
    \label{fig:simplify}
\end{figure}

% vim: spelllang=es

\chapter{Conclusiones y trabajo futuro}

\section{Concusiones}

Como se ha explicado, la complejidad del proyecto resultó ser mucho mayor que lo
esperado por problemas con el ABI. Por tanto, resultó imposible desarrollar en
el tiempo disponible un sistema de plugins tan completo y eficiente como se
especificaba.

La última versión del sistema de plugins es funcional y, mediante contribuciones
de código abierto, ha hecho posible su futura inclusión en una versión de
Tremor. Muchas de las librerías usadas no disponían de la funcionalidad
necesaria para este proyecto, así como \code{async_ffi}, \code{abi_stable},
\code{halfbrown} o \code{simd-json}.

El problema principal tiene que ver con el rendimiento. Dada la naturaleza de
Tremor, es un requerimiento imprescindible para poderlo incluir en una futura
versión. Tras unas mediciones iniciales, se calculó que su inclusión reducía el
rendimiento del programa un 30\%, así que los siguientes pasos consistieron en
reducir dicha cifra.

TODO: incluir todas las figuras de benchmarks con las explicaciones

\section{Futuro}

\section{Valoración personal}

\endgroup

% BIBLIOGRAFÍA Y REFERENCIAS
\printbibliography%
\nocite{*} % Include all the entries in the bibliography, even if not mentioned

% ANEXOS

% Vuelta a configurar los párrafos para que no afecte al índice
\setlength{\parskip}{\baselineskip}%
\setlength{\parindent}{0pt}%

\newpage
\appendix
\clearpage
\addappheadtotoc%
\appendixpage%
\chapter{Guía de Rust}\label{annex:rust}

Es posible que en este anexo se omitan algunos conceptos o que algunas
explicaciones no sean completamente precisas por razones de simplicidad.
\textcite{rustbook} es el libro oficial para aprender Rust por completo, pero es
una lectura larga y posiblemente demasiado exhaustiva. Para mayor brevedad, se
recomienda leer \textcite{rustprofessionals}, \textcite{rustgentleintro} o
\textcite{rust30min}.

La comunidad dispone de otros libros que explican aspectos más avanzados del
lenguaje en específico, como \unsafe o la programación asíncrona. En esos casos,
se recomienda leer \textcite{nomicon} y \textcite{rustasyncbook},
respectivamente.

\section{Primeros pasos}

Comenzando por el clásico ``Hola Mundo'', se incluyen algunos ejemplos de cómo
es la sintaxis de Rust más básica. Los binarios o librerías en Rust reciben el
nombre de \crate. Nuestra \crate se podría ejecutar fácilmente con \emph{Cargo},
el administrador de dependencias oficial, o específicamente con el comando
\code{cargo run}.

\begin{minted}{rust}
fn main() {
    println!("Hello World!");
}
\end{minted}

\code{main} es nuestra función principal, que invoca al macro \code{println!}
para escribir por pantalla. Notar que la invocación de macros, a diferencia de
funciones, requiere un \rust{!} al final.

\section{Conceptos principales}

Los bloques básicos (\rust{if}, \rust{else}, \rust{while}, \rust{for}) son muy
similares a en otros lenguajes. También existe \rust{match}, que permite extraer
patrones de variables.

\begin{minted}{rust}
fn factorial(i: u64) -> u64 {
    match i {
        // Primer caso: i = 0
        0 => 1,
        // El resto de casos, asignado a una variable `n`
        n => n * factorial(n-1)
    }
}
\end{minted}

Uso de variables y métodos:

\begin{minted}{rust}
fn main() {
    // Declaración de una variable, cuyo tipo se infiere
    // automáticamente.
    let my_number = 1234;
    // Declaración de una variable con un tipo especificado
    // manualmente. Notar que se puede usar el mismo nombre, y la
    // variable anterior será destruida.
    let my_number: i32 = 4321;
    // Invocación de la función estática (constructor) `new` dentro
    // del tipo `String`. El uso de `mut` indica que la instancia del
    // tipo se puede modificar. Funciona de forma inversa a C++, que
    // por defecto es mutable y `const` indica que *no* se puede
    // modificar.
    let mut my_str = String::new();
    // Invocación del método `push` de `my_str`, que añade un
    // carácter al final de la cadena.
    my_str.push('a');
}
\end{minted}

Otros componentes principales de Rust son:

\begin{itemize}
    \item Estructuras de datos:

\begin{minted}{rust}
struct MessageA {
    // Campo público con una cadena de caracteres
    pub text: String,
    // Campo privado con un entero
    user_id: i32,
}
\end{minted}

\begin{minted}{rust}
// Sin nombres de campos; se pueden acceder con `data.0`
// y `data.1`, respectivamente.
struct MessageB(pub String, i32);
\end{minted}

    \item Enumeraciones, que también permiten contener datos:

\begin{minted}{rust}
enum MessageC {
    Join,
    Text(String, i32),
    Leave(i32),
}
\end{minted}

    \item \emph{Traits}, similares a las interfaces de Java en el sentido de que
        son una serie de requerimientos y que un tipo puede implementar
        múltiples \traits, pero también permiten implementaciones por defecto:

\begin{minted}{rust}
trait Sender {
    // Los métodos requieren especificar `self` explícitamente,
    // que es lo mismo que `this` en Java o C++. En este caso,
    // `&send` tomará una referencia al tipo que implemente
    // `Sender`. También podría ser una referencia mutable con
    // `&mut self`, o el mismo tipo con `self`.
    fn send(&self, msg: String);

    // Implementación por defecto.
    fn send_twice(&self, msg: String) {
        self.send(msg);
        self.send(msg.clone());
    }
}
\end{minted}

        Y para implementar un trait para un tipo:

\begin{minted}{rust}
impl Sender for MessageC {
    fn send(&self, msg: String) {
        match self {
            Join => println!("Joined"),
            Text(txt, id) => println!("{} sent: {}", id, txt),
            // Las variables `_` son ignoradas
            Leave(_) => println!("Left"),
        }
    }

    // `send_twice` se implementará automáticamente.
}
\end{minted}

    Notar que, sin embargo, Rust no es un lenguaje orientado a objetos. Un
    \trait puede heredar de otro \trait, pero un \struct no puede heredar de
    otro \struct.

\end{itemize}

\section{Genéricos y librería estándar}

De forma similar a C++, Rust posee tipos genéricos. Esto permite la
implementación de una librería estándar flexible, con varias estructuras de
datos importantes a conocer:

% NOTE: esto creo que lo puedo evitar si no incluyo cómo se usa async_ffi, que
% me parece demasiado complejo para el documento.
% \begin{minted}{rust}
% // Función genérica, donde `ToString` es un trait
% fn print<T: ToString>(t: T) {}

% // Otra manera de especificar genéricos con diferencias
% // menores que no se explicarán en esta introducción.
% fn print(t: impl ToString) {}
% \end{minted}

\begin{itemize}
    \item Tipos primitivos:
        \begin{itemize}
            \item Carácteres con \rust{char}.

            \item Punto flotante con \rust{f32} y \rust{f64}.

            \item Booleanos con \rust{bool}.

            \item Enteros: \rust{u8}, \rust{i8}, \rust{u16}, \rust{i16},
                \rust{u32}, \rust{i32}, \rust{u64}, \rust{i64}, e incluso
                \rust{i128} y \rust{u128} en las arquitecturas que lo soportan.

            \item Vectores de tamaño fijo: por ejemplo \rust{[1, 2, 3, 4, 5]}.

            \item N-tuplas como \rust{(1, true, 9.2)}.

            \item El tipo ``unidad'', \rust{()}, equivalente a \code{void} en C
                o C++.

            \item Punteros básicos con \rust{*const T} o \rust{*mut T}.

        \end{itemize}

    \item \rust{Vec<T>} representa un vector contiguo y redimensionable.

    \item \rust{HashMap<K, V>} es una tabla hash, genérica respecto a su clave
        \rust{K} y su valor \rust{V}. No se encuentra en el preludio, por lo que
        requeriría la siguiente declaración, similar a un \code{import} de Java:

\begin{minted}{rust}
use std::collections::HashMap;
\end{minted}

    \item \rust{Box<T>}, usado para localizar un tipo \rust{T} no nulo en
        memoria. Además de un \rust{*const T}, incluye el tamaño que ocupa
        \rust{T} y tiene una interfaz limitada para que su uso sea siempre
        seguro.

    \item \rust{str} es una cadena UTF-8 de solo lectura, típicamente usada con
        una referencia \rust{&str}. Va acompañada por su longitud, por lo que no
        hace falta terminarla con \code{\0}, a diferencia de C. \rust{String} es
        su versión modificable asignada en memoria.

\end{itemize}

\section{Gestión de errores}

En Rust, los errores se indican con el tipo \rust{Result<T, E>}. Este se trata
de una enumeración cuyo valor puede ser \rust{Ok(T)}, con el resultado obtenido
satisfactoriamente, o \rust{Err(E)}, con el tipo de error que ha sucedido. Dado
que el resultado está contenido dentro suyo, es imposible olvidar comprobar si
se ha producido algún error. Se puede usar \rust{match} para comprobar el
resultado, o una serie de funciones disponibles para hacer el proceso más
ergonómico:

\begin{minted}{rust}
match load_file(input) {
    Ok(data) => /* ... */,
    Err(e) => eprintln!("Error: {e}"),
}
\end{minted}

En caso de que se produjera un error del que el programa no se puediera
recuperar, como quedarse sin memoria o un fallo inesperado en la implementación,
se usa la funcionalidad de \emph{pánicos}. Un pánico se propaga de forma similar
a una excepción de C++ o Java, y terminará la ejecución por completo. Se puede
invocar con el macro \rust{panic!} o utilidades similares.

\section{Macros}

Rust cuenta con dos tipos de macros: \emph{declarativos} y \emph{procedurales}.
Ambos permiten generar código a tiempo de compilación, pero se diferencian
principalmente en la flexibilidad que ofrecen, a coste de un coste de desarrollo
menor o mayor, respectivamente.

Los macros declarativos se crean con una sintaxis especializada, similar a un
\rust{match} con patrones de tokens (identificadores, tipos, etc) como entrada,
y los tokens nuevos como salida. Son similares a los macros de C o C++, pero más
potentes e higiénicos (i.e., su expansión no captura identificadores
accidentalmente).

Los macros procedurales se describen como extensiones del lenguaje.
Esencialmente, ejecutan código en la compilación que consume y produce sintaxis
de Rust; consisten en directamente transformar el Árbol de Sintaxis Abstracta
(AST)~\cite[Procedural Macros]{rustref}. Consecuentemente, su complejidad es
mucho mayor, pero expanden las posibilidades de los macros enormemente.

\begin{minted}{rust}
some_macro!(1, 2, 3); // Puede ser tanto declarativo como procedural
\end{minted}

\begin{minted}{rust}
// Sintaxis típica de invocación de un macro
some_macro! {
    fn some_function() { /* ... */ }
}

// También permitido en el caso de los procedurales
#[some_macro]
fn some_function() { /* ... */ }
\end{minted}

Finalmente, los macros procedurales se pueden declarar de forma que
\emph{deriven} (implementen automáticamente) un \trait. Esto evita escribir
código repetitivo de forma muy sencilla:

\begin{minted}{rust}
// Con un macro `derive` para el trait `Debug`, que sirve para
// mostrar variables por pantalla.
#[derive(Debug)]
struct X(i32);

// Sin ellos sería lo siguiente. Como es trivial se puede
// simplificar en un macro procedural de tipo `derive`.
impl fmt::Debug for X {
    fn fmt(&self, f: &mut fmt::Formatter) -> fmt::Result {
        write!(f, "{:?}", self.0)
    }
}
\end{minted}

\section{Lifetimes}

La seguridad que provee Rust en memoria se basa en un modelo a tiempo de
compilación con \lifetimes. Una \lifetime es 

TODO: esto depende de cómo se acaba incluyendo la sección de ``problemas con
varianza y subtipado''.

\section{Unsafe}

Para poder mantener control completo a bajo nivel, es posible ignorar sus
garantías de seguridad con el sub-lenguaje llamado \emph{unsafe Rust}. El
análisis estático de Rust es conservativo; en algunas ocasiones es posible que
rechace algunos programas correctos. El desarrollador puede indicar que es
consciente de la situación y puede apagar este análisis para corregirlo por sí
mismo, arriesgándose a cometer un error en su código.

Se puede acceder a \emph{unsafe Rust} conteniendo el código dentro de un bloque
\rust{unsafe { /* ... */ }} o una función \rust{unsafe fn name() { /* ... */ }}.
Funciona igual que rust, pero incluye varias nuevas habilidades, entre otras:

\begin{itemize}
    \item Leer un puntero bruto en memoria

    \item Acceder o modificar una variable estática mutable

    \item Llamar a una función \unsafe

\end{itemize}

\section{Programación asíncrona}

Como muchos lenguajes modernos, Rust da soporte a la programación asíncrona, un
modelo de programación concurrente. Sin entrar en excesivo detalle, esta permite
tener una gran cantidad de \emph{tareas} concurrentes ejecutándose sobre unos
pocos hilos del Sistema Operativo. Su caso de uso principal es programas cuyo
rendimiento está limitado por operaciones de entrada y salida, como servidores o
bases de datos~\cite{rustasyncbook}.

\begin{minted}{rust}
// Con `async` se indica que la función es asíncrona.
async fn get_two_sites_async() {
    // Creación de dos "futuros" que, al completarse, descargarán
    // asíncronamente las páginas web. Similar a la creación de
    // un nuevo hilo.
    let future_one = download_async("https://www.foo.com");
    let future_two = download_async("https://www.bar.com");

    // Ejecutar las dos tareas. Similar a esperar la terminación de
    // los hilos.
    join!(future_one, future_two);

    // Con `.await` se puede esperar a la terminación de un futuro
    // individual.
    let future_three = download_async("https://www.bar.com").await;
}
\end{minted}

% vim: spelllang=es

\chapter{Funcionamiento interno de Tremor}\label{annex:tremor}

\section{Arquitectura}

Antes de comenzar a modificar el código existente en Tremor, era importante
conocer cómo funciona para evitar perder el tiempo. Tremor se basa en el modelo
actor. Citando Wikipedia:

% ORIGINAL:
% ``[The actor model treats the] actor as the universal primitive of concurrent
% computation. In response to a message it receives, an actor can: make local
% decisions, create more actors, send more messages, and determine how to
% respond to the next message received. Actors may modify their own private
% state, but can only affect each other indirectly through messaging (removing
% the need for lock-based synchronization).''

``[El modelo actor trata al] actor como el componente universal de computación
concurrente. En respuesta a un mensaje que recibe, un actor puede: tomar
decisiones locales, crear más actores, enviar más mensajes y determinar cómo
responder al siguiente mensaje recibido. Los actores pueden modificar su propio
estado privado, pero solo pueden afectarse entre sí indirectamente a través de
mensajería (eliminando la necesidad de sincronización con
\emph{locks}).''~\cite{wikiactor}

No usa un lenguaje (como \namecite{erlang}) o framework (como
\namecite{bastion}, quizá en el futuro) que siga estrictamente este modelo, pero
re-implementa los mismos patrones de forma manual. Tremor se basa en
\emph{programación asíncrona}, es decir, que en vez de hilos trabaja con
\emph{tareas}, un concepto de nivel más alto y especializado para operaciones de
entrada/salida. De la documentación de \namecite{async_std}, la runtime
asíncrona que usa Tremor:

% ORIGINAL:
% ``An executing asynchronous Rust program consists of a collection of native OS
% threads, on top of which multiple stackless coroutines are multiplexed. We
% refer to these as “tasks”. Tasks can be named, and provide some built-in
% support for synchronization.''

``La ejecución de un programa asíncrono en Rust consiste en una recopilación de
hilos nativos del Sistema Operativo, sobre los cuales múltiples corutinas no
apilables (\emph{stackless}) son multiplexadas. Nos referimos a ellas como
``tareas''. Las tareas pueden tener nombre e incluir soporte para
sincronización.''~\cite{asyncstd_task}

Podría resumirse su arquitectura con la frase: ``Tremor se basa en actores
corriendo en tareas diferentes, que se comunican asíncronamente con canales''.

\section{Detalles de implementación}

A nivel de implementación, los conectores se definen con el \trait
\rust{Connector}, incluido en la figura \ref{fig:tremor_connector_trait}.
Esencialmente, los plugins de tipo conector exportarán públicamente esta
interfaz en su binario y la runtime deberá ser capaz de cargarlo dinámicamente.
Inicialmente, todos los conectores disponibles se listaban y cargaban de forma
estática al inicio del programa.

El actor principal se llama \rust{World}. Contiene el estado del programa, como
los artefactos disponibles (\emph{repositorios}) y los que se están ejecutando
(\emph{registros}) y se usa para inicializar y controlar el programa.

Los \emph{managers} o \emph{gestores} son simplemente actores en el sistema que
envuelven una funcionalidad. Ayudan a desacoplar la comunicación de los detalles
de implementación. De esta forma, se puede eliminar código repetitivo en la
inicialización, como la creación de canales de comunicación o el lanzamiento del
componente en una tarea nueva. Generalmente, hay un gestor por cada tipo de
artefacto para facilitar su creación y también uno por cada instancia que se
esté ejecutando para controlar su comunicación.

Notar que la inicialización de los conectores ocurre en dos pasos. Primero se
\emph{registran}, es decir, se indica su disponibilidad para cargarlo
añadiéndolo al repositorio. Posteriormente, no se comenzará a ejecutar hasta
conectarse con otro artefacto con \rust{launch_binding}, lo cual lo movería del
repositorio al registro, junto al resto de artefactos ejecutándose.

\begin{figure}[h]
    \centering
    \begin{minted}{rust}
pub trait Connector {
    /// Crea la parte "source" del conector, si es aplicable.
    async fn create_source(
        &mut self,
        _source_context: SourceContext,
        _builder: source::SourceManagerBuilder,
    ) -> Result<Option<source::SourceAddr>> {
        Ok(None)
    }

    /// Crea la parte "sink" del conector, si es aplicable.
    async fn create_sink(
        &mut self,
        _sink_context: SinkContext,
        _builder: sink::SinkManagerBuilder,
    ) -> Result<Option<sink::SinkAddr>> {
        Ok(None)
    }

    /// Intenta conectarse con el mundo exterior. Por ejemplo, inicia la
    /// conexión con una base de datos.
    async fn connect(
        &mut self,
        _c: &ConnectorContext,
        _attempt: &Attempt
    ) -> Result<bool> {
        Ok(true)
    }

    /// Llamado una vez cuando el conector inicia.
    async fn on_start(&mut self, _c: &ConnectorContext) -> Result<()> {
        Ok(())
    }
    /// Llamado cuando el conector pausa.
    async fn on_pause(&mut self, _c: &ConnectorContext) -> Result<()> {
        Ok(())
    }
    /// Llamado cuando el conector continúa.
    async fn on_resume(&mut self, _c: &ConnectorContext) -> Result<()> {
        Ok(())
    }
    /// Llamado ante un evento de "drain", que se asegura de que no
    /// lleguen más eventos a este conector.
    async fn on_drain(&mut self, _c: &ConnectorContext) -> Result<()> {
        Ok(())
    }
    /// Llamado cuando el conector termina.
    async fn on_stop(&mut self, _c: &ConnectorContext) -> Result<()> {
        Ok(())
    }
}
    \end{minted}
    \caption{Simplificación del \trait \rust{Connector}}%
    \label{fig:tremor_connector_trait}
\end{figure}

\subsection{Registro}

La Figura~\ref{fig:tremor_registering} detalla todos los pasos seguidos en el
código. Primero se inicializan los gestores y a continuación se registran los
artefactos. Esta parte se realizaba de forma estática con
\rust{register_builtin_types}, pero después de implementar el PDK, debería
ocurrir dinámicamente. Tremor buscaría automáticamente plugins en sus
directorios configurados e intentaría registrar todos los que encuentre. En una
futura versión, el usuario podría solicitar manualmente el cargado de un plugin
nuevo mientras se está ejecutando Tremor.

\begin{figure}
    \centering
    \includegraphics[width=\textwidth]{./Imagenes/registering.pdf}
    \caption{Registro de un conector en el programa}%
    \label{fig:tremor_registering}
\end{figure}

\subsection{Inicialización}

Ya que es un proceso en múltiples pasos (en la implementación es más complicado
que registro y creación), la primera parte provee las herramientas para
inicializar el conector (un \builder). Cuando el conector necesite comenzar a
ejecutarse porque se haya añadido a una \pipeline, el \builder ayuda a construir
y configurarlo de forma genérica. Finalmente, se añade a una tarea asíncrona
nueva para que se pueda comunicar con otras partes de Tremor. El gestor
\rust{connectors::Manager} contiene todos los conectores ejecutándose en Tremor,
como se muestra en la Figura~\ref{fig:tremor_initializing}.

\begin{figure}
    \centering
    \includegraphics[width=\textwidth]{./Imagenes/initializing.pdf}
    \caption{Inicialización de un conector en el programa}%
    \label{fig:tremor_initializing}
\end{figure}

La Figura~\ref{fig:tremor_pipeline} muestra un ejemplo de una \pipeline,
definida con Troy, su propio lenguaje inspirado en SQL.

\begin{figure}
    \centering
    \begin{minted}[escapeinside=||]{mysql}
|\textcolor{blue}{define pipeline}| main
# El puerto `exit` no existe por defecto, así que tenemos que
# sobreescribir la selección de puertos incorporada.
into |out, exit|
|\textcolor{blue}{pipeline}|
  # Uso del módulo `std::string`.
  use std::|string|;
  use lib::scripts;

  # Creación de nuestro script.
  create |\textcolor{blue}{script}| punctuate from scripts::punctuate;

  # Filtrado de cualquier evento que sea un `"exit"` y enviarlo al
  # puerto de salida.
  select {"graceful": false} from |in| where event == "exit" into |exit|;

  # Conexión de nuestro texto convertido a mayúsculas al script.
  select |string|::capitalize(event) from |in| where event != "exit"
    into punctuate;
  # Conexión de nuestro script a la salida.
  select event from punctuate into |out|;
end;
    \end{minted}
    \caption{Ejemplo de una \pipeline definida para Tremor.}%
    \label{fig:tremor_pipeline}
\end{figure}

\subsection{Configuración}

Una vez haya un conector corriendo, la Figura~\ref{fig:tremor_setting_up}
visualiza cómo se divide en una parte \sink y otra \source. Estas son
opcionales, pero no exclusivas; se puede tener una de las dos o ambas. De forma
similar, un \builder se usa para inicializar las partes y a continuación lanza
una nueva tarea asíncrona para ellos.

\begin{figure}
    \centering
    \includegraphics[width=\textwidth]{./Imagenes/setting-up.pdf}
    \caption{Configuración de un conector en el programa}%
    \label{fig:tremor_setting_up}
\end{figure}

También se crea un gestor por cada instancia de \sink o \source, que se
encargará de la comunicación con otros actores. De esta forma, sus interfaces
pueden mantenerse lo más simples posibles. Esos gestores recibirán peticiones de
conexión de la \pipeline y posteriormente leerán o enviarán eventos en ella.

La diferencia principal entre \sources y \sinks a nivel de implementación es que
este último también puede responder a mensajes usando la misma conexión. Esto es
útil para notificar que el paquete ha llegado (\rust{Ack}) o que algo ha fallado
(\rust{Fail} para un evento específico, \rust{CircuitBreaker} para dejar de
recibir datos por completo).

Los códecs y preprocesadores se involucran aquí para tanto los \sources como los
\sinks. En la parte de \source, los datos son transformados a través de una
serie de preprocesadores y posteriormente se aplica un códec. Para los \sinks,
se sigue el proceso inverso: los datos se codifican primero a bytes con el códec
y posteriormente una serie de postprocesadores se aplican sobre los datos
binarios.

\subsection{Notas adicionales}

Algunos conectores se basan en \emph{flujos}. Son equivalentes a los flujos de
TCP, que agrupan paquetes para evitar entremezclarlos. Su inicio y fin es
marcado mediante mensajes asíncronos. Algunos preprocesadores puedan querer
guardar datos internos, así que el gestor se guarda el estado del flujo en un
campo llamado \rust{states}. Si un conector no necesita flujos, como
\rust{metronome} (que únicamente envía eventos periódicamente), puede
especificar su identificador de flujo como \rust{DEFAULT_STREAM_ID} siempre.

Tras terminar la interfaz de los conectores para el sistema de plugins, las
primeras implementaciones a desarrollar deberían ser:

\begin{itemize}
    \item \emph{Blackhole}, usado para analizar el rendimiento. Realiza
        mediciones de tiempos de final a final para cada evento pasando por la
        \pipeline, y al final guarda un histograma HDR (\emph{High Dynamic
        Range}).

    \item \emph{Blaster}, usado para repetir una serie de eventos de un archivo,
        que es especialmente útil para pruebas de rendimiento.

\end{itemize}

Ambos son relativamente simples y serán de gran ayuda para entender la
inevitable degradación de eficiencia causada por el sistema de plugins. De todos
modos, el equipo de Tremor insistía que lo más importante primero es que
funcione, y después me podría preocupar sobre eficiencia.

\chapter{Conversión del ABI de Rust al ABI de C}\label{annex:abi}

Para ilustrar mejor el problema de conversión de ABIs, se introduce la
estructura más complicada al respecto, \rust{Value}. Esta estructura sirve para
representar datos pseudo-JSON y es definido a continuación de forma
simplificada:

\begin{minted}{rust}
pub enum Value {
    /// Valores estáticos (enteros, booleanos, etc)
    Static(StaticNode),
    /// Tipo para cadenas de caracteres
    String(String),
    /// Tipo para listas
    Array(Vec<Value>),
    /// Tipo para objetos (mapas clave-valor)
    Object(Box<HashMap<String, Value>>),
    /// Tipo para datos binarios
    Bytes(Vec<u8>),
}
\end{minted}

Para poder usar \rust{Value} en la interfaz del sistema de plugins, se puede
convertir a:

\begin{minted}{rust}
#[repr(C)] // La representación en memoria de Value seguirá el ABI de C
#[derive(StableAbi)] // Solo necesario cuando se usa abi_stable
pub enum Value {
    Static(StaticNode),
    /// Ahora usa `RString`, la altenativa a `String` de abi_stable
    String(RString),
    /// De forma similar, usa `RVec` en vez de `Vec`
    Array(RVec<Value>),
    /// Cambio de `Box`, `HashMap` y `String` por sus alternativas
    Object(RBox<RHashMap<RString, Value>>),
    /// Otro cambio de `Vec`
    Bytes(RVec<u8>),
}
\end{minted}

El primer problema surge en la variante \rust{Static}. Su tipo contenido
internamente, \rust{StaticNode}, es externo y usa \rust{#[repr(Rust)]}. Se
declara en el \crate \cratelink{value_trait}, que lo declara tal que:

\begin{minted}{rust}
pub enum StaticNode {
    I64(i64),
    U64(u64),
    F64(f64),
    Bool(bool),
    Null,
}
\end{minted}

Esto se podría arreglar siguiendo el mismo procedimiento recursivamente, hasta
que todo sea \rust{#[repr(C)]}. Pero como se trata de una librería externa,
tendrá que abrirse un nuevo pull request y esperar que al autor le parezcan bien
los cambios~\cite{openstaticnode}. Será importante también que la estructura use
\rust{#[repr(C)]} únicamente cuando opcionalmente se configure a tiempo de
compilación como necesario. De esta forma, el resto de usuarios podrán seguir
aprovechándose de las ventajas de rendimiento que ofrece \rust{#[repr(Rust)]}.

\section{Consecuencias del sistema de plugins}

Por desgracia, este cambio no termina ahí; cambiar las variantes de \rust{Value}
implica que el código que lo usaba se romperá de numerosas formas:

\begin{minted}{rust}
// No funcionará porque Value::Array contiene un RVec ahora
let value = Value::Array(Vec::new());
\end{minted}

Este caso es el más sencillo: simplemente hace falta cambiar \rust{Vec} por
\rust{RVec}. La intención de los tipos de \abistable es que sean un reemplazo
directo de los de la librería estándar, i.e., su interfaz será la misma:

\begin{minted}{rust}
let value = Value::Array(RVec::new());
\end{minted}

TODO: 'return' es devolver o retornar?

Es un poco más complicado cuando los tipos anteriores se exponían en métodos,
porque requiere tomar una decisión entre expandir el límite de FFI del
\emph{funcionamiento interno} de \rust{Value} a los \emph{usuarios} de
\rust{Value}. Por ejemplo, la variante \rust{Value::Object} contiene un
\rust{RHashMap} ahora, pero el método \rust{Value::as_object} solía devolver una
referencia a \rust{HashMap}. Se producirá un error nuevo ahí y tendrá que
tomarse la decisión de devolver \rust{RHashMap} o añadir una conversión interna
a \rust{HashMap}:

\begin{minted}{rust}
impl Value {
    // Código original
    fn as_object(&self) -> Option<&HashMap<String, Value>> {
        match self {
            // Problema: `m` ahora es una `RHashMap`, pero la función
            // devuelve un `HashMap`.
            //
            // Solución 1: cambiar el tipo devuelto a `RHashMap`
            // Solución 2: convertir `m` a un `HashMap` con `m.into()`
            Self::Object(m) => Some(m),
            _ => None,
        }
    }
}
\end{minted}

\begin{itemize}
    \item Si se cambia el tipo devuelto a \rust{RHashMap}, casi todas las veces
        que se llamaba a \rust{as_object} ahora dejarán de compilar porque se
        esperan un \rust{HashMap}.

        Esto puede ser complicado porque, para evitar realizar conversiones, el
        sistema de plugins \emph{infectaría} la base de código por completo.
        Tendría que propagarse el uso de \rust{RHashMap} por todo el programa,
        incluso cuando el PDK no es importante. Por ejemplo, \rust{Value}
        también se usaba en la implementación del lenguaje de Tremor, Troy.
        Tener que usar un \rust{RHashMap} en esa situación sería confuso y
        acabarían modificándose gran cantidad de ficheros sin relación al
        sistema de plugins.

    \item Si se realiza una conversión interna a \rust{HashMap} en
        \rust{as_object}, evitaremos todos esos errores, con un pequeño coste de
        rendimiento. Es la opción más fácil, pero si \rust{Value::as_object} se
        usara frecuentemente, e.g., en el bucle principal, sí que podría causar
        una degradación considerable.
\end{itemize}

Como indica la sección~\ref{abiperf}, las conversiones entre la librería
estándar y \abistable son $O(1)$. Esto es dónde la metodología \work es
relevante: simplemente dejaremos el límite del FFI en su mínimo y añadiremos
conversiones cuanto antes sea posible. Al terminar, si se detectan problemas de
rendimiento en un caso en concreto, se puede reconsiderar.

\section{Problemas con tipos externos}\label{sec:abi_ext}

En algunos casos, los tipos de \abistable no habían sido actualizados para
incluir métodos nuevos de la librería estándar, por lo que era necesario un pull
request para añadirlo\footnote{El anexo \ref{annex:contributions} lista todas
las contribuciones de código abierto realizadas para este proyecto}. Pero por lo
general, convertir los tipos \emph{de la librería a estándar a \abistable} es
una tarea trivial, simplemente un tanto tedioso.

Los problemas surgen cuando es necesario convertir \emph{tipos externos a
\abistable}. La declaración anterior de \rust{Value} era una simplificación;
realmente, Tremor usa la implementación de \cratelink{halfbrown} de
\rust{HashMap<K, V>}. Esto se debe a que es más eficiente para su caso de uso, y
que posee algunas funcionalidades adicionales necesarias. El mismo caso se da
para otro tipo \rust{Cow}, cuya alternativa en la \crate \cratelink{beef} ocupa
menos espacio en memoria y ofrece un mejor rendimiento en Tremor.

Ninguno de estos dos tipos tienen soporte dentro de \abistable, y aunque estos
tipos estén basados en otros de la librería estándar, la conversión no es
directa. Se pueden tomar cuatro posibles alternativas:

\subsection{Evitar el tipo externo}

Basándose en \work, una solución perfectamente válida es eliminar las
optimizaciones temporalmente y dejar un \code{TODO} para que se pueda revisar
posteriormente. Es posible que el sistema de plugins tenga excesiva complejidad,
y limitarse a usar tipos de la librería estándar podría ser suficiente.

En el caso específico de \rust{Value}, eliminar las optimizaciones problemáticas
parece la manera más fácil de arreglar el problema. Y lo sería, si no fuera
porque eliminar código también puede ser complicado, como muestra la
Figura~\ref{fig:errors}, especialmente cuando la funcionalidad extra del tipo
externo no está disponible.

\begin{figure}
    \centering
    \includegraphics[width=\textwidth]{./Imagenes/errors.png}
    \caption{Al intentar evitar los tipos externos se produjeron más de 120
    errores de compilación.}%
    \label{fig:errors}
\end{figure}

\subsection{Encapsular el tipo externo}

Otra opción es crear un \emph{wrapper} para \code{halfbrown}, de la misma forma
que lo hace ya \abistable con otras librerías más conocidas. Este
encapsulamiento hace posible su uso desde el ABI de C de forma segura. Sin
embargo, estos ejemplos ya existentes son complejos~\cite{complexwrapper} y
difíciles de mantener, ya que tendrán que actualizarme con cada nueva versión de
\code{halfbrown}.


\begin{figure}
    \centering
    \begin{minted}{rust}
// Así funciona la programación asíncrona en Rust; la primera
// función es prácticamente equivalente a la segunda.
async fn example() -> String {
    read_file().await
}
fn example() -> impl Future<Output = String> {
    async {
        read_file().await
    }
}

// No pueden haber genéricos en FFI, por lo que ahora `Future`
// es un tipo concreto `FfiFuture` en vez de un trait. La
// conversión de `Future` a `FfiFuture` se puede realizar con
// `into_ffi`.
fn example() -> FfiFuture<String> {
    async move {
        read_file().await
    }
    .into_ffi()
}
// `FfiFuture<T>` implementa `Future<Output = T>`, por lo que
// su uso es el mismo.
async fn user() {
    example().await
}
    \end{minted}
    \caption{Interfaz modificada para la programación asíncrona en el sistema de
    plugins con la \crate \code{async_ffi}.}%
    \label{fig:async_ffi}
\end{figure}

\subsection{Reimplementar el tipo con el ABI de C desde cero}

Similar a la solución anterior, pero con incluso más costoso, dado que también
requeriría reimplementar la funcionalidad. Puede parecer indeseable, pero es la
mejor forma de asegurar un rendimiento máximo. Los tipos externos mencionados
son parte de optimizaciones; encapsularlos podría tener un impacto en su
rendimiento y hacerlos inútiles.

Si esta parte del proyecto es lo suficientemente importante y existen los
recursos, debería considerarse. De hecho, el mismo tipo \rust{Value} en Tremor
surgió por esta razón: ya existía \rust{simd_json::Value} de otra librería, pero
carecía de la suficiente flexibilidad y el equipo implementó uno personalizado.

\subsection{Simplificar el tipo para la interfaz}

Esta última opción resultó ser la más sencilla de implementar: crear una copia
de \rust{Value} cuyo único uso es para comunicarse entre runtime y plugins,
ilustrado en la Figura~\ref{fig:simplify}.

\begin{figure}
    \centering
    \includegraphics[width=10cm]{./Imagenes/simplify.pdf}
    \caption{Comunicación entre runtime y plugins en el PDK.}%
    \label{fig:simplify}
\end{figure}

Ya que es un tipo nuevo, no se romperá nada del código ya existente, y
únicamente hará falta cambiarlo donde se use la interfaz. Su implementación es
sencilla (notar el cambio de nombre a \rust{PdkValue}):

\begin{minted}{rust}
pub enum PdkValue {
    /// Valores estáticos (enteros, booleanos, etc)
    Static(StaticNode),
    /// Tipo para cadenas de caracteres
    String(String),
    /// Tipo para listas
    Array(Vec<Value>),
    /// Tipo para objetos (mapas clave-valor)
    Object(Box<HashMap<String, PdkValue>>),
    /// Tipo para datos binarios
    Bytes(Vec<u8>),
}
\end{minted}

No es necesario escribir métodos adicionales para el nuevo \rust{PdkValue}, solo
sus conversiones desde y hasta el tipo original, \rust{Value}. Esto sería
equivalente a, en vez de pasar un \rust{Vec<T>} al PDK, reemplzarlo con un
\rust{*const u8} para los datos y un \rust{u32} para la longitud. Simplemente
consiste en simplificar los tipos en la interfaz, y convertirlos de vuelta para
usar la funcionalidad completa.

El problema principal es que la conversión entre tipos es ahora $O(n)$ en vez de
$O(1)$, dado que es necesario iterar los datos en los objetos y vectores para la
conversión. Su uso sería el siguiente:

\begin{minted}{rust}
// Esta función es exportada por el plugin. Funcionará porque
// `PdkValue` está declarado con el ABI de C.
pub extern "C" fn plugin_funfuncue: PdkValue) {
    let value = Value::from(value);
    value.do_func()
}

// Esto se puede implementar en la runtime para facilitar su uso,
// convirtiendo al tipo original.
fn runtime_wrapper(value: Value) {
    plugin_func(value.into());
}
\end{minted}

Es la alternativa más sencilla, pero implica un coste de rendimiento; dos
conversiones implican iterar los datos dos veces. Tras mediciones posteriores,
se verificó que convertir los datos era un 5-10\% de la ejecución del programa.
Es menos de lo esperado, pero sigue sin ser suficiente para Tremor.

También tiene un coste de usabilidad; en comparación con tener un único
\rust{Value}, es necesario convertir los tipos y posiblemente crear encapsularlo
con una función de más alto nivel (\rust{runtime_wrapper}). Es una tarea
relativamente trivial, por lo que se podría automatizar con macros procedurales
en Rust, pero esto debería dejarse para el final del proyecto.

En conclusión, esta alternativa es la más fácil de implementar en el corto plazo
y por tanto la que mejor sigue \work. Se puede visualizar la diferencia de
rendimiento entre usar \code{PdkValue} y reimplementar el tipo con C ``desde
cero'' --- hecho para la segunda versión --- en el Anexo~\ref{annex:benchmarks}.

\chapter{Problemas con varianza y subtipado}\label{annex:covariance}

Otro problema inesperado tuvo que ver con la \emph{varianza} y \emph{subtipado}.
Son dos conceptos de teoría de sistemas de tipos, especialmente conocidos por
desarrolladores de lenguajes orientados a objetos como Java o C\#. En el caso de
Rust solo se da en las \lifetimes, así que no es tan popular. Lo que lo hace más
complicado de tratar es que es completamente implícito: mejora la usabilidad del
lenguaje cuando \emph{funciona}; en caso contrario, resulta en errores
intricados y difíciles de identificar.

Este tema no se cubre en \textcite{rustbook}, sino en \textcite[Subtyping and
Variance]{nomicon} y \textcite[Subtyping and Variance]{rustref}. También es
recomendable consultar el artículo \textcite{lcnr_covandcontra} o a
\textcite{video_covandcontra} para un formato en vídeo.

Deriva también de los problemas encontrados con el tipo \rust{Value}. La
historia completa se incluye en el issue \textcite{abi_covandcontra}. Al cambiar
los tipos de la librería estándar a los de \abistable, se producían errores de
\lifetimes \emph{inexplicables} (ver Figura~\ref{fig:errors}). Estuve bloqueado
con dicho problema durante mucho tiempo, así que tras comentárselo a mis
mentores, Heinz me ayudó a reproducir el problema de forma mínima. Por alguna
razón que todavía desconocíamos, dos tipos supuestamente equivalentes diferían a
la hora de compilar:

\begin{minted}{rust}
use abi_stable::std_types::RCow;
use std::borrow::Cow;

fn cmp_cow<'a, 'b>(left: &Cow<'a, ()>, right: &Cow<'b, ()>) -> bool {
    left == right
}

// Este caso falla en compilación, pero es aparentemente igual
fn cmp_rcow<'a, 'b>(left: &RCow<'a, ()>, right: &RCow<'b, ()>) -> bool {
    left == right
}
\end{minted}

\begin{minted}{text}
> cargo build
error[E0623]: lifetime mismatch
  --> src/lib.rs:10:10
   |
9  | fn cmp_rcow<'a, 'b>(
   |        left: &RCow<'a, ()>, right: &RCow<'b, ()>) -> bool {
   |              ------------          ------------
   |              |
   |              these two types are declared with
   |              different lifetimes...
10 |     left == right
   |          ^^ ...but data from `left` flows into `right` here

For more information about this error, try `rustc --explain E0623`.
error: could not compile `repro` due to previous error
\end{minted}

Este tipo de error suele darse en caso de que la \lifetime de un valor no viva
lo suficiente. Por ejemplo, el ejemplo de \code{rustc --explain E0623} es el
siguiente. Se tienen dos \lifetimes \emph{sin relación entre sí}, \rust{'short}
y \rust{'long}. La estructura \rust{Foo} que se pasa como parámetro tiene la
\lifetime \rust{'short}, pero dentro de la función se le intenta asignar una
\lifetime \rust{'long}, que es imposible porque el compilador no sabe cuál tiene
un tiempo de vida mayor. Asignarle una \lifetime que viva más de lo que debe
significaría que se podría seguir usando \rust{Foo} después de que \rust{'short}
acabe, es decir, después de que \rust{Foo} haya sido destruido. Finalmente, esto
causaría inconsistencias en memoria porque nuestra variable de tipo \rust{Foo}
ya no existe, pero se está intentando acceder a ella.

\begin{minted}{rust}
struct Foo<'a> {
    x: &'a isize,
}

fn bar<'short, 'long>(c: Foo<'short>, l: &'long isize) {
    // Equivalente a asignarle otra lifetime a c
    let c: Foo<'long> = c; // error!
}
\end{minted}

Solucionarlo es tan simple como indicar que \rust{'short} tiene al menos el
mismo tiempo de vida que \rust{'long}. Es decir, que no se podría dar el caso de
que \rust{Foo} es usado después de destruirse:

\begin{minted}{rust}
// Notar que ahora `'short` se declara tal que `'short: 'long`
fn bar<'short: 'long, 'long>(c: Foo<'short>, l: &'long isize) {
    let c: Foo<'long> = c; // ok!
}
\end{minted}

Por tanto, uno pensaría que tiene que ver con el operador \rust{==}, que se
delega al \trait \rust{PartialEq}, así que dediqué tiempo intentando encontrar
la diferencia entre su implementación en \rust{Cow<'a, T>} y la de
\rust{RCow<'a, T>}. La mención anterior a estos errores como
\emph{inexplicables} se debe a que en este caso únicamente existe una \lifetime
\rust{'a}, así que no se podría arreglar indicando que \rust{'short: 'long}. No
obstante, existía alguna pequeña diferencia sin relación a este problema entre
las implementaciones, y al usar \emph{exactamente lo mismo} que en \rust{Cow<'a,
T>}, compilaba correctamente.

% TODO: credit properly
Tras cambiar \rust{left == right} por \rust{left.cmp(right)} en la reproducción
inicial, se repetía el problema, incluso con aparentemente la misma
implementación de \rust{Ord} (el \trait con el método \rust{cmp}). No fue hasta
que expliqué mi problema en un servidor de Discord con más desarrolladores de
Rust que descubrí que el verdadero problema era un término llamado la
\emph{varianza} de \rust{RCow<'a, T>}.

Todo acabó reduciéndose a la única diferencia en la implementación del \trait
\rust{Ord}. \rust{RCow<'a, T>} implementa un \trait llamado
\rust{BorrowOwned<'a>} y \rust{Cow<'a, T>} implementa otro llamado
\rust{ToOwned}. Ambos \traits son iguales, excepto que en \rust{BorrowOwned<'a>}
se incluye funcionalidad adicional para \abistable. El problema no tiene que ver
con esta diferencia en funcionalidad, sino que \rust{BorrowOwned<'a>} es
genérico respecto a la \lifetime \rust{'a}, lo cual no es el caso de
\rust{ToOwned}.

Al implementar \rust{Ord}, se tenía que indicar que \rust{T: ToOwned} o \rust{T:
BorrowOwned<'a>}. El problema era que al relacionar la \lifetime \rust{'a} de
esta forma, estaba rompiendo una regla que hacía a \rust{RCow}
\emph{invariante}, en vez de \emph{covariante}.

TODO: es esta sección demasiado técnica? Debería continuar? Me falta explicar
qué es invarianza y covarianza. La verdad que esto sí que me llevó mucho tiempo
en su momento, así que me gustaría incluirlo de alguna forma.

\chapter{Fases de desarrollo}\label{annex:hours}

Este anexo lista las fases en las que se llevó a cabo el proyecto, y las horas
invertidas en cada una. Se incluye también el periodo en el que se realizaron,
comenzando en abril de 2021 con la propuesta a Tremor. Una vez el proyecto fue
aceptado, se comenzó el proyecto oficialmente en agosto.

\setlist{topsep=2pt,nosep}

\begin{longtable}[H]{| m{9.7cm} | P{1.3cm} | P{3.4cm} |}
\caption{Fases de desarrollo del proyecto}\\

\hline
\textbf{Fase}
    & \textbf{Horas}
    & \textbf{Periodo} \\

\hline
Propuesta a Tremor: incluye una introducción sobre quién soy, qué proyecto
quiero hacer y una breve planificación de la metodología a seguir.

\vspace{5mm}
\emph{\url{https://nullderef.com/blog/gsoc-proposal/}}
    & 5
    & Abr.~\textquotesingle21 \\

\hline
Investigación inicial de las tecnologías disponibles para el sistema de
plugins y discusión con el equipo. Esto también formó parte de la propuesta, de
forma no oficial.

\vspace{4mm}
\emph{\url{https://nullderef.com/blog/plugin-tech/}}
    & 52
    & Abr.~\textquotesingle21 -- May.~\textquotesingle21 \\

\hline
Aceptación del proyecto. Introducción a Tremor y a su equipo.

\vspace{4mm}
\emph{\url{https://nullderef.com/blog/plugin-start/}}
    & 8
    & Ago.~\textquotesingle21 \\

\hline
Implementación de los prototipos iniciales.

\vspace{4mm}
\emph{\url{https://nullderef.com/blog/plugin-start/}}

\vspace{4mm}
\emph{\url{https://nullderef.com/blog/plugin-dynload/}}
    & 25
    & Ago.~\textquotesingle21 -- Oct.~\textquotesingle21 \\

\hline
Investigación en detalle de cargado dinámico.
    & 20
    & Sep.\textquotesingle21 -- Oct.~\textquotesingle21 \\

\hline
Investigación del funcionamiento interno de Tremor.
    & 15
    & Oct.~\textquotesingle21 \\

\hline
Aprendizaje de la librería \abistable.
    & 10
    & Nov.~\textquotesingle21 \\

\hline
Soporte para programación asíncrona con \abistable.
    & 10
    & Ene.~\textquotesingle22 \\

\hline
Primer diseño e implementación del sistema de plugins.
    & 60
    & Dic.~\textquotesingle22 -- Feb.~\textquotesingle22 \\

\hline
Mediciones de rendimiento.
    & 7
    & Feb.~\textquotesingle22 \\

\hline
Resolución de los problemas de covarianza y
subtipado\footnote{\url{https://github.com/rodrimati1992/abi_stable_crates/issues/75}}.
    & 20
    & Feb.~\textquotesingle22 -- Mar.~\textquotesingle22\\

\hline
Añadir soporte de la librería \code{halfbrown} para
\abistable\footnote{\url{https://github.com/rodrimati1992/abi_stable_crates/pull/83}}.
    & 15
    & Mar.~\textquotesingle22 -- Jun.~\textquotesingle22 \\

\hline
Segunda versión del sistema de plugins con las dos mejoras anteriores de
rendimiento.
    & 30
    & Mar.~\textquotesingle22 -- Jun.~\textquotesingle22 \\

\hline
Documentación final para que Tremor pueda continuar con el desarrollo del
proyecto.
    & 5
    & Jun.~\textquotesingle22 \\

\hline
Memoria.
    & 35
    & Jun.~\textquotesingle22 \\

\hline
\textbf{Total}
    & \textbf{277}
    & \textbf{Abr.~\textquotesingle21 --
Jun.~\textquotesingle22} \\

\hline
\end{longtable}


\end{document}
