% vim: spelllang=es

\chapter{Problemas con varianza y subtipado}\label{annex:covariance}

\section{Entendiendo el problema}

Otro problema inesperado tuvo que ver con la \emph{varianza} y \emph{subtipado}.
Son dos conceptos de teoría de sistemas de tipos, especialmente conocidos por
desarrolladores de lenguajes orientados a objetos como Java o C\#. En el caso de
Rust solo se da en las \lifetimes, así que no es tan popular. Lo que lo hace más
complicado de tratar es que es completamente implícito: mejora la usabilidad del
lenguaje cuando \emph{funciona}; en caso contrario, resulta en intricados
errores.

Este tema no se cubre en \namecite{rustbook}, sino en \namecite[Subtyping and
Variance]{nomicon} y \namecite[Subtyping and Variance]{rustref}. También es
recomendable consultar el artículo \namecite{lcnr_covandcontra} o a
\textcite{video_covandcontra} para un formato en vídeo.

Este anexo deriva de los problemas encontrados con el tipo \rust{Value} del
Anexo~\ref{annex:abi}. Al cambiar los tipos de la librería estándar a los de
\abistable, se producían errores de \lifetimes inexplicables (ver
Figura~\ref{fig:errors}). Estuve bloqueado con dicho problema durante mucho
tiempo, así que tras comentárselo a mis mentores, Heinz me ayudó a reproducir el
problema de forma mínima. Por alguna razón que todavía desconocíamos, dos tipos
supuestamente equivalentes diferían a la hora de compilar:

\begin{minted}{rust}
use abi_stable::std_types::RCow;
use std::borrow::Cow;

fn cmp_cow<'a, 'b>(left: &Cow<'a, ()>, right: &Cow<'b, ()>) -> bool {
    left == right
}

// Este caso falla en compilación, pero es aparentemente igual
fn cmp_rcow<'a, 'b>(left: &RCow<'a, ()>, right: &RCow<'b, ()>) -> bool {
    left == right
}
\end{minted}

\begin{minted}{text}
> cargo build
error[E0623]: lifetime mismatch
  --> src/lib.rs:10:10
   |
9  | fn cmp_rcow<'a, 'b>(
   |        left: &RCow<'a, ()>, right: &RCow<'b, ()>) -> bool {
   |              ------------          ------------
   |              |
   |              these two types are declared with
   |              different lifetimes...
10 |     left == right
   |          ^^ ...but data from `left` flows into `right` here

For more information about this error, try `rustc --explain E0623`.
error: could not compile `repro` due to previous error
\end{minted}

Este tipo de error suele darse en caso de que la \lifetime de un valor no viva
lo suficiente. En particular, el ejemplo de \code{rustc --explain E0623} es el
siguiente. Se tienen dos \lifetimes \emph{sin relación entre sí}, \rust{'short}
y \rust{'long}. La estructura \rust{Foo} que se pasa como parámetro tiene la
\lifetime \rust{'short}, pero dentro de la función se le intenta asignar una
\lifetime \rust{'long}. Esto es imposible porque el compilador no sabe cuál de
los dos tiene un tiempo de vida mayor. Asignarle una \lifetime que viva más de
lo que debe significaría que se podría seguir usando \rust{Foo} después de que
\rust{'short} acabe, es decir, después de que \rust{Foo} haya sido destruido.
Finalmente, esto causaría inconsistencias en memoria porque nuestra variable de
tipo \rust{Foo} ya no existe, pero se está intentando acceder a ella.

\begin{minted}{rust}
struct Foo<'a> {
    x: &'a isize,
}

fn bar<'short, 'long>(c: Foo<'short>, l: &'long isize) {
    // Equivalente a asignarle otra lifetime a c
    let c: Foo<'long> = c; // error!
}
\end{minted}

Solucionarlo es tan simple como indicar que \rust{'short} tiene al menos el
mismo tiempo de vida que \rust{'long}. Ahora el compilador tiene garantizado que
no se podría dar el caso de que \rust{Foo} es usado después de destruirse:

\begin{minted}{rust}
// Notar que ahora `'short` se declara tal que `'short: 'long`
// ('short contiene a 'long, o 'short tiene un tiempo de 
// vida mayor que 'long)
fn bar<'short: 'long, 'long>(c: Foo<'short>, l: &'long isize) {
    let c: Foo<'long> = c; // ok!
}
\end{minted}

Por tanto, uno pensaría que, en el caso de \rust{Value}, el error tiene que ver
con el operador \rust{==}. He descrito anteriormente estos errores como
\emph{inexplicables} porque, en un principio, una comparación binaria no
modifica la \lifetime de \rust{RCow<'a, T>}. Su \lifetime no debería importar
porque simplemente es una función que compara dos estructuras en un momento
dado --- no es necesario que una tenga un tiempo de vida mayor que otra.

De todos modos, \rust{==} se delega al
\trait \rust{PartialEq}, así que dediqué tiempo intentando encontrar la
diferencia entre su implementación en \rust{Cow<'a, T>} y la de \rust{RCow<'a,
T>}. Aparentemente, \rust{RCow<'a, T>} declaraba las \lifetimes en su
implementación de una forma distinta (aunque igualmente válida). Al usar
\emph{exactamente lo mismo} que en \rust{Cow<'a, T>}, compilaba correctamente.
Alegrado por haber arreglado el error, pero sin saber muy bien aún cómo, abrí un
nuevo pull request en \abistable y realicé alguna prueba
más~\cite{fail_covariance}.

Sin embargo, tras cambiar \rust{left == right} por \rust{left.cmp(right)} en la
reproducción inicial, se repetía el mismo problema. Incluso con aparentemente la
misma implementación de \rust{Ord} (el \trait con el método \rust{cmp}). Parecía
que, al haber arreglado \rust{PartialEq}, el problema había pasado a estar en
\rust{Ord}, pero esta vez no había manera de ``arreglarlo'' porque ambas
implementaciones eran iguales.

No fue hasta que compartí mi problema en un servidor de Discord con expertos en
el lenguaje~\cite{rustdiscord} que descubrí que el verdadero problema era un
concepto llamado la \emph{varianza} de \rust{RCow<'a, T>}.

\section{Breve introducción a varianza y subtipado}

La \emph{varianza} y \emph{subtipado} son un mecanismo que aumentan la
flexibilidad del modelo de memoria de Rust implícitamente. Si \rust{'a: 'b}
(``\rust{'a} contiene a \rust{'b}'' o ``\rust{'a} tiene un tiempo de vida mayor
que \rust{'b}''), entonces \rust{'a} es un \emph{subtipo} de \rust{'b}. Por
ejemplo, si una función espera un \rust{&'a u8}, también se le podrá pasar un
\rust{&'static u8}, porque sabemos que la referencia estática \rust{'static}
siempre va a vivir más que \rust{'a}. Asimismo, según una serie de reglas
relacionadas con el subtipado, el compilador determinará si un tipo es
\emph{covariante}, \emph{invariante} o \emph{contravariante} respecto a sus
parámetros (tipos genéricos y \lifetimes).

La referencia inmutable \rust{&'a T} es covariante respecto a tanto \rust{'a}
como \rust{T} y la mutable \rust{&'a mut T} es covariante en \rust{T}, pero es
invariante en \rust{'a}, entre otros. Lo importante es que la varianza de un
tipo compuesto como un \rust{struct} o \rust{enum} se ``hereda'', dependiendo de
los campos con los que se define. Si todos sus campos tienen el mismo tipo de
varianza respecto a un parámetro, entonces la estructura entera también tendrá
la misma respecto a ese parámetro. En caso de discrepancias, será invariante.
Por ejemplo, si un campo fuera covariante en \rust{'a}, pero otro fuera
contravariante o invariante en \rust{'a}, entonces la estructura entera sería
invariante en \rust{'a}.

\begin{minted}{rust}
use std::cell::UnsafeCell;

struct MiTipo<'a, 'b, T, U: 'a> {
    x: &'a U,               // Esto hace `MiTipo` covariante en 'a, y
                            // lo haría covariante en U también, pero
                            // U se usa más tarde.
    y: *const T,            // Covariante en T
    z: UnsafeCell<&'b f64>, // Invariante en 'b
    w: *mut U,              // Invariante en U, hace que el struct
                            // entero sea invariante.
}
\end{minted}

\rust{&'a T} tiene sentido que sea covariante en \rust{'a} porque
siempre se podría pasar un \rust{&'static T} en su lugar. Sin embargo, esto no
es el caso con \rust{&'a mut T}, o podrían producirse ciertos errores de
memoria. Por tanto, un tipo invariante será menos flexible que uno covariante
porque asume que podrían suceder los mismos errores que con \rust{&'a mut T}.
Estas conversiones de \lifetimes implícitas no podrán ocurrir si el tipo es
invariante.

El compilador no es lo suficientemente avanzado como para mencionar la varianza
en la descripción de sus errores. Únicamente sabe que si un tipo es invariante,
entonces algunos usos de sus \lifetimes son imposibles. Esto se está mejorando
en las futuras versiones~\cite{smarterchecker}, pero durante el desarrollo del
sistema plugins resultó muy complicado entender qué estaba sucediendo.

Se recomienda consultar los recursos adicionales listados al inicio del anexo,
que explican el concepto en detalle y con más ejemplos, dado que es un tema
especialmente complicado de entender.

\section{Resolviendo el problema}

Todo acabó reduciéndose a la única diferencia en la implementación del \trait
\rust{Ord}. \rust{RCow<'a, T>} implementa un \trait llamado
\rust{BorrowOwned<'a>} y \rust{Cow<'a, T>} implementa otro llamado
\rust{ToOwned}. Ambos \traits son iguales, excepto que en \rust{BorrowOwned<'a>}
se incluye funcionalidad adicional para \abistable. El problema no tiene que ver
con esta diferencia en funcionalidad, sino que \rust{BorrowOwned<'a>} es
genérico respecto a la \lifetime \rust{'a}, lo cual no es el caso de
\rust{ToOwned}.

Al implementar \rust{Ord}, se tenía que indicar que \rust{T: ToOwned} en
\rust{Cow} o \rust{T: BorrowOwned<'a>} en \rust{RCow}. El problema era que al
relacionar la \lifetime \rust{'a} de esta forma, estaba rompiendo una regla que
hacía a \rust{RCow} invariante en \rust{'a}. Tenemos que \rust{Cow<'a, T>} es
covariante en tanto \rust{T} como \rust{'a}, pero nuestro \rust{RCow<'a, T>} es
covariante en \rust{T} e invariante en \rust{'a}.

Adicionalmente, esta regla de covarianza específica tampoco está documentada
apropiadamente en Rust. Las guías únicamente explican el sistema de herencia de
varianzas, pero no que por asignar una \lifetime a un \trait su implementador
pasará a ser invariante. Para saber esto, uno tiene que referirse a la guía de
desarrollo del compilador, que sí que lo
menciona~\cite{associated_types_invariance}. Esto se ha reportado en el
repositorio de \namecite{nomicon}, el libro oficial de Rust donde debería
haberse incluido~\cite{nomicon_issue}.

Tras algunos experimentos míos~\cite{attempt_covariance}, discusión con el autor
de \abistable y con ayuda de Heinz, llegamos a un nuevo diseño para
\rust{RCow<'a, T>} que no involucraba \rust{BorrowOwned<'a>} y que por tanto era
covariante en \rust{'a}~\cite{abi_covandcontra}. Resultó que realmente, la
mayoría de tipos en \abistable eran invariantes, así que esos también tendrían
que arreglarse de formas distintas. La implementación final la llevó a cabo el
autor de \abistable y el arreglo se incluyó en la versión 0.11 de la librería.
