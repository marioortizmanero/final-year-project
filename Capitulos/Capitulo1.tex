% vim: spelllang=es

\chapter{Breve introducción a Rust}\label{ch:rust}

Dado que Rust es un lenguaje de programación que tan solo anunció su primera
versión en 2015, aún no es conocido por muchos desarrolladores. Este proyecto
requiere ser familiar con cómo funciona, por lo que en este capítulo se
introducirán los conceptos más básicos necesarios. Sí que se asume conocimiento
de lenguajes de propósito general, como C, C++, Python o Java.

Sin embargo, es posible que se omitan algunos conceptos o que algunas
explicaciones no sean completamente precisas por razones de simplicidad.
\textcite{rustfullbook} es el libro oficial para aprender Rust por completo,
pero es una lectura larga y posiblemente demasiado exhaustiva. Para mayor
brevedad, se recomienda leer \textcite{rustprofessionals},
\textcite{rustgentleintro} o \textcite{rust30min}.

La comunidad dispone de otros libros que explican aspectos más avanzados del
lenguaje en específico, como \unsafe o la programación asíncrona. En esos casos,
se recomienda leer \textcite{rustnomicon} y \textcite{rustasyncbook},
respectivamente.

\section{¿Qué es Rust?}

Rust es un lenguaje de programación de sistemas compilado y de propósito
general. Su objetivo es maximizar rendimiento y usabilidad, esto último
basándose en seguridad integrada en el lenguaje, en vez de en el \

\section{Primeros pasos}

Comenzando por el clásico ``Hola Mundo'', se incluyen algunos ejemplos de cómo
es la sintaxis de Rust más básica:

\begin{minted}{rust}
fn main() {
    println!("Hello World!");
}
\end{minted}

\code{main} es nuestra función principal, que invoca al macro \emph{println}
para escribir por pantalla. Esto se sabe porque, a diferencia de una llamada a
función, la invocación termina con una exclamación (\code{!}).

Los bloques básicos (\code{if}, \code{else}, \code{while}, \code{for}) son los
mismos que en otros lenguajes, con la introducción de \code{match}, que permite
extraer patrones.

\begin{minted}{rust}
fn factorial(i: u64) -> u64 {
    match i {
        0 => 1,
        n => n * factorial(n-1)
    }
}
\end{minted}

\section{Gestión de errores}

Panic

\code{Result<T>}

\section{}
