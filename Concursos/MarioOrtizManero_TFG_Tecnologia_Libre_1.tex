% vim: spelllang=es

% Página:
% https://osluz.unizar.es/
%
% Bases:
% https://osluz.unizar.es/sites/default/files/BasesConcursoTFG1ed.pdf

\documentclass[a4paper,12pt,twoside,hidelinks,openright]{article}

% \usepackage[T1]{fontenc}
\usepackage[spanish]{babel}
\usepackage[utf8]{inputenc}
\usepackage{hyperref}
\usepackage{enumitem}% http://ctan.org/pkg/enumitem

% Easier to read font
\usepackage{fourier} % For math
\usepackage{fontspec}
\setmainfont{Heuristica} % For the text
\setmonofont{Liberation Mono} % For the code

% Less margin between lists, otherwise after overriding \parskip and etc it's
% too much.

%	CONFIGURACIÓN DE PÁGINA

\setlength{\paperwidth}{21cm}          % Ancho de página
\setlength{\paperheight}{29,7cm}       % Alto de página
\setlength{\textwidth}{15.5cm}         % Ancho de zona con texto
\setlength{\textheight}{24.6cm}        % Ancho de zona con texto
\setlength{\topmargin}{-1.0cm}         % Margen superior
                                      
\setlength{\oddsidemargin}{0.46cm}     % Margen izquierdo 
\setlength{\evensidemargin}{0.46cm}    

% Less margin between lists, otherwise after overriding \parskip and etc it's
% too much.
\usepackage{enumitem}
\setlist{topsep=0pt}

\begin{document}

% PORTADA
\newpage

\title{I Concurso ``Trabajos Fin de Grado Relacionados con Tecnología Libre'' (TFGTL 2023)}
\date{}

\pagenumbering{Roman}

% Párrafos de forma más convencional, me parece más fácil leerlo así.
\begingroup
\setlength{\parskip}{\baselineskip}%
\setlength{\parindent}{0pt}%

\maketitle

\textbf{Autor}:\\
Mario Ortiz Manero

\textbf{Mentores}:\\
Matthias Wahl, Heinz N. Gies y Darach Ennis

\textbf{Director académico}:\\
Francisco Javier Fabra Caro

\textbf{Titulación y universidad}:\\
Grado en Ingeniería Informática de la Universidad de Zaragoza

\textbf{Título del trabajo fin de grado}:\\
Cargado dinámico de plugins en Rust en ausencia de estabilidad en la Interfaz Binaria de Aplicación

\textbf{Palabras clave}:\\
Rust, C, Tremor, Wayfair, Procesado de Eventos, Cargado Dinámico, Alto
Rendimiento, WebAssembly, eBPF, Lenguajes Interpretados, IPC

\textbf{Lugar y fecha de defensa}:\\
Edificio Ada Byron, EINA, C. María de Luna, 1, 50018 Zaragoza, España\\
12 de julio de 2022\\
Prueba adjuntada con el archivo
\texttt{convocatoria\_defensa\_tfg\_07\_07\_2022.pdf}.

\textbf{Nota del trabajo}:\\
9.0/10

\textbf{Difusión de los resultados y número de descargas, si corresponde}:\\
\setlength{\parskip}{0cm}%
\begin{itemize}[noitemsep, topsep=0pt]
    \item Visitas a los artículos escritos durante el proyecto
        (\url{https://nullderef.com/series/rust-plugins}) a 4 de febrero de
        2023:\\
        % 7948 + 6674 + 5664 + 4494 + 3331 + 1812 + 124
        30.047

    \item Contribuciones (\emph{pull requests} e \emph{issues}) realizadas para
        repositorios externos a Tremor:\\
        33

    \item Contribuciones realizadas para repositorios dentro de la organización
        de Tremor:\\
        13

    \item Presentación durante el proyecto, a 19 de enero de 2022, sobre el
        progreso hasta ese punto. Parte del evento ``LFX Mentorship
        Showcase'':\\
        \url{https://youtu.be/htLCyqY0kt0?t=3166}

    \item Presentación \emph{a posteriori}, a 18 de octubre de 2022, sobre la
        realización del proyecto respecto a la usabilidad del lenguaje de
        programación Rust. Parte del evento ``TremorCon 2022'':\\
        \url{https://nullderef.com/blog/tremorcon22/}

\end{itemize}
\setlength{\parskip}{\baselineskip}%

\textbf{Enlace al TFG}:\\
\setlength{\parskip}{0cm}%
\begin{itemize}[noitemsep, topsep=0pt]

    \item \emph{Con código fuente}:\\
        \url{https://github.com/marioortizmanero/final-year-project}

    \item \emph{Fuente oficial}:\\
        \url{https://deposita.unizar.es/record/68677}

\end{itemize}
\setlength{\parskip}{\baselineskip}%

\textbf{Resumen}:\\ La toma de decisiones de muchas empresas modernas se basa en
la recolección y análisis de datos de sus sistemas. En el caso de
\emph{Wayfair}, debía llevarse a cabo de forma eficiente en una escala masiva,
por lo que creó en 2018 su propia herramienta de procesado de eventos,
\emph{Tremor}. Esta mantuvo una perspectiva de alto rendimiento desde el inicio,
siendo escrita con el lenguaje de programación compilado Rust, concurrencia
asíncrona con hilos y SIMD.

Posteriormente, a través de la licencia \emph{Apache License 2.0}, Tremor se
lanzó en código libre y pasó a formar parte de la \emph{Cloud Native Computing
Foundation} (CNCF), organización fundada y gestionada por \emph{The Linux
Foundation} (LFX). Entre los múltiples beneficios que aportó esto, también le
abrió las puertas a participar en iniciativas como \emph{LFX Mentorship},
gracias a la cual se llevó a cabo este proyecto. En la plataforma de la
iniciativa se especifican tareas concretas que cualquier desarrollador puede
realizar, proporcionando a cambio un mentor que le guíe durante el proceso y una
ayuda monetaria.

La tarea escogida consistía en desarrollar un ``sistema de plugins'' para
Tremor. A medida que el programa evoluciona, sus tiempos de compilación crecen
y, consecuentemente, se deteriora la experiencia de desarrollo. Esto se puede
aliviar extrayendo la funcionalidad de su único binario a componentes más
pequeños que se puedan compilar independientemente (plugins).
Concretamente, se debía modificar Tremor de forma que el programa principal
fuese capaz de cargar los plugins como librerías dinámicas durante su ejecución
y de que, después, se pudieran comunicar entre sí. Dado que son binarios
completamente distintos, cambiar el código de un plugin no requiere recompilar
el de los demás, reduciendo así los tiempos de compilación drásticamente.

Existen gran cantidad de tecnologías disponibles para su desarrollo: lenguajes
interpretados, WebAssembly, eBPF, comunicación inter-proceso o cargado dinámico.
Sin embargo, muchas de ellas deben descartarse por no cumplir los estándares de
eficiencia de Tremor. Entre las alternativas restantes, se escoge cargado
dinámico por ser la más usable y popular.

El cargado dinámico es imposible con tipos y funciones declarados con Rust puro,
ya que la Interfaz Binaria de Aplicación (ABI) del lenguaje no es ``estable''.
Para agilizar su evolución en esta temprana etapa, prefiere no definir un
protocolo a través del cual diferentes binarios se puedan comunicar entre sí de
forma directa y determinista. Será necesario convertir los tipos al ABI de C ---
que sí se especifica rigurosamente en su estándar --- y viceversa.

Desafortunadamente, en el ámbito público y libre de los sistemas de plugins en
Rust, las guías y artículos de la comunidad, librerías existentes y herramientas
eran escasas o estaban desactualizadas. Gran parte del trabajo consiste en
intentar mejorar esta situación de diversas formas.

En primer lugar, se han realizado más de treinta contribuciones a diez
diferentes dependencias de Tremor. Mayoritariamente se tratan de \emph{pull
requests}, donde se intentaba dar soporte a su uso en un sistema de plugins o
mejorar su documentación. También se crearon varios \emph{issues} proponiendo
nuevas características que se podrían incluir en las librerías y debatiendo con
sus autores.

En segundo lugar, puesto que Tremor también es completamente libre, las más de
diez contribuciones realizadas están disponibles en sus repositorios. El diseño
se llevó a cabo dentro de su servidor de Discord público, tanto en formato
escrito como en videollamada.

En tercer lugar, se participó en eventos como \emph{LFX Mentorship Showcase},
donde algunos participantes de la iniciativa presentaban su experiencia hasta el
momento para futuros interesados. Incluso fue posible la asistencia presencial a
la conferencia KubeCon 2022 en Valencia como colaborador del proyecto, pudiendo
atender a charlas de organizaciones y empresas en el círculo de Tremor y conocer
a sus miembros.

Finalmente, durante la realización del proyecto, se han escrito seis artículos
con abundantes recursos (\url{https://nullderef.com/series/rust-plugins/}),
entrando en profundo detalle sobre el diseño del sistema, la investigación
realizada y las decisiones tomadas. Su publicación e interacción con la
comunidad ha resultado en un total de treinta mil visitas hasta el momento.

A lo largo de los catorce meses transcurridos hasta la defensa, se colaboró muy
cercanamente con el equipo principal de Tremor, contratado por Wayfair. Su ayuda
fue esencial para comprender el funcionamiento interno del programa y para dar
consejo cuando fuera necesario. La inestabilidad del ABI y falta de recursos
para Rust causaron un aumento significativo de complejidad respecto al plan
original. Por tanto, aunque funcional, la implementación no alcanza alguno de
los objetivos iniciales, principalmente relacionados con el rendimiento. Sin
embargo, sirve como una base sólida para futuras versiones de Tremor que sí que
lo incluyan en producción, continuando así la evolución del trabajo.

\endgroup

\end{document}
