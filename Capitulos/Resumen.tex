% vim: spelllang=es

\begin{center}
{\LARGE \bfseries RESUMEN}

\vspace{2.5cm}
\end{center}

La estrategia de muchas empresas modernas como Wayfair se basa en la recolección
y análisis de datos de su sistema. Para llevarlo a cabo de forma eficiente en
una escala masiva es necesario el uso de herramientas de alto rendimiento, como
Tremor.

Tremor está escrito en Rust. A medida que evoluciona, sus tiempos de compilación
crecen, y consecuentemente se deteriora la experiencia de desarrollo. Esto se
puede aliviar implementando un sistema de plugins que divide su único binario en
componentes más pequeños compilables independientemente.

Existen múltiples posibles tecnologías para su desarrollo: lenguajes
interpretados, WebAssembly, eBPF, comunicación inter-proceso, o cargado
dinámico. Sin embargo, gran parte de ellas deben descartarse por no cumplir los
requisitos de eficiencia de Tremor. Entre las alternativas restantes, se escoge
cargado dinámico por ser la más usable y popular.

El cargado dinámico es imposible con tipos y funciones declarados con Rust puro,
ya que su Interfaz Binaria de Aplicación (ABI) no es estable. Será necesario
traducir los tipos al ABI de C, que sí es estable, y viceversa. Para facilitar
el proceso, se pueden aprovechar librerías existentes y herramientas del
lenguaje como macros procedurales.

Dado que el cargado dinámico en Rust es un ecosistema muy nuevo,, es necesario
contribuir en código abierto a gran cantidad de sus dependencias para
implementar las funcionalidades necesarias para un sistema de plugins.

La complejidad del proyecto incrementó significativamente respecto al plan
original, por lo que aunque funcional, no alcanza alguno de los objetivos
iniciales, principalmente relacionados con el rendimiento. Sin embargo, sirve
como una buena base para futuras versiones de Tremor que sí que lo incluyan,
y continuará evolucionando con el programa.
